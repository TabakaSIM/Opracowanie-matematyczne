\documentclass[12pt,a4paper]{article}
\usepackage[utf8]{inputenc} %polskie znaki
\usepackage[T1]{fontenc}	%polskie znaki
\usepackage{amsmath}		%matematyczne znaczki :3
\usepackage{enumerate}		%Dodatkowe opcje do funkcji enumerate
\usepackage{geometry} 		%Ustawianie marginesow
\usepackage{graphicx}		%Grafika
\usepackage{wrapfig}		%Grafika obok textu
\usepackage{float}			%Allows H in fugire
\usepackage{hyperref}		%Allows hyperlinks
%\pagestyle{empty} 			%usuwa nr strony
\usepackage{todonotes}		%Todo notatki
\usepackage{lipsum}         %Lorem text
\usepackage{ntheorem}   	% for theorem-like environments
\usepackage{mdframed}   	% for framing
\usepackage{subcaption}		% subfigure (image placing)
\usepackage{pdfcomment}		% Komentarze (z bazowego pdf'a)
\usepackage{xparse}			% New commands with optional arguments
\usepackage{ifthen}			% If then - funkcje!
\usepackage{expl3}			% Deklarowanie zmiennych

\newgeometry{tmargin=2cm, bmargin=2cm, lmargin=2cm, rmargin=2cm} 

%Counter commands{
	\newcounter{counter} % Creates a new counter
	\setcounter{counter}{1} % Sets the counter to 5
	\newcommand{\counter}[1]{
		\arabic{#1} \stepcounter{#1} 
	}
	\newcommand{\counterreset}[1]{\setcounter{#1}{1}}
	%}

%Define styles{
	\theoremstyle{break}
	\theoreminframepreskip{0.5cm}
	\theoremheaderfont{\bfseries}
	\newmdtheoremenv[%
	linecolor=white,%
	innertopmargin=\topskip,
	shadowsize=0,%
	innertopmargin=5,%
	innerbottommargin=5,%
	leftmargin=10,%
	rightmargin=10,%
	backgroundcolor=gray!20,%
	innertopmargin=0pt,%
	ntheorem]{zad}{Zadanie}
	
	\mdfdefinestyle{zadanie}{
		linecolor=white,%
		innertopmargin=5,%
		innerbottommargin=5,%
		leftmargin=10,%
		rightmargin=10,%
		backgroundcolor=gray!20,%
		innertopmargin=8,
		innerbottommargin=8,
		skipabove = 5,
	}
	\mdfdefinestyle{wzor}{
		linecolor=cyan,%
		linewidth=2pt,%
		innertopmargin=8,
		innerbottommargin=8,
		leftmargin=10,%
		rightmargin=10,%
		backgroundcolor = white, 
		fontcolor = black,
		skipabove = 5,
		skipbelow = 5,
	}
	%}

%Zadania templatex%{
	\newcommand{\Wzor}[1]{
		\begin{mdframed}[style=wzor]
			\centering #1
		\end{mdframed}
	}
	\newcommand{\ZadanieTextowe}[1]{
		\begin{mdframed}[style=zadanie]
			\textbf{Zadanie \counter{counter} } \\
			#1
		\end{mdframed}
	}
	\newcommand{\Zadanie}[2]{
		\ZadanieTextowe{#1}
		#2
	}
	\newcommand{\ZadanieABCD}[6]{
		\ZadanieTextowe{#1}
		#2 \\\\
		\begin{tabular}{p{7cm} p{7cm}}
			\textbf{A. }#3&
			\textbf{B. }#4\\\\
			\textbf{C. }#5&
			\textbf{D. }#6\\
		\end{tabular}
	}
	\newcommand{\ZadanieABCDEF}[8]{
		\ZadanieTextowe{#1}
		#2 \\\\
		\begin{tabular}{p{7cm} p{7cm}}
			\textbf{A. }#3&
			\textbf{B. }#4\\\\
			\textbf{C. }#5&
			\textbf{D. }#6\\\\
			\textbf{E. }#7&
			\textbf{F. }#8\\\\
		\end{tabular}
	}
	\newcommand{\Zadanietwoxtwo}[5]{
		\ZadanieTextowe{#1}
		\begin{tabular}{p{7cm} p{7cm}}
			\textbf{a)} #2&
			\textbf{b)} #3\\\\
			\textbf{c)} #4&
			\textbf{d)} #5\\\\
		\end{tabular}
	}
	\newcommand{\Zadanietwoxthree}[7]{
		\ZadanieTextowe{#1}
		\begin{tabular}{p{7cm} p{7cm}}
			\textbf{a)} #2&
			\textbf{b)} #3\\\\
			\textbf{c)} #4&
			\textbf{d)} #5\\\\
			\textbf{e)} #6&
			\textbf{f)} #7\\\\
		\end{tabular}
	}
	\newcommand{\Zadanietwoxfour}[9]{
		\ZadanieTextowe{#1}
		\begin{tabular}{p{7cm} p{7cm}}
			\textbf{a)} #2&
			\textbf{b)} #3\\\\
			\textbf{c)} #4&
			\textbf{d)} #5\\\\
			\textbf{e)} #6&
			\textbf{f)} #7\\\\
			\textbf{g)} #8&
			\textbf{h)} #9\\\\
		\end{tabular}
	}
	%}

\begin{document}
	\Zadanie{Wykonać działania na macierzach}{
		\begin{enumerate}[a)]
			\item $\begin{bmatrix}
				2 & -3  & 7\\
				4 & -11 & -9\\
			\end{bmatrix} + \begin{bmatrix}
				4 & 8 & -11\\
				5 & 0 & -2\\
			\end{bmatrix}=$
			\item $\begin{bmatrix}
				2 & -3 & 7\\
				4 & -11 & -9
			\end{bmatrix} + \begin{bmatrix}
				4 & 8 & -11\\
				5 & 0 & -2
			\end{bmatrix} =$
			
			\item $\begin{bmatrix}
				1 & 2\\
				-5 & 4
			\end{bmatrix} + \begin{bmatrix}
				3 & -2\\
				6 & 1
			\end{bmatrix} =$
			\item $\begin{bmatrix}
				7 & 0 & -3\\
				1 & 2 & 5
			\end{bmatrix} - \begin{bmatrix}
				4 & 8 & -1\\
				2 & -3 & 0
			\end{bmatrix} =$
			
			\item $\begin{bmatrix}
				5 & -1\\
				9 & 6
			\end{bmatrix} - \begin{bmatrix}
				2 & 4\\
				3 & 7
			\end{bmatrix} =$
		
			\item $\begin{bmatrix}
				1 & 2\\
				3 & 4
			\end{bmatrix} \cdot \begin{bmatrix}
				5 & 6\\
				7 & 8
			\end{bmatrix} =$
			
			\item $\begin{bmatrix}
				2 & 0\\
				-1 & 3
			\end{bmatrix} \cdot \begin{bmatrix}
				4 & -2\\
				1 & 5
			\end{bmatrix} =$
			
			\item $\left( \begin{bmatrix}
				1 & 2\\
				3 & 4
			\end{bmatrix} + \begin{bmatrix}
				0 & 1\\
				-1 & 2
			\end{bmatrix} \right) \cdot \begin{bmatrix}
				2 & 0\\
				0 & 2
			\end{bmatrix} =$
			
			\item $\begin{bmatrix}
				5 & -2\\
				0 & 1
			\end{bmatrix} \cdot \begin{bmatrix}
				1 & 0\\
				3 & 4
			\end{bmatrix} - \begin{bmatrix}
				2 & 2\\
				1 & 1
			\end{bmatrix} =$
			
			\item $\begin{bmatrix}
				1 & 2 & 3\\
				4 & 5 & 6
			\end{bmatrix} \cdot \begin{bmatrix}
				7 & 8\\
				9 & 10\\
				11 & 12
			\end{bmatrix} - \begin{bmatrix}
				-1 & 4\\
				3 & 0
			\end{bmatrix} + \begin{bmatrix}
				2 & -5\\
				7 & 6
			\end{bmatrix} =$
			
			\item $\begin{bmatrix}
				10 & 5 & -2\\
				0 & 4 & 7\\
				3 & 6 & 1
			\end{bmatrix} - \left( \begin{bmatrix}
				1 & 2\\
				3 & 4\\
				5 & 6
			\end{bmatrix} \cdot \begin{bmatrix}
				1 & 0 & -1\\
				2 & 3 & 4
			\end{bmatrix} \right) =$
			
			
		\end{enumerate}
	}
	\newpage
	\Zadanie{Dane są macierze
		$$
		A=\begin{bmatrix}
			3 & -1\\
			0 & 5
		\end{bmatrix},\quad
		B=\begin{bmatrix}
			2 & 4\\
			-3 & 1\\
			0 & 6
		\end{bmatrix},\quad
		C=\begin{bmatrix}
			7 & -2 & 0\\
			1 & 3 & 5
		\end{bmatrix}$$
		$$
		D=\begin{bmatrix}
			1 & 0 & -1\\
			4 & 5 & 6\\
			-2 & 3 & 2
		\end{bmatrix},\quad
		E=\begin{bmatrix}
			8 & -3\\
			0 & 1\\
			7 & 2
		\end{bmatrix},\quad
		F=\begin{bmatrix}
			-4 & 9\\
			2 & -7
		\end{bmatrix}
		$$
		Obliczyć (o ile to możliwe; w podpunkcie i) zastanów się, które mnożenie trzeba wykonać najpierw albo czy jest to dowolne):
	}{\begin{enumerate}[a)]
		\item $A+F=$
		\item $A\cdot B=$
		\item $B\cdot A=$
		\item $B\cdot C=$
		\item $C\cdot B=$
		\item $D\cdot B + E=$
		\item $F+A\cdot C=$
		\item $C \cdot E \cdot A=$
		\item $C \cdot D \cdot B=$ 
\end{enumerate}}

\end{document}