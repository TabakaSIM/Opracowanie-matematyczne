\documentclass[12pt,a4paper]{article}
\usepackage[utf8]{inputenc} %polskie znaki
\usepackage[T1]{fontenc}	%polskie znaki
\usepackage{amsmath}		%matematyczne znaczki :3
\usepackage{enumerate}		%Dodatkowe opcje do funkcji enumerate
\usepackage{geometry} 		%Ustawianie marginesow
\usepackage{graphicx}		%Grafika
\usepackage{wrapfig}		%Grafika obok textu
\usepackage{float}			%Allows H in fugire
\usepackage{hyperref}		%Allows hyperlinks
%\pagestyle{empty} 			%usuwa nr strony
\usepackage{todonotes}		%Todo notatki
\usepackage{lipsum}         %Lorem text
\usepackage{ntheorem}   	% for theorem-like environments
\usepackage{mdframed}   	% for framing
\usepackage{subcaption}		% subfigure (image placing)
\usepackage{pdfcomment}		% Komentarze (z bazowego pdf'a)
\usepackage{xparse}			% New commands with optional arguments
\usepackage{ifthen}			% If then - funkcje!
\usepackage{expl3}			% Deklarowanie zmiennych

\newgeometry{tmargin=2cm, bmargin=2cm, lmargin=2cm, rmargin=2cm} 

%Counter commands{
	\newcounter{counter} % Creates a new counter
	\setcounter{counter}{1} % Sets the counter to 5
	\newcommand{\counter}[1]{
		\arabic{#1} \stepcounter{#1} 
	}
	\newcommand{\counterreset}[1]{\setcounter{#1}{1}}
	%}

%Define styles{
	\theoremstyle{break}
	\theoreminframepreskip{0.5cm}
	\theoremheaderfont{\bfseries}
	\newmdtheoremenv[%
	linecolor=white,%
	innertopmargin=\topskip,
	shadowsize=0,%
	innertopmargin=5,%
	innerbottommargin=5,%
	leftmargin=10,%
	rightmargin=10,%
	backgroundcolor=gray!20,%
	innertopmargin=0pt,%
	ntheorem]{zad}{Zadanie}
	
	\mdfdefinestyle{zadanie}{
		linecolor=white,%
		innertopmargin=5,%
		innerbottommargin=5,%
		leftmargin=10,%
		rightmargin=10,%
		backgroundcolor=gray!20,%
		innertopmargin=8,
		innerbottommargin=8,
		skipabove = 5,
	}
	\mdfdefinestyle{wzor}{
		linecolor=cyan,%
		linewidth=2pt,%
		innertopmargin=8,
		innerbottommargin=8,
		leftmargin=10,%
		rightmargin=10,%
		backgroundcolor = white, 
		fontcolor = black,
		skipabove = 5,
		skipbelow = 5,
	}
	%}

%Zadania templatex%{
	\newcommand{\Wzor}[1]{
		\begin{mdframed}[style=wzor]
			\centering #1
		\end{mdframed}
	}
	\newcommand{\ZadanieTextowe}[1]{
		\begin{mdframed}[style=zadanie]
			\textbf{Zadanie \counter{counter} } \\
			#1
		\end{mdframed}
	}
	\newcommand{\Zadanie}[2]{
		\ZadanieTextowe{#1}
		#2
	}
	\newcommand{\ZadanieABCD}[6]{
		\ZadanieTextowe{#1}
		#2 \\\\
		\begin{tabular}{p{7cm} p{7cm}}
			\textbf{A. }#3&
			\textbf{B. }#4\\\\
			\textbf{C. }#5&
			\textbf{D. }#6\\
		\end{tabular}
	}
	\newcommand{\ZadanieABCDEF}[8]{
		\ZadanieTextowe{#1}
		#2 \\\\
		\begin{tabular}{p{7cm} p{7cm}}
			\textbf{A. }#3&
			\textbf{B. }#4\\\\
			\textbf{C. }#5&
			\textbf{D. }#6\\\\
			\textbf{E. }#7&
			\textbf{F. }#8\\\\
		\end{tabular}
	}
	\newcommand{\Zadanietwoxtwo}[5]{
		\ZadanieTextowe{#1}
		\begin{tabular}{p{7cm} p{7cm}}
			\textbf{a)} #2&
			\textbf{b)} #3\\\\
			\textbf{c)} #4&
			\textbf{d)} #5\\\\
		\end{tabular}
	}
	\newcommand{\Zadanietwoxthree}[7]{
		\ZadanieTextowe{#1}
		\begin{tabular}{p{7cm} p{7cm}}
			\textbf{a)} #2&
			\textbf{b)} #3\\\\
			\textbf{c)} #4&
			\textbf{d)} #5\\\\
			\textbf{e)} #6&
			\textbf{f)} #7\\\\
		\end{tabular}
	}
	\newcommand{\Zadanietwoxfour}[9]{
		\ZadanieTextowe{#1}
		\begin{tabular}{p{7cm} p{7cm}}
			\textbf{a)} #2&
			\textbf{b)} #3\\\\
			\textbf{c)} #4&
			\textbf{d)} #5\\\\
			\textbf{e)} #6&
			\textbf{f)} #7\\\\
			\textbf{g)} #8&
			\textbf{h)} #9\\\\
		\end{tabular}
	}
	%}

\begin{document}

\Zadanie{Wyknać działania:}{
	\begin{enumerate}[a)]
		\item $\begin{bmatrix}
			1 & 2 \\
			3 & 4
		\end{bmatrix}^T$
		\item $\begin{bmatrix}
			1 & 0 & -1 \\
			2 & 3 & 4
		\end{bmatrix}^T$
		\item $\begin{bmatrix}
			0 & -1 & 2 \\
			-3 & 4 & -5 \\
			6 & -7 & 8
		\end{bmatrix}^T$
	
		\item $\begin{bmatrix}
			5 & 6 & 7 \\
			8 & 9 & 10
		\end{bmatrix}^T$
	
		\item $\begin{bmatrix}
			-5 & 0 \\
			7 & 8
		\end{bmatrix}^T$
	
		\item $\begin{bmatrix}
			1 & 2 \\
			3 & 4 \\
			5 & 6 \\
			7 & 8
		\end{bmatrix}^T$
		
		\item $\begin{bmatrix}
			1 & 2 & 3 \\
			4 & 5 & 6 \\
			7 & 8 & 9
		\end{bmatrix}^T$
		
		\item $\begin{bmatrix}
			0 & -1 \\
			2 & -3 \\
			4 & -5 \\
			6 & -7
		\end{bmatrix}^T$
		
		
	\end{enumerate}
}\newpage

\Zadanietwoxfour{Sprawdzić niezależność wektorów}{\[
	\mathbf{v}_1 =
	\begin{bmatrix}
		1 \\
		2
	\end{bmatrix}, \quad
	\mathbf{v}_2 =
	\begin{bmatrix}
		3 \\
		4
	\end{bmatrix}
	\]
}{\[
\mathbf{v}_1 =
\begin{bmatrix}
	1 \\
	2
\end{bmatrix}, \quad
\mathbf{v}_2 =
\begin{bmatrix}
	0 \\
	1
\end{bmatrix}, \quad
\mathbf{v}_3 =
\begin{bmatrix}
	1 \\
	1
\end{bmatrix}
\]
}{\[
\mathbf{v}_1 =
\begin{bmatrix}
	1 \\
	0 \\
	2
\end{bmatrix}, \quad
\mathbf{v}_2 =
\begin{bmatrix}
	3 \\
	1 \\
	4
\end{bmatrix}
\]
}{\[
\mathbf{v}_1 =
\begin{bmatrix}
	1 \\
	2 \\
	1
\end{bmatrix}, \quad
\mathbf{v}_2 =
\begin{bmatrix}
	3 \\
	1 \\
	-2
\end{bmatrix}, \quad
\mathbf{v}_3 =
\begin{bmatrix}
	3 \\
	-2 \\
	1
\end{bmatrix}
\]
}{\[
\mathbf{v}_1 =
\begin{bmatrix}
	1 \\
	2 \\
	1
\end{bmatrix}, \quad
\mathbf{v}_2 =
\begin{bmatrix}
	5 \\
	2 \\
	-2
\end{bmatrix}, \quad
\mathbf{v}_3 =
\begin{bmatrix}
	6 \\
	-4 \\
	-8
\end{bmatrix}
\]
}{\[
\mathbf{v}_1 =
\begin{bmatrix}
	1 \\
	0 \\
	0
\end{bmatrix}, \quad
\mathbf{v}_2 =
\begin{bmatrix}
	0 \\
	1 \\
	0
\end{bmatrix}, \quad
\mathbf{v}_3 =
\begin{bmatrix}
	0 \\
	0 \\
	1
\end{bmatrix}, \quad
\mathbf{v}_4 =
\begin{bmatrix}
	1 \\
	1 \\
	1
\end{bmatrix}
\]
}{\[
\mathbf{v}_1 =
\begin{bmatrix}
	1 \\
	2 \\
	0 \\
	0
\end{bmatrix}, \quad
\mathbf{v}_2 =
\begin{bmatrix}
	5 \\
	1 \\
	0 \\
	0
\end{bmatrix}, \quad
\mathbf{v}_3 =
\begin{bmatrix}
	-3 \\
	-2 \\
	1 \\
	0
\end{bmatrix}
\]
}{\[
\mathbf{v}_1 =
\begin{bmatrix}
	1 \\
	2 \\
	3 \\
	4
\end{bmatrix}, \quad
\mathbf{v}_2 =
\begin{bmatrix}
	2 \\
	4 \\
	6 \\
	8
\end{bmatrix}, \quad
\mathbf{v}_3 =
\begin{bmatrix}
	3 \\
	6 \\
	9 \\
	12
\end{bmatrix}
\]
}
	
	\Zadanie{Rozwiać równania macierzowe}{
		\begin{enumerate}[a)]
		\item \[
		X =
		\begin{bmatrix}
			1 & 2 \\
			3 & 4
		\end{bmatrix}
		\cdot
		\begin{bmatrix}
			0 & 1 \\
			-1 & 0
		\end{bmatrix}
		-
		\begin{bmatrix}
			5 & 6 \\
			7 & 8
		\end{bmatrix}
		\]
		
		\item \[
		X -
		\begin{bmatrix}
			1 & 2 \\
			3 & 4 \\
			5 & 6
		\end{bmatrix}
		=
		\begin{bmatrix}
			7 & 8 \\
			9 & 10 \\
			11 & 12
		\end{bmatrix}
		-
		\begin{bmatrix}
			2 & 1 \\
			0 & -1 \\
			3 & 3
		\end{bmatrix}
		\]
		
		\item \[
		X -
		\begin{bmatrix}
			1 & 2 \\
			3 & 4
		\end{bmatrix}
		=
		\begin{bmatrix}
			1 & 0 & -1 \\
			2 & 3 & 1
		\end{bmatrix}^T
		\cdot
		\begin{bmatrix}
			4 & 1 & 0 \\
			0 & -2 & 3 \\
			5 & 3 & 1
		\end{bmatrix}
		\]
		
		
		\item \[
		X +
		\begin{bmatrix}
			1 & 2 \\
			3 & 4
		\end{bmatrix}
		=
		\begin{bmatrix}
			1 & 0 & -1 \\
			2 & 3 & 1
		\end{bmatrix}
		\cdot
		\begin{bmatrix}
			1 & 2 \\
			0 & -1 \\
			3 & 0
		\end{bmatrix}
		\]
		
		\item \[
		\begin{bmatrix}
			1 & 0 & 2 \\
			-1 & 3 & 1
		\end{bmatrix}^\mathrm{T}
		\cdot
		\begin{bmatrix}
			4 & -2 & 0 \\
			1 & 5 & 3
		\end{bmatrix}
		- X =
		\begin{bmatrix}
			2 & 1 & 0 \\
			-3 & 4 & 5 \\
			6 & -2 & 1
		\end{bmatrix}
		\]
		
		\end{enumerate}
	}
	\Zadanie{Obliczyć wyznacznik macierzy}{
		\begin{enumerate}[a)]
			\item Macierze wymariu 2x2:
			\[
			\begin{bmatrix}
				1 & 2 \\
				3 & 4
			\end{bmatrix}
			, \quad
			\begin{bmatrix}
				0 & 5 \\
				-1 & 3
			\end{bmatrix}
			, \quad
			\begin{bmatrix}
				2 & 0 \\
				0 & 2
			\end{bmatrix}
			, \quad
			\begin{bmatrix}
				4 & 1 \\
				2 & 2
			\end{bmatrix}
			\]
			
			\item Macierze wymariu 3x3:
			\[
			\begin{bmatrix}
				1 & 0 & 2 \\
				-1 & 3 & 1 \\
				0 & 4 & 5
			\end{bmatrix}
			, \quad
			\begin{bmatrix}
				2 & -1 & 0 \\
				1 & 3 & 4 \\
				0 & 2 & 1
			\end{bmatrix}
			, \quad
			\begin{bmatrix}
				0 & 1 & 2 \\
				3 & 0 & 4 \\
				1 & 5 & 0
			\end{bmatrix}
			, \quad
			\begin{bmatrix}
				1 & 2 & 3 \\
				0 & 1 & 4 \\
				5 & 6 & 0
			\end{bmatrix}
			\]
			
			\item Macierze wymariu 4×4:
			\[
			\begin{bmatrix}
				1 & 0 & 2 & 3 \\
				0 & 1 & 4 & 5 \\
				6 & 7 & 0 & 8 \\
				9 & 10 & 11 & 0
			\end{bmatrix}
			, \quad
			\begin{bmatrix}
				2 & 3 & 1 & 0 \\
				0 & 1 & 4 & 2 \\
				1 & 0 & 5 & 3 \\
				4 & 2 & 0 & 1
			\end{bmatrix}
			\]
			
			\item Dla podanej macierzy $B$ oblicz wyznacznik macierzy $B^TB$ oraz $BB^T$
			\[
			\begin{bmatrix}
				1 & 2 & 3 \\
				4 & 5 & 6
			\end{bmatrix}
			, \quad
			\begin{bmatrix}
				0 & 1 & -1 \\
				2 & 0 & 3
			\end{bmatrix}
			\]
		\end{enumerate}
		
	}
	
\end{document}