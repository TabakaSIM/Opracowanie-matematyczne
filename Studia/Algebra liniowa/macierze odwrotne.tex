\documentclass[12pt,a4paper]{article}
\usepackage[utf8]{inputenc} %polskie znaki
\usepackage[T1]{fontenc}	%polskie znaki
\usepackage{amsmath}		%matematyczne znaczki :3
\usepackage{enumerate}		%Dodatkowe opcje do funkcji enumerate
\usepackage{geometry} 		%Ustawianie marginesow
\usepackage{graphicx}		%Grafika
\usepackage{wrapfig}		%Grafika obok textu
\usepackage{float}			%Allows H in fugire
\usepackage{hyperref}		%Allows hyperlinks
%\pagestyle{empty} 			%usuwa nr strony
\usepackage{todonotes}		%Todo notatki
\usepackage{lipsum}         %Lorem text
\usepackage{ntheorem}   	% for theorem-like environments
\usepackage{mdframed}   	% for framing
\usepackage{subcaption}		% subfigure (image placing)
\usepackage{pdfcomment}		% Komentarze (z bazowego pdf'a)
\usepackage{xparse}			% New commands with optional arguments
\usepackage{ifthen}			% If then - funkcje!
\usepackage{expl3}			% Deklarowanie zmiennych

\newgeometry{tmargin=2cm, bmargin=2cm, lmargin=2cm, rmargin=2cm} 

%Counter commands{
	\newcounter{counter} % Creates a new counter
	\setcounter{counter}{1} % Sets the counter to 5
	\newcommand{\counter}[1]{
		\arabic{#1} \stepcounter{#1} 
	}
	\newcommand{\counterreset}[1]{\setcounter{#1}{1}}
	%}

%Define styles{
	\theoremstyle{break}
	\theoreminframepreskip{0.5cm}
	\theoremheaderfont{\bfseries}
	\newmdtheoremenv[%
	linecolor=white,%
	innertopmargin=\topskip,
	shadowsize=0,%
	innertopmargin=5,%
	innerbottommargin=5,%
	leftmargin=10,%
	rightmargin=10,%
	backgroundcolor=gray!20,%
	innertopmargin=0pt,%
	ntheorem]{zad}{Zadanie}
	
	\mdfdefinestyle{zadanie}{
		linecolor=white,%
		innertopmargin=5,%
		innerbottommargin=5,%
		leftmargin=10,%
		rightmargin=10,%
		backgroundcolor=gray!20,%
		innertopmargin=8,
		innerbottommargin=8,
		skipabove = 5,
	}
	\mdfdefinestyle{wzor}{
		linecolor=cyan,%
		linewidth=2pt,%
		innertopmargin=8,
		innerbottommargin=8,
		leftmargin=10,%
		rightmargin=10,%
		backgroundcolor = white, 
		fontcolor = black,
		skipabove = 5,
		skipbelow = 5,
	}
	%}

%Zadania templatex%{
	\newcommand{\Wzor}[1]{
		\begin{mdframed}[style=wzor]
			\centering #1
		\end{mdframed}
	}
	\newcommand{\ZadanieTextowe}[1]{
		\begin{mdframed}[style=zadanie]
			\textbf{Zadanie \counter{counter} } \\
			#1
		\end{mdframed}
	}
	\newcommand{\Zadanie}[2]{
		\ZadanieTextowe{#1}
		#2
	}
	\newcommand{\ZadanieABCD}[6]{
		\ZadanieTextowe{#1}
		#2 \\\\
		\begin{tabular}{p{7cm} p{7cm}}
			\textbf{A. }#3&
			\textbf{B. }#4\\\\
			\textbf{C. }#5&
			\textbf{D. }#6\\
		\end{tabular}
	}
	\newcommand{\ZadanieABCDEF}[8]{
		\ZadanieTextowe{#1}
		#2 \\\\
		\begin{tabular}{p{7cm} p{7cm}}
			\textbf{A. }#3&
			\textbf{B. }#4\\\\
			\textbf{C. }#5&
			\textbf{D. }#6\\\\
			\textbf{E. }#7&
			\textbf{F. }#8\\\\
		\end{tabular}
	}
	\newcommand{\Zadanietwoxtwo}[5]{
		\ZadanieTextowe{#1}
		\begin{tabular}{p{7cm} p{7cm}}
			\textbf{a)} #2&
			\textbf{b)} #3\\\\
			\textbf{c)} #4&
			\textbf{d)} #5\\\\
		\end{tabular}
	}
	\newcommand{\Zadanietwoxthree}[7]{
		\ZadanieTextowe{#1}
		\begin{tabular}{p{7cm} p{7cm}}
			\textbf{a)} #2&
			\textbf{b)} #3\\\\
			\textbf{c)} #4&
			\textbf{d)} #5\\\\
			\textbf{e)} #6&
			\textbf{f)} #7\\\\
		\end{tabular}
	}
	\newcommand{\Zadanietwoxfour}[9]{
		\ZadanieTextowe{#1}
		\begin{tabular}{p{7cm} p{7cm}}
			\textbf{a)} #2&
			\textbf{b)} #3\\\\
			\textbf{c)} #4&
			\textbf{d)} #5\\\\
			\textbf{e)} #6&
			\textbf{f)} #7\\\\
			\textbf{g)} #8&
			\textbf{h)} #9\\\\
		\end{tabular}
	}
	%}

\begin{document}
	
	\Zadanie{Wyznaczyć rząd macierzy}{\begin{enumerate}[a)]
			\item 
			$\displaystyle A = 
			\begin{bmatrix}
				1 & 2 \\
				3 & 6
			\end{bmatrix}$
			
			\item 
			$\displaystyle B = 
			\begin{bmatrix}
				1 & 0 \\
				0 & 1
			\end{bmatrix}$
			
			\item 
			$\displaystyle C = 
			\begin{bmatrix}
				1 & 2 & 3 \\
				4 & 5 & 6
			\end{bmatrix}$
			
			\item 
			$\displaystyle D = 
			\begin{bmatrix}
				1 & 2 & 3 \\
				2 & 4 & 6 \\
				3 & 6 & 9
			\end{bmatrix}$
			
			\item 
			$\displaystyle E = 
			\begin{bmatrix}
				0 & 1 & 2 \\
				0 & 0 & 0 \\
				0 & 0 & 1
			\end{bmatrix}$
			
			\item 
			$\displaystyle F = 
			\begin{bmatrix}
				1 & 2 & 3 & 4 \\
				0 & 1 & 2 & 3 \\
				0 & 0 & 1 & 2
			\end{bmatrix}$
			
			\item 
			$\displaystyle G = 
			\begin{bmatrix}
				2 & 4 & 1 \\
				0 & 0 & 0 \\
				1 & 2 & 0
			\end{bmatrix}$
			
			\item 
			$\displaystyle H = 
			\begin{bmatrix}
				1 & 2 & 3 & 4 \\
				2 & 4 & 6 & 8 \\
				0 & 0 & 0 & 0 \\
				1 & 0 & 1 & 0
			\end{bmatrix}$
	\end{enumerate}}
	
	\Zadanie{Wyznaczyć macierz odwrotną do podanej macierzy:}{\begin{enumerate}[a)]
			\item 
			$\displaystyle A = 
			\begin{bmatrix}
				2 & 3 \\
				1 & 4
			\end{bmatrix}$
			
			\item 
			$\displaystyle B = 
			\begin{bmatrix}
				5 & -2 \\
				7 & 1
			\end{bmatrix}$
			
			\item 
			$\displaystyle C = 
			\begin{bmatrix}
				0 & 1 \\
				-1 & 2
			\end{bmatrix}$
			
			\item 
			$\displaystyle D = 
			\begin{bmatrix}
				1 & 2 & 3 \\
				0 & 1 & 4 \\
				5 & 6 & 0
			\end{bmatrix}$
			
			\item 
			$\displaystyle E = 
			\begin{bmatrix}
				3 & 0 & 2 \\
				2 & 0 & -2 \\
				0 & 1 & 1
			\end{bmatrix}$
			
			\item 
			$\displaystyle F = 
			\begin{bmatrix}
				4 & 7 & 2 \\
				3 & 6 & 1 \\
				2 & 5 & 3
			\end{bmatrix}$
			
			\item 
			$\displaystyle G = 
			\begin{bmatrix}
				1 & 0 & 2 & -1 \\
				3 & 0 & 0 & 5 \\
				2 & 1 & 4 & -3 \\
				1 & 0 & 5 & 0
			\end{bmatrix}$
			
			\item 
			$\displaystyle H = 
			\begin{bmatrix}
				2 & 0 & 1 & 3 \\
				1 & 1 & 0 & 0 \\
				0 & 2 & 1 & 1 \\
				4 & 0 & 1 & 2
			\end{bmatrix}$
	\end{enumerate}}

	\Zadanie{Rozwiązać równania macierzowe}{\begin{enumerate}[a)]
			\item 
			$A X = B$, \quad gdzie
			$A = \begin{bmatrix}
				2 & 1 \\
				3 & 4
			\end{bmatrix}, \quad
			B = \begin{bmatrix}
				1 \\
				2
			\end{bmatrix}$
			
			\item 
			$X A = B$, \quad gdzie
			$A = \begin{bmatrix}
				1 & 0 \\
				-2 & 3
			\end{bmatrix}, \quad
			B = \begin{bmatrix}
				4 & 1 \\
				2 & 0
			\end{bmatrix}$
			
			\item 
			$A X = C$, \quad gdzie
			$A = \begin{bmatrix}
				1 & 2 & 3 \\
				0 & 1 & 4 \\
				5 & 6 & 0
			\end{bmatrix}, \quad
			C = \begin{bmatrix}
				1 \\
				0 \\
				-1
			\end{bmatrix}$
		
		\item 
		$A X + B = C$, \quad gdzie
		\[
		A = \begin{bmatrix}
			1 & 2 \\
			3 & 4
		\end{bmatrix}, \quad
		B = \begin{bmatrix}
			1 \\
			-1
		\end{bmatrix}, \quad
		C = \begin{bmatrix}
			6 \\
			9
		\end{bmatrix}
		\]
		
		\item 
		$X A - B = C$, \quad gdzie
		\[
		A = \begin{bmatrix}
			2 & 0 \\
			1 & 3
		\end{bmatrix}, \quad
		B = \begin{bmatrix}
			1 & 1 \\
			0 & 2
		\end{bmatrix}, \quad
		C = \begin{bmatrix}
			3 & 1 \\
			2 & 4
		\end{bmatrix}
		\]
		
		\item 
		$A X - B X = C$, \quad gdzie
		\[
		A = \begin{bmatrix}
			4 & 1 \\
			0 & 2
		\end{bmatrix}, \quad
		B = \begin{bmatrix}
			1 & 1 \\
			0 & 1
		\end{bmatrix}, \quad
		C = \begin{bmatrix}
			3 & 0 \\
			0 & 1
		\end{bmatrix}
		\]
		
		\item \[
		AX + C = BX - D
		\]
		gdzie:
		\[
		A = \begin{bmatrix}
			2 & 1 \\
			0 & 3
		\end{bmatrix}, \quad
		B = \begin{bmatrix}
			1 & 4 \\
			2 & 1
		\end{bmatrix}, \quad
		C = \begin{bmatrix}
			5 & 2 \\
			1 & -1
		\end{bmatrix}, \quad
		D = \begin{bmatrix}
			3 & 0 \\
			2 & 4
		\end{bmatrix}
		\]
	\end{enumerate}}
	
\end{document}