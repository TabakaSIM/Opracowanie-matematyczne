\documentclass[12pt,a4paper]{article}
\usepackage[utf8]{inputenc} %polskie znaki
\usepackage[T1]{fontenc}	%polskie znaki
\usepackage{amsmath}		%matematyczne znaczki :3
\usepackage{enumerate}		%Dodatkowe opcje do funkcji enumerate
%\pagestyle{empty} 			%usuwa nr strony

\begin{document}

	\begin{enumerate}
		\item 	Oblicz pole obszaru ograniczonego krzywą o równaniu $y=f(x)$, prostymi $x=a$, $x=b$ oraz osią $OX$, jeżeli:
	\begin{enumerate}
		\item $f(x)=\frac{1}{\sqrt{x}}$, $a=1$, $b=4$
		\item $f(x)=-x^2+7x-10$, $a=0$, $b=5$
		\item $f(x)=e^{-x}$, $a=-1$, $b=0$
		\item *$f(x)=x\ln(x)$, $a=0$, $b=4$
	\end{enumerate}
		\item Wyznacz pole obszaru ograniczone przez proste $x=0$, $y=0$ oraz $y+2x=8$.
		\item Wyznacz wartość $\iint \limits_D f(x,y) dydx $, gdy:
	\begin{enumerate}
		\item $f(x,y)=x+y$ oraz $D$ jest obszarem ograniczonym przez parabole $4-x^2$ oraz oś $OX$.
		\item $f(x,y)=xy$ oraz $D$ jest obszarem ograniczonym przez parabole $y^2=x+4$ oraz prostą $x=5$.
	\end{enumerate}
	\end{enumerate}


	Oblicz pochodne cząstkowe pierwszego i drugiego rzędu
	\begin{enumerate}[a)]
		\item $f(x,y)=x^4-x^2y^2$
		\item $f(x,y)=x \ln (x+y)$
		\item $f(x,y)=\frac{y}{x}$
		\item $f(x,y)=xy+\sqrt{x^2+y^2}$
		\item $f(x,y)=\frac{x+y}{x-y}$
		\item $f(x,y)=(x+y)e^{x^2-y^2}$		
	\end{enumerate}
\end{document}