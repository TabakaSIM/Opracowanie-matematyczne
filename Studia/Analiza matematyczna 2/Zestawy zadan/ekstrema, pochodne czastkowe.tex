\documentclass[12pt,a4paper]{article}
\usepackage[utf8]{inputenc} %polskie znaki
\usepackage[T1]{fontenc}	%polskie znaki
\pagestyle{empty} 			%usuwa nr strony

\begin{document} 
	Wyznacz pochodne cząstkowe drugiego rzędu poniższych funkcji:
	\begin{enumerate}
		\item $f(x,y)=x^3-y^5$
		\item $f(x,y)=x^4-x^2y^2$
		\item $f(x,y)=\frac{x}{y^2}$
		\item $f(x,y)=x\sin(x+y)$
		\item $f(x,y)=e^{2x-y^2}$
		\item $f(x,y)=e^{x^3} \cdot e^{y^3}$
		\item $f(x,y)=\cos(e^x+e^y)$
		\item $f(x,y)=\frac{1}{x^3+3y}$
	\end{enumerate}
	Wyznacz ekstrema lokalne poniższych funkcji:
	\begin{enumerate}
		\item $f(x,y)=1+6x-x^2-xy-y^2$
		\item $f(x,y)=x^2+(y-1)^2$
		\item $f(x,y)=x^3+y^3+3xy$
		\item $f(x,y)=x^2-xy+y^2-2x+y$
	\end{enumerate}
\end{document}