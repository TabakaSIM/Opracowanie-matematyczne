\documentclass[12pt,a4paper]{article}
\usepackage[utf8]{inputenc} %polskie znaki
\usepackage[T1]{fontenc}	%polskie znaki
\pagestyle{empty} 			%usuwa nr strony

\begin{document}
	
	\begin{center}
		\LARGE
		\textbf{PRÓBNY EGZAMIN Z ANALIZY MATEMATYCZNEJ 2}
		\textbf{Zestaw 3}
	\end{center}
	\textbf{Poniższe zadania są punktowane w skali 0-5 każde. Powodzenia!}
	\begin{enumerate}
		\item Wyznacz pochodne cząstkowe drugiego rzędu funkcji:
		$$xy\cdot e^{x^2+y^2}.$$
		
		\item Zbadaj ekstrema funkcji:
		$$f(x,y)=x^4-2x^2+y^4+2y^2.$$
		
		\item Wyznacz całkę podwójną z funkcji $f(x,y)=x+y$, po trójkącie o wierzchołkach $A=(0,0)$,  $B=(2,-4)$, $C=(6,0).$ 
	
		
		\item Oblicz pole obszaru ograniczone przez funkcje $y=x^2+4x+1$ \\ i $y=-x-3$.
	\end{enumerate}
\end{document}