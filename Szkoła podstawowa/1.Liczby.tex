\documentclass[12pt,a4paper]{article}

\usepackage[utf8]{inputenc} %polskie znaki
\usepackage[T1]{fontenc}	%polskie znaki
\usepackage{amsmath}		%matematyczne znaczki :3
\usepackage{enumerate}		%Dodatkowe opcje do funkcji enumerate
\usepackage{geometry} 		%Ustawianie marginesow
\usepackage{graphicx}		%Grafika
\usepackage{wrapfig}		%Grafika obok textu
\usepackage{float}			%Allows H in figure
\usepackage{hyperref}		%Allows hyperlinks
%\pagestyle{empty} 			%usuwa nr strony
\usepackage{todonotes}		%Todo notatki
\usepackage{lipsum}         %Lorem text
\usepackage{ntheorem}   	% for theorem-like environments
\usepackage{mdframed}   	% for framing
\usepackage{subcaption}		% subfigure (image placing)
\usepackage{pdfcomment}		% Komentarze (z bazowego pdf'a)
\usepackage{xparse}			% New commands with optional arguments
\usepackage{ifthen}			% If then - funkcje!
\usepackage{expl3}			% Deklarowanie zmiennych
\usepackage{pgf}			% Aktualne rachunki \pgfmathparse{}
\usepackage{amsmath} 		% For mathematical symbols and structures
\usepackage{amsfonts}		% Zbiór liczb naturalnych + formatowanie
\usepackage{ulem}			% Przekreślony text
%\usepackage[colorlinks=true, linkcolor=blue, urlcolor=red, citecolor=green]{hyperref}
\usepackage{fontawesome5}
\usepackage{mathtools}
%\usepackage{multirow} 		% Required for merging rows
\newcommand{\niton}{\not\owns}

\newgeometry{tmargin=2cm, bmargin=2cm, lmargin=2cm, rmargin=2cm} 

%Counter commands{
	\newcounter{definicja}
	\setcounter{definicja}{1} 
	
	\newcounter{twierdzenie}
	\setcounter{twierdzenie}{1} 
	
	\newcounter{przyklady}
	\setcounter{przyklady}{1} 
	
	\newcounter{wnioski}
	\setcounter{wnioski}{1} 
	
	\newcommand{\counter}[1]{
		\arabic{#1} \stepcounter{#1} 
	}
	\newcommand{\counterreset}[1]{\setcounter{#1}{1}}
	%}

%Define styles{
	\theoremstyle{break}
	\theoreminframepreskip{0.5cm}
	\theoremheaderfont{\bfseries}
	\newmdtheoremenv[%
	linecolor=white,%
	innertopmargin=\topskip,
	shadowsize=0,%
	innertopmargin=5,%
	innerbottommargin=5,%
	leftmargin=10,%
	rightmargin=10,%
	backgroundcolor=white!20,%
	innertopmargin=0pt,%
	ntheorem]{zad}{Zadanie}
	
	\mdfdefinestyle{zadanie}{
		linecolor=white,%
		innertopmargin=5,%
		innerbottommargin=5,%
		leftmargin=0,%
		rightmargin=0,%
		backgroundcolor=lightgray!20,%
		innertopmargin=8,
		innerbottommargin=8,
		skipabove = 5,
	}
	\mdfdefinestyle{wzor}{
		linecolor=cyan,%
		linewidth=2pt,%
		innertopmargin=0,
		innerbottommargin=8,
		leftmargin=10,%
		rightmargin=10,%
		backgroundcolor = white, 
		fontcolor = black,
		skipabove = 5,
		skipbelow = 5,
	}
	%}

%Zadania templatex%{
	\newcommand{\Obramowka}[1]{
		\begin{mdframed}[style=wzor]
			\centering #1
		\end{mdframed}
	}
	\newcommand{\Komentarz}[1]{
		\begin{mdframed}[style=zadanie]
			\textbf{Komentarz}\\
			#1
		\end{mdframed}
	}
	
	\newcommand{\Odp}[1]{
		\begin{mdframed}[style=zadanie]
			\textbf{Odpowiedź}\\
			#1
		\end{mdframed}
	}
	
	%}

% Set spacing before and after theorems
\setlength{\theorempreskipamount}{20pt}  % Space above the theorem
\setlength{\theorempostskipamount}{20pt} % Space below the theorem

\newtheorem{definition}{Definicja}[section]

\newtheorem{theorem}{Twierdzenie}[section]
\newtheorem{lemma}{Lemat}[section]
\newtheorem{wniosek}{Wniosek}[theorem]
\newtheorem{example}{Przykład}[section]
\newtheorem{exercise}{Ćwiczenie}[section]
\newtheorem{stwierdzenie}{Stwierdzenie}[section]
\newtheorem{obserwacja}{Obserwacja}[section]

%%%%%%%%%%%%%%%%%%%%%%%%%%%%%%%%%%%%%%%%%%%%%
% Misc commands{
	\newcommand{\tg}{\text{tg}}
	\newcommand{\ctg}{\text{ctg}}
	\newcommand{\arctg}{\text{arctg}}
	\newcommand{\arcctg}{\text{arcctg}}
	
	\newcommand{\UkladRownan}[2]{
		\left\{
		\begin{array}{l}
			#1 \\
			#2
		\end{array}
		\right.
	}
% }Misc commands
%%%%%%%%%%%%%%%%%%%%%%%%%%%%%%%%%%%%%%%%%%%%%

%\pagestyle{empty} 			%usuwa nr strony

\begin{document}
	
	\ZbiorZadanieTwoXThree{Zapisać w postaci liczby arabskiej}{MMCCCLXIV}{MCDVI}{DCXLVIII}{MCCCLXXXVIII}{MMDCCLXIV}{CCXXII}
	
	\Kratka{8}
	
	\ZbiorZadanieTwoXThree{Zapisać w postaci liczby rzymskiej}{1410}{1687}{469}{1999}{3769}{499}
	
	\Kratka{10}
	
	\ZbiorZadanie{Na lekcji matematyki uczniowie mieli zapisać liczbę 1699. Marysia zapisała liczbę MDCIC, a Kamil liczbę MDCXCIX. Czy obaj zapisali tą liczbę poprawnie? Jeśli nie, to gdzie został popełniony błąd?}
	
	\Kratka{10}
	
	\newpage
	\ZbiorZadanieABCD{Poniżej podano 4 zbiory, który z nich zawiera tylko liczby pierwsze?}{7,49,149,811}{15,31,48,125}{13,17,53,89}{11,29,47,91}
	
	\Kratka{10}
	
	\ZbiorZadanie{Michał napisał na tablicy cyfry 2, 3, 9, 4 oraz powiedział, że da się ułożyć je wszystkie w takiej kolejności, aby móc otrzymać liczbę pierwszą. Czy Michał ma rację?}
	
	\Kratka{14}
	
	\ZbiorZadanie{Tomek spytał dziadka o szyfr do sejfu. Dziadek odpowiedział, że go nie pamięta, ale wie że szyfr do sejfu składa się z największej i najmniejszej dwucyfrowej liczby pierwszej. Jaki jest kod do sejfu dziadka Tomka?}
	
	\Kratka{14}
	
	\newpage
	\ZbiorZadanieTwoXThree{Obliczyć NWD i NWW liczb:}{96 i 120}{75 i 125}{56 i 175}{66 i 121}{50 i 200}{84 i 308}
	
	\Kratka{20}
	
	\ZbiorZadanieABCD{Na tablicy zapisano liczby 135 oraz 81. Natępnie Radek obliczył liczbę x, która była najmniejszą wspólną wielokrotnością tych liczb, a Karol obliczył liczbę y, która była największym wspólnym dzielikiem tych licz. Różnica liczb x i y jest równa}{1026}{378}{108}{404}
	
	\Kratka{10}
	
	\newpage
	\ZbiorZadaniePF{Ania wraz z babcią planuje upiec babeczni na imprezę. Na jednej blaszce jest miejsce na 12 babeczek. Na spotkanie zaproszonych jest 16 osób, a Ania chce tak upiec babeczki, by dla każdego gościa było po równo baberzek.}{Ania musi upiec najmniej 3 blachy babeczek.}{Każdy z gości dostanie przynajmniej 3 babeczki.}
	
	\Kratka{10}
	
	\ZbiorZadanie{Paweł i Gaweł robili okrążenia wokół boiska, obaj wystartowali z tego samego miejsca. Paweł zrobił 5 okrążeń w ciągu 12 minut, a Gaweł 3 okrązenia w ciągu 10 minut. Kiedy Paweł i Gaweł spotkają się znowu na lini startu? Ile zrobią razem okrążeń?}
	
	\Kratka{10}
	
	\ZbiorZadanie{Maciek postanowił przeczytać książkę w ciągu pięciu dni. Pierwszego dnia przeczytał najmniej, natomiast drugiego dnia przeczytał najwięcej w porównaniu do pozostałych i dokładnie dwa razy więcej niż pierwszego. Trzeciego dnia przeczytał stron tyle co pewna liczba posiadająca dokłądnie dwa dzielniki. Czwartego dnia liczbę która jest pierwsza. A piątego średnią arytmetyczną liczby dwóch poprzednich dni. Gdyby obliczył iloczyn tych  wszystkich liczb otrzymał 25.200. Ile stron miała ta książka?}
	
	\Kratka{10}
	
\end{document}