\documentclass[12pt,a4paper]{article}

\usepackage[utf8]{inputenc} %polskie znaki
\usepackage[T1]{fontenc}	%polskie znaki
\usepackage{amsmath}		%matematyczne znaczki :3
\usepackage{enumerate}		%Dodatkowe opcje do funkcji enumerate
\usepackage{geometry} 		%Ustawianie marginesow
\usepackage{graphicx}		%Grafika
\usepackage{wrapfig}		%Grafika obok textu
\usepackage{float}			%Allows H in figure
\usepackage{hyperref}		%Allows hyperlinks
%\pagestyle{empty} 			%usuwa nr strony
\usepackage{todonotes}		%Todo notatki
\usepackage{lipsum}         %Lorem text
\usepackage{ntheorem}   	% for theorem-like environments
\usepackage{mdframed}   	% for framing
\usepackage{subcaption}		% subfigure (image placing)
\usepackage{pdfcomment}		% Komentarze (z bazowego pdf'a)
\usepackage{xparse}			% New commands with optional arguments
\usepackage{ifthen}			% If then - funkcje!
\usepackage{expl3}			% Deklarowanie zmiennych
\usepackage{pgf}			% Aktualne rachunki \pgfmathparse{}
\usepackage{amsmath} 		% For mathematical symbols and structures
\usepackage{amsfonts}		% Zbiór liczb naturalnych + formatowanie
\usepackage{ulem}			% Przekreślony text
%\usepackage[colorlinks=true, linkcolor=blue, urlcolor=red, citecolor=green]{hyperref}
\usepackage{fontawesome5}
\usepackage{mathtools}
%\usepackage{multirow} 		% Required for merging rows
\newcommand{\niton}{\not\owns}

\newgeometry{tmargin=2cm, bmargin=2cm, lmargin=2cm, rmargin=2cm} 

%Counter commands{
	\newcounter{definicja}
	\setcounter{definicja}{1} 
	
	\newcounter{twierdzenie}
	\setcounter{twierdzenie}{1} 
	
	\newcounter{przyklady}
	\setcounter{przyklady}{1} 
	
	\newcounter{wnioski}
	\setcounter{wnioski}{1} 
	
	\newcommand{\counter}[1]{
		\arabic{#1} \stepcounter{#1} 
	}
	\newcommand{\counterreset}[1]{\setcounter{#1}{1}}
	%}

%Define styles{
	\theoremstyle{break}
	\theoreminframepreskip{0.5cm}
	\theoremheaderfont{\bfseries}
	\newmdtheoremenv[%
	linecolor=white,%
	innertopmargin=\topskip,
	shadowsize=0,%
	innertopmargin=5,%
	innerbottommargin=5,%
	leftmargin=10,%
	rightmargin=10,%
	backgroundcolor=white!20,%
	innertopmargin=0pt,%
	ntheorem]{zad}{Zadanie}
	
	\mdfdefinestyle{zadanie}{
		linecolor=white,%
		innertopmargin=5,%
		innerbottommargin=5,%
		leftmargin=0,%
		rightmargin=0,%
		backgroundcolor=lightgray!20,%
		innertopmargin=8,
		innerbottommargin=8,
		skipabove = 5,
	}
	\mdfdefinestyle{wzor}{
		linecolor=cyan,%
		linewidth=2pt,%
		innertopmargin=0,
		innerbottommargin=8,
		leftmargin=10,%
		rightmargin=10,%
		backgroundcolor = white, 
		fontcolor = black,
		skipabove = 5,
		skipbelow = 5,
	}
	%}

%Zadania templatex%{
	\newcommand{\Obramowka}[1]{
		\begin{mdframed}[style=wzor]
			\centering #1
		\end{mdframed}
	}
	\newcommand{\Komentarz}[1]{
		\begin{mdframed}[style=zadanie]
			\textbf{Komentarz}\\
			#1
		\end{mdframed}
	}
	
	\newcommand{\Odp}[1]{
		\begin{mdframed}[style=zadanie]
			\textbf{Odpowiedź}\\
			#1
		\end{mdframed}
	}
	
	%}

% Set spacing before and after theorems
\setlength{\theorempreskipamount}{20pt}  % Space above the theorem
\setlength{\theorempostskipamount}{20pt} % Space below the theorem

\newtheorem{definition}{Definicja}[section]

\newtheorem{theorem}{Twierdzenie}[section]
\newtheorem{lemma}{Lemat}[section]
\newtheorem{wniosek}{Wniosek}[theorem]
\newtheorem{example}{Przykład}[section]
\newtheorem{exercise}{Ćwiczenie}[section]
\newtheorem{stwierdzenie}{Stwierdzenie}[section]
\newtheorem{obserwacja}{Obserwacja}[section]

%%%%%%%%%%%%%%%%%%%%%%%%%%%%%%%%%%%%%%%%%%%%%
% Misc commands{
	\newcommand{\tg}{\text{tg}}
	\newcommand{\ctg}{\text{ctg}}
	\newcommand{\arctg}{\text{arctg}}
	\newcommand{\arcctg}{\text{arcctg}}
	
	\newcommand{\UkladRownan}[2]{
		\left\{
		\begin{array}{l}
			#1 \\
			#2
		\end{array}
		\right.
	}
% }Misc commands
%%%%%%%%%%%%%%%%%%%%%%%%%%%%%%%%%%%%%%%%%%%%%

%\pagestyle{empty} 			%usuwa nr strony

\begin{document}
	\setcounter{zadaniaRozdzial}{3}
	
	\ZbiorZadanieTwoXThree{Wykonać dodawanie}{$3\frac{1}{5}+5\frac{2}{3}$}{$4\frac{2}{7}+3\frac{5}{6}$}{$2\frac{3}{4}+1\frac{1}{8}$}{$5\frac{1}{9}+6\frac{2}{3}$}{$7\frac{5}{12}+2\frac{3}{4}$}{$1\frac{4}{5}+3\frac{7}{10}$}
	
	\Kratka{8}
	
	\ZbiorZadanieTwoXThree{Wykonać odejmowania}{$7\frac{3}{5}-2\frac{1}{4}$}{$5\frac{6}{7}-3\frac{2}{3}$}{$9\frac{1}{8}-4\frac{5}{16}$}{$6\frac{3}{4}-1\frac{2}{5}$}{$8\frac{2}{9}-3\frac{4}{9}$}{$10\frac{5}{6}-7\frac{1}{3}$}
	
	\Kratka{8}
	
	\ZbiorZadanieTwoXThree{Wykonaj mnożenie i dzielenie}{$\frac{1}{3}\cdot \frac{3}{4}$}{$4\frac{2}{3}\cdot 1\frac{2}{7}$}{$5\frac{1}{6}\cdot 2\frac{4}{7}$}{$5\frac{1}{4} : 3$}{$8 : 1\frac{1}{3}$}{$3\frac{2}{5}\cdot 2\frac{1}{8}$}
	
	\Kratka{8}
	
	\newpage
	
	\ZbiorZadanie{Ania miała \(\frac{5}{6}\) litra soku. Wypiła \(\frac{1}{3}\) litra. Ile soku jej zostało?}
	
	\Kratka{7}
	
	\ZbiorZadanie{Tomek przebiegł w poniedziałek \(\frac{5}{8}\) trasy, a we wtorek jeszcze \(\frac{1}{4}\) tej samej trasy. Ostatniego dnia przebiegł 0,5 km. Jakiej długości była cała trasa?}
		
	\Kratka{7}
	
	\ZbiorZadanie{Pani Kasia upiekła ciasto i podzieliła je na równe części. Goście zjedli \(\frac{2}{5}\) ciasta, a dzieci \(\frac{1}{3}\) ciasta. Jaka część pozostała?}
	
	\Kratka{7}
	
	\ZbiorZadanie{W ogrodzie \(\frac{3}{7}\) kwiatów stanowiły tylipany, \(\frac{1}{3}\) pozostałych kwiatów toróze, a reszta to storczyki których było 24. Obliczyć ile było tulipanów i róż w tym ogrodzie.}
	
	\Kratka{7}
	
	\ZbiorZadanie{Michał wraz z trójką braci chcieli podzielić czekoladę. Michał zaproponował, że najstarszy z braci dostanie połowe czekolady, obaj młodsi bracia $\frac{1}{4}$ czekolady, a sam dostanie $\frac{1}{3}$. Czy jego bracia powinni się na taki podział zgodzić? Odpowiedź uzasadnić.}
	
	\Kratka{6}
	
\end{document}