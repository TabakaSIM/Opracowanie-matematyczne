\documentclass[12pt,a4paper]{article}
\usepackage[utf8]{inputenc} %polskie znaki
\usepackage[T1]{fontenc}	%polskie znaki
\usepackage{amsmath}		%matematyczne znaczki :3
\usepackage{enumerate}		%Dodatkowe opcje do funkcji enumerate
\usepackage{geometry} 		%Ustawianie marginesow
\usepackage{graphicx}		%Grafika
\usepackage{wrapfig}		%Grafika obok textu
\usepackage{float}			%Allows H in fugire
\usepackage{hyperref}		%Allows hyperlinks
%\pagestyle{empty} 			%usuwa nr strony
\usepackage{dsfont}
\usepackage{todonotes}		%Todo notatki
\usepackage{lipsum}         %Lorem text
\usepackage{ntheorem}   	% for theorem-like environments
\usepackage{mdframed}   	% for framing
\usepackage{subcaption}		% subfigure (image placing)
\usepackage{pdfcomment}		% Komentarze (z bazowego pdf'a)
\usepackage{xparse}			% New commands with optional arguments
\usepackage{ifthen}			% If then - funkcje!
\usepackage{expl3}			% Deklarowanie zmiennych

\newgeometry{tmargin=2cm, bmargin=2cm, lmargin=2cm, rmargin=2cm} 

%Counter commands{
	\newcounter{counter} % Creates a new counter
	\setcounter{counter}{1} % Sets the counter to 5
	\newcommand{\counter}[1]{
		\arabic{#1} \stepcounter{#1} 
	}
	\newcommand{\counterreset}[1]{\setcounter{#1}{1}}
	%}

%Define styles{
	\theoremstyle{break}
	\theoreminframepreskip{0.5cm}
	\theoremheaderfont{\bfseries}
	\newmdtheoremenv[%
	linecolor=white,%
	innertopmargin=\topskip,
	shadowsize=0,%
	innertopmargin=5,%
	innerbottommargin=5,%
	leftmargin=10,%
	rightmargin=10,%
	backgroundcolor=gray!20,%
	innertopmargin=0pt,%
	ntheorem]{zad}{Zadanie}
	
	\mdfdefinestyle{zadanie}{
		linecolor=white,%
		innertopmargin=5,%
		innerbottommargin=5,%
		leftmargin=10,%
		rightmargin=10,%
		backgroundcolor=gray!20,%
		innertopmargin=8,
		innerbottommargin=8,
		skipabove = 5,
	}
	\mdfdefinestyle{wzor}{
		linecolor=cyan,%
		linewidth=2pt,%
		innertopmargin=8,
		innerbottommargin=8,
		leftmargin=10,%
		rightmargin=10,%
		backgroundcolor = white, 
		fontcolor = black,
		skipabove = 5,
		skipbelow = 5,
	}
	%}

%Zadania templatex%{
	\newcommand{\Wzor}[1]{
		\begin{mdframed}[style=wzor]
			\centering #1
		\end{mdframed}
	}
	\newcommand{\ZadanieTextowe}[1]{
		\begin{mdframed}[style=zadanie]
			\textbf{Zadanie \counter{counter} } \\
			#1
		\end{mdframed}
	}
	\newcommand{\Zadanie}[2]{
		\ZadanieTextowe{#1}
		#2
	}
	\newcommand{\ZadanieABCD}[6]{
		\ZadanieTextowe{#1}
		#2 \\\\
		\begin{tabular}{p{7cm} p{7cm}}
			\textbf{A. }#3&
			\textbf{B. }#4\\\\
			\textbf{C. }#5&
			\textbf{D. }#6\\
		\end{tabular}
	}
	\newcommand{\ZadanieABCDEF}[8]{
		\ZadanieTextowe{#1}
		#2 \\\\
		\begin{tabular}{p{7cm} p{7cm}}
			\textbf{A. }#3&
			\textbf{B. }#4\\\\
			\textbf{C. }#5&
			\textbf{D. }#6\\\\
			\textbf{E. }#7&
			\textbf{F. }#8\\\\
		\end{tabular}
	}
	\newcommand{\Zadanietwoxtwo}[5]{
		\ZadanieTextowe{#1}
		\begin{tabular}{p{7cm} p{7cm}}
			\textbf{a)} #2&
			\textbf{b)} #3\\\\
			\textbf{c)} #4&
			\textbf{d)} #5\\\\
		\end{tabular}
	}
	\newcommand{\Zadanietwoxthree}[7]{
		\ZadanieTextowe{#1}
		\begin{tabular}{p{7cm} p{7cm}}
			\textbf{a)} #2&
			\textbf{b)} #3\\\\
			\textbf{c)} #4&
			\textbf{d)} #5\\\\
			\textbf{e)} #6&
			\textbf{f)} #7\\\\
		\end{tabular}
	}
	\newcommand{\Zadanietwoxfour}[9]{
		\ZadanieTextowe{#1}
		\begin{tabular}{p{7cm} p{7cm}}
			\textbf{a)} #2&
			\textbf{b)} #3\\\\
			\textbf{c)} #4&
			\textbf{d)} #5\\\\
			\textbf{e)} #6&
			\textbf{f)} #7\\\\
			\textbf{g)} #8&
			\textbf{h)} #9\\\\
		\end{tabular}
	}
	%}

\begin{document}
	
\ZadanieTextowe{Wypisać wszystkie wysokości trójkąta prostokątnego o bokach $6\text{ cm}$ i $8\text{ cm}$.}

\ZadanieTextowe{Odpowiedzieć na pytania Prawda/Fałsz}

\vspace{0.5cm}
\begin{tabular}{|p{12.5cm}|p{1cm}|p{1cm}|}
	\hline
	\begin{flushleft}
		Prawdziwe jest równanie $\sqrt{8}+\sqrt{8}=\sqrt{32}$
	\end{flushleft}&\begin{center}
		\textbf{P}
	\end{center}&\begin{center}
		\textbf{F}
	\end{center}\\
	\hline
	\begin{flushleft}
		Wyrażenie $7-5\sqrt{2}$ jest dodatnie.
	\end{flushleft}&\begin{center}
		\textbf{P}
	\end{center}&\begin{center}
		\textbf{F}
	\end{center}\\
	\hline
\end{tabular}

\ZadanieTextowe{W konkursie matematycznym było 20 zadań, za dobrą odpowiedź dostawało się 5 punktów, a za złą lub jej brak traciło 2 punkty. Romek zdobył w tym konkursie 72 punkty. Obliczyć, na ile zadań odpowiedział poprawnie?}

\ZadanieTextowe{Biznesmen zapłacił za obiad 300~zł wraz z 20\% napiwkiem. Obliczyć, jaka była cena obiadu bez napiwku?}

\ZadanieABCD{W sierpniu cena pewnej hulajnogi elektrycznej wzrosła o 25\%, natomiast we wrześniu cena ta zmalała o 20\%. Dokończyć zdanie:}{Cena tej hulajnogi po drugiej zmianie względem ceny początkowej:}{zwiększyła się}{zmniejszyła się}{jest taka sama}{nie da się ustalić}

\ZadanieTextowe{Prędkość rozchodzenia się impulsu elektrycznego u człowieka wynosi około $2 \text{ m/s}$. U roślin impuls elektryczny może rozchodzić się z prędkością około $60 \text{ cm/min}$. \\\\ Obliczyć, ile razy prędkość rozchodzenia się impulsu elektrycznego u człowieka jest większa od prędkości rozchodzenia się impulsu elektrycznego u roślin.}

\ZadanieTextowe{Uzasadnić, że suma trzech kolejnych liczb naturalnych jest podzielna przez 3.}

	
\end{document}