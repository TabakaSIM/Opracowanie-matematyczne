\documentclass[12pt,a4paper]{article}
\usepackage[utf8]{inputenc} %polskie znaki
\usepackage[T1]{fontenc}	%polskie znaki
\usepackage{amsmath}		%matematyczne znaczki :3
\usepackage{enumerate}		%Dodatkowe opcje do funkcji enumerate
\usepackage{geometry} 		%Ustawianie marginesow
\usepackage{graphicx}		%Grafika
\usepackage{wrapfig}		%Grafika obok textu
\usepackage{float}			%Allows H in fugire
\pagestyle{empty} 			%usuwa nr strony

\newgeometry{tmargin=2cm, bmargin=2cm, lmargin=2cm, rmargin=2cm} 

\begin{document}
	\begin{center}
		\LARGE Ciągi - karta pracy
	\end{center}
	\vspace{1.5cm}
	\begin{tabular}{p{13cm} r}
		Imię i nazwisko: ............................................................................
		&[....../30pkt]\\ 
		\vspace{0.5cm}
	\end{tabular}
	\begin{enumerate}[1.]
		\item  \begin{tabular}{p{13cm} r}
			Suma trzeciego i czwartego wyrazu ciągu arytmetycznego wynosi 20, natomiast różnica siódmego i drugiego 10. Wyznacz wzór ogólny tego ciągu. &[4pkt]\\ 
		\end{tabular}
	
	\item  \begin{tabular}{p{13cm} r}
		Wykaż, że podany ciąg jest ciągiem geometrycznym: &[3pkt]\\ 
	\end{tabular}
	$$(\sqrt{5}-2, \quad \frac{1}{2}, \quad \frac{\sqrt{5}+2}{4})$$
	\item  \begin{tabular}{p{13cm} r}
		Wyznacz wartości "x", dla których podany ciąg jest ciągiem arytmetycznym: &[5pkt]\\ 
	\end{tabular}
		$$(x^2+1,\quad 5x-2, \quad 2x^2+x+1)$$
		
		\item  \begin{tabular}{p{13cm} r}
			Znajdź wzór ogólny ciągu arytmetycznego, gdzie $S_n=518$, $a_1=50$, $n=14$. &[4pkt]\\ 
		\end{tabular}
	
	\item  \begin{tabular}{p{13cm} r}
		Rozwiąż równanie: &[7pkt]\\ 
	\end{tabular}

	$$2+4+6+\dots + 2(x+1) = 200$$
	
	\item  \begin{tabular}{p{13cm} r}
		Trzy liczby $x,y,z$ których suma jest równa 24, tworzą w podanej kolejności ciąg arytmetyczny. Ciąg $(x+1,y-2,z-2)$ jest ciągiem geometrycznym. Wyznacz $x,y,z$. &[7pkt]\\ 
	\end{tabular}

	\end{enumerate}
	
\end{document}