\documentclass[12pt,a4paper]{article}
\usepackage[utf8]{inputenc} %polskie znaki
\usepackage[T1]{fontenc}	%polskie znaki
\usepackage{amsmath}		%matematyczne znaczki :3
\usepackage{enumerate}		%Dodatkowe opcje do funkcji enumerate
\usepackage{geometry} 		%Ustawianie marginesow
\usepackage{graphicx}		%Grafika
\usepackage{wrapfig}		%Grafika obok textu
\usepackage{float}			%Allows H in fugire
\usepackage{hyperref}		%Allows hyperlinks
%\pagestyle{empty} 			%usuwa nr strony
\usepackage{todonotes}		%Todo notatki
\usepackage{lipsum}         %Lorem text
\usepackage{ntheorem}   	% for theorem-like environments
\usepackage{mdframed}   	% for framing
\usepackage{subcaption}		% subfigure (image placing)
\usepackage{pdfcomment}		% Komentarze (z bazowego pdf'a)
\usepackage{xparse}			% New commands with optional arguments
\usepackage{ifthen}			% If then - funkcje!
\usepackage{expl3}			% Deklarowanie zmiennych

\newgeometry{tmargin=2cm, bmargin=2cm, lmargin=2cm, rmargin=2cm} 

%Counter commands{
	\newcounter{counter} % Creates a new counter
	\setcounter{counter}{1} % Sets the counter to 5
	\newcommand{\counter}[1]{
		\arabic{#1} \stepcounter{#1} 
	}
	\newcommand{\counterreset}[1]{\setcounter{#1}{1}}
	%}

%Define styles{
	\theoremstyle{break}
	\theoreminframepreskip{0.5cm}
	\theoremheaderfont{\bfseries}
	\newmdtheoremenv[%
	linecolor=white,%
	innertopmargin=\topskip,
	shadowsize=0,%
	innertopmargin=5,%
	innerbottommargin=5,%
	leftmargin=10,%
	rightmargin=10,%
	backgroundcolor=gray!20,%
	innertopmargin=0pt,%
	ntheorem]{zad}{Zadanie}
	
	\mdfdefinestyle{zadanie}{
		linecolor=white,%
		innertopmargin=5,%
		innerbottommargin=5,%
		leftmargin=10,%
		rightmargin=10,%
		backgroundcolor=gray!20,%
		innertopmargin=8,
		innerbottommargin=8,
		skipabove = 5,
	}
	\mdfdefinestyle{wzor}{
		linecolor=cyan,%
		linewidth=2pt,%
		innertopmargin=8,
		innerbottommargin=8,
		leftmargin=10,%
		rightmargin=10,%
		backgroundcolor = white, 
		fontcolor = black,
		skipabove = 5,
		skipbelow = 5,
	}
	%}

%Zadania templatex%{
	\newcommand{\Wzor}[1]{
		\begin{mdframed}[style=wzor]
			\centering #1
		\end{mdframed}
	}
	\newcommand{\ZadanieTextowe}[1]{
		\begin{mdframed}[style=zadanie]
			\textbf{Zadanie \counter{counter} } \\
			#1
		\end{mdframed}
	}
	\newcommand{\Zadanie}[2]{
		\ZadanieTextowe{#1}
		#2
	}
	\newcommand{\ZadanieABCD}[6]{
		\ZadanieTextowe{#1}
		#2 \\\\
		\begin{tabular}{p{7cm} p{7cm}}
			\textbf{A. }#3&
			\textbf{B. }#4\\\\
			\textbf{C. }#5&
			\textbf{D. }#6\\
		\end{tabular}
	}
	\newcommand{\ZadanieABCDEF}[8]{
		\ZadanieTextowe{#1}
		#2 \\\\
		\begin{tabular}{p{7cm} p{7cm}}
			\textbf{A. }#3&
			\textbf{B. }#4\\\\
			\textbf{C. }#5&
			\textbf{D. }#6\\\\
			\textbf{E. }#7&
			\textbf{F. }#8\\\\
		\end{tabular}
	}
	\newcommand{\Zadanietwoxtwo}[5]{
		\ZadanieTextowe{#1}
		\begin{tabular}{p{7cm} p{7cm}}
			\textbf{a)} #2&
			\textbf{b)} #3\\\\
			\textbf{c)} #4&
			\textbf{d)} #5\\\\
		\end{tabular}
	}
	\newcommand{\Zadanietwoxthree}[7]{
		\ZadanieTextowe{#1}
		\begin{tabular}{p{7cm} p{7cm}}
			\textbf{a)} #2&
			\textbf{b)} #3\\\\
			\textbf{c)} #4&
			\textbf{d)} #5\\\\
			\textbf{e)} #6&
			\textbf{f)} #7\\\\
		\end{tabular}
	}
	\newcommand{\Zadanietwoxfour}[9]{
		\ZadanieTextowe{#1}
		\begin{tabular}{p{7cm} p{7cm}}
			\textbf{a)} #2&
			\textbf{b)} #3\\\\
			\textbf{c)} #4&
			\textbf{d)} #5\\\\
			\textbf{e)} #6&
			\textbf{f)} #7\\\\
			\textbf{g)} #8&
			\textbf{h)} #9\\\\
		\end{tabular}
	}
	%}

\newcommand{\tg}{\text{tg}}
\newcommand{\ctg}{\text{ctg}}
\newcommand{\UkladRownan}[2]{
	$\left\{
	\begin{array}{l}
		#1 \\
		#2
	\end{array}
	\right.$
}

\begin{document}
	\begin{center}
		\LARGE Ciągi - ogólnie
	\end{center}
	\vspace{1cm}
	
	
	
		\Zadanietwoxthree{Wyznacz 3 kolejne wyrazy podanego ciągu:}
		{ $1,-2,3,-4,5,-6,\dots $}{$5,7,9,11,13,15,\dots $}{$0,0,0,0,0,0,\dots $}{$0,1,1,2,3,5,8,\dots $}{$0,\frac{1}{2},\frac{2}{3},\frac{3}{4},\frac{4}{5},\dots $}{$1,2,4,8,16,32,\dots $}

		
		\Zadanietwoxthree{Zapisz wzór na ogólny wyraz ciągu:}{ $1,2,3,4,5,6,\dots $}{$\frac{1}{2},\frac{1}{4},\frac{1}{6},\frac{1}{8},\frac{1}{10},\dots $}{$3,6,9,12,15,18,\dots $}{$5,3,1,-1,-3,-5,\dots $}{$1,-1,1,-1,1,-1,\dots $}{$1,4,9,16,25,36,\dots $} 

		
		\Zadanietwoxthree{Wyznacz pierwsze 5 wyrazów ciągu o wzorze:}{$a_n=n+5 $}{$a_n=-n+7 $}{$a_n=\frac{1}{2}n+3 $}{$a_n=\sqrt{n-1} $}{$a_n=3^n $}{$a_n=\frac{n+3}{2n+1} $} 
		
		\Zadanietwoxthree{Zbadaj monotoniczność ciągu:}{$a_n=n-3 $}{$a_n=2n+1 $}{$a_n=-3n+3 $}{$a_n=n^2+3 $}{$a_n=n^2-5n $}{$a_n=5 $} 
	
	\counterreset{counter}
	\newpage
	\begin{center}
		\LARGE Ciąg arytmetyczny
	\end{center}
	\vspace{1cm}
	
	
	
		\Zadanietwoxthree{Które z podanych ciągów są ciągami arytmetycznymi?}
		{$a_n=3n+1$}{$a_n=7$}
		{$a_n=5n^2-3$}{$a_n=\sqrt{2}n$}
		{$a_n=\frac{2n+1}{3}$}{$a_n=2n$} 
		
		\Zadanietwoxthree{Wypisz wyrazy $a_3$, $a_7$ i $a_{13}$ ciągu arytmetycznego, gdzie: }
		{$a_1=-1 \quad r=3$}{$a_1=10 \quad r=-3$}
		{$a_1=2 \quad a_2=5$}{ $a_1=0 \quad a_2=4$}
		{$a_1=-1 \quad a_5=7$}{$a_1=5 \quad a_5=3$}
	
		\Zadanietwoxtwo{ Wyznacz $a_1$ ciągu arytmetycznego wiedząc, że:}
		{$a_{22}=-92 \quad r=-3$}{$a_7=37 \quad r=6$}
		{$a_{39}15 \quad a_{35}=11$}{$a_{30}=4 \quad a_{20}=3$}
	
		\ZadanieTextowe{Między liczby 65 i 35 wstaw dziewięć liczb tak, aby liczby te utworzyły ciąg arytmetyczny.}
		
		\ZadanieTextowe{Suma czwartego i siódmego wyrazu ciągu arytmetycznego jest równa 86, natomiast suma drugiego i trzynastego jest równa 22. Wyznacz wzór ogólny tego ciągu.}
		
		\ZadanieTextowe{Suma dwóch pierwszych wyrazów ciągu arytmetycznego jest równa 27, natomiast suma trzeciego, piątego i siódmego jest równa 9. Wyznacz wzór ogólny tego ciągu.}
		\newpage
		\ZadanieTextowe{Widząc, że suma drugiego i dziesiątego wyrazu ciągu arytmetycznego jest równa 10, oblicz szósty wyraz tego ciągu.}
		
		 Dla jakich wartości $x$ podany ciąg jest arytmetyczny?
		\begin{enumerate}[a)]
			\item $(3;\qquad x ;\qquad 17) $ 
			\item $(3x+1; \quad 10;\quad 16) $
			\item $(3x+1; \quad 2x-4; \quad 5x+3) $
			\item $(x^2+1; \quad 5x-2; \quad2x^2+x+1) $
		\end{enumerate}
		\Zadanie{Oblicz sumę:}{	
			\begin{enumerate}[a)]
				\item trzydziestu kolejnych liczb będących wielokrotnością liczby 9, z których najmniejszą jest 9
				\item pięćdziesięciu kolejnych liczb będących wielokrotnością liczby 12, z których najmniejszą jest 24.
				\item wszystkich liczb całkowitych od 0 do 150 włącznie
				\item liczb dwucyfrowych podzielnych przez 7
				\item $3+7+11+15+\dots+103=$
				\item $29+22+15+8+\dots+(-272)=$
			\end{enumerate}
		}

	\Zadanietwoxtwo{Wyznacz liczbę wyrazów ciągu arytmetycznego, mając dane:}
	{$S_n=407 \quad a_1=62 \quad a_n=12$}{$S_n=420 \quad a_1=62 \quad r=3$}{$S_n=1016,5 \quad a_1=22 \quad a_n=85$}{$S_n=578 \quad a_1=58 \quad a_n=-3$}
	
	\counterreset{counter}
	\newpage
	\begin{center}
	\LARGE Ciąg geometryczny
	\end{center}
	\vspace{1cm}
	
	
	\Zadanietwoxthree{Wyznacz wzór ogólny ciągu geometrycznego:}
	{$6,12,24,\dots$}{$6,12,24,\dots$}
	{$8,-4,2,\dots$}{$\frac{1}{2},\frac{1}{4},\frac{1}{8},\dots$}
	{$\frac{2}{5},\frac{1}{2},\frac{5}{8},\dots$}{$2,3,4\frac{1}{2},\dots$} 

	\Zadanietwoxtwo{Wypisz pierwsze 5 wyrazów ciągu geometrycznego, gdzie: }{$a_1=1 \quad q=2$}{$a_1=\frac{1}{3} \quad q=3$}{$a_1=2 \quad q=-5$}{$a_1=16 \quad q=\frac{1}{2}$}
	
	\Zadanietwoxtwo{Wyznacz iloraz ciągu geometrycznego q, jeśli:}{$a_1=27 \quad a_2=9$}{$a_1=-1 \quad a_{10}=-512$}{$a_2=1 \quad a_4=625$}{ $a_1=16 \quad a_5=\frac{1}{2}$}

	\ZadanieTextowe{Suma trzech wyrazów ciągu geometrycznego jest równa 21, a ich iloczyn jest równy 216. Wyznacz wyrazy tego ciągu.}


\end{document}