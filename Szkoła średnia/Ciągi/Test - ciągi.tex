\documentclass[12pt,a4paper]{article}
\usepackage[utf8]{inputenc} %polskie znaki
\usepackage[T1]{fontenc}	%polskie znaki
\usepackage{amsmath}		%matematyczne znaczki :3
\usepackage{enumerate}		%Dodatkowe opcje do funkcji enumerate
\usepackage{geometry} 		%Ustawianie marginesow
\usepackage{graphicx}		%Grafika
\usepackage{wrapfig}		%Grafika obok textu
\usepackage{float}			%Allows H in fugire
\pagestyle{empty} 			%usuwa nr strony

\newgeometry{tmargin=2cm, bmargin=2cm, lmargin=2cm, rmargin=2cm} 

\begin{document}
	\begin{center}
		\LARGE Ciągi - sprawdzian
	\end{center}
	\vspace{1.5cm}
	\begin{flushright}
		\textbf{GRUPA A}
	\end{flushright}
	\begin{tabular}{p{13cm} r}
		Imię i nazwisko: ............................................................................
		&[....../30pkt]\\ 
		\vspace{0.5cm}
	\end{tabular}
	\begin{enumerate}[1.]
		\item  \begin{tabular}{p{13cm} r}
			Zbadaj monotoniczność ciągu  &[3pkt]\\ 
		\end{tabular}
	$$a_n=2-\frac{2}{3}n$$
		
		\item  \begin{tabular}{p{13cm} r}
			Dany jest ciąg arytmetyczny $a_n$, którego suma siódmego i trzynastego wyrazu wynosi 14, a suma dziesiątego i dwudziestego drugiego wynosi 10. Wyznacz wzór ogólny tego ciągu. 
			 &[5pkt]\\ 
		\end{tabular}
		
		\item \begin{tabular}{p{13cm} r}
			Wyznacz parametr "k", dla którego trójwyrazowy ciąg
			 &[5pkt]\\
		\end{tabular}
	$$(k+5,\quad k^2+4, \quad k^2+3k)$$
	jest ciągiem arytmetycznym. 
		
		\item \begin{tabular}{p{13cm} r}
			Oblicz ile jest liczb należących do przedziału $(10,400\rangle$, których reszta z dzielenia przez 3 wynosi 1.&[3pkt]\\ 
		\end{tabular}
		
		\item \begin{tabular}{p{13cm} r}
			Suma szesnastu kolejnych wyrazów ciągu arytmetycznego wynosi 640. Wiedząc, że $a_{16}=25$, wyznacz wzór ogólny tego ciągu.&[4pkt]\\ 
		\end{tabular}
		
		\item \begin{tabular}{p{13cm} r}
			 Wiedząc, że ciąg $a_n$ jest ciągiem geometrycznym. oraz że $a_4=6$ i $a_6=18$ oblicz: $q,a_1,a_{10}$. &[3pkt]\\ 
		\end{tabular}
	
		\item \begin{tabular}{p{13cm} r}
			Dany jest trójwyrazowy ciąg arytmetyczny, którego średnia arytmetyczna wynosi 12. Jeżeli pierwszy wyraz tego ciągu zwiększymy o 2, to otrzymamy ciąg geometryczny. Wyznacz ten ciąg. &[7pkt]\\ 
		\end{tabular}

	\end{enumerate}

\newpage

	\begin{center}
	\LARGE Ciągi - sprawdzian
\end{center}
\vspace{1.5cm}
\begin{flushright}
	\textbf{GRUPA B}
\end{flushright}
\begin{tabular}{p{13cm} r}
	Imię i nazwisko: ............................................................................
	&[....../30pkt]\\ 
	\vspace{0.5cm}
\end{tabular}
\begin{enumerate}[1.]
	\item  \begin{tabular}{p{13cm} r}
		Zbadaj monotoniczność ciągu  &[3pkt]\\ 
	\end{tabular}
	$$a_n= n^2 - 5n$$
	
	\item  \begin{tabular}{p{13cm} r}
		Dany jest ciąg arytmetyczny $a_n$, którego suma siódmego i jedenastego wyrazu wynosi -8, a różnica dziewiątego i trzynastego wynosi -1. Wyznacz wzór ogólny tego ciągu. 
		&[5pkt]\\ 
	\end{tabular}
	
	\item \begin{tabular}{p{13cm} r}
		Wyznacz parametr "k", dla którego trójwyrazowy ciąg
		&[5pkt]\\
	\end{tabular}
	$$(k-5,\quad k+3, \quad -5k-3)$$
	jest ciągiem geometrycznym. 
	
	\item \begin{tabular}{p{13cm} r}
		Oblicz ile jest liczb naturalnych, mniejszych od 600, które są \textbf{\underline{nie}} podzielne przez 7.&[3pkt]\\ 
	\end{tabular}
	
	\item \begin{tabular}{p{13cm} r}
		Oblicz sumę wszystkich liczb 3-cyfrowych podzielnych przez 4.&[4pkt]\\ 
	\end{tabular}
	
	\item \begin{tabular}{p{13cm} r}
		Wiedząc, że ciąg $a_n$ jest ciągiem arytmetycznym, oraz że $a_3=\sqrt{8}$ i $a_5=\sqrt{32}$ oblicz: $r,a_1,a_9$. &[3pkt]\\ 
	\end{tabular}
	
	\item \begin{tabular}{p{13cm} r}
		Dany jest trójwyrazowy ciąg geometryczny o wyrazach dodatnich, którego iloczyn pierwszego i trzeciego wyrazu wynosi 36. Jeżeli pierwszy wyraz tego ciągu zmniejszymy o 1, to otrzymamy ciąg arytmetyczny. Wyznacz ten ciąg. &[7pkt]\\ 
	\end{tabular}
	
\end{enumerate}
	
\end{document}