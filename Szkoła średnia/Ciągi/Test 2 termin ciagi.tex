\documentclass[12pt,a4paper]{article}
\usepackage[utf8]{inputenc} %polskie znaki
\usepackage[T1]{fontenc}	%polskie znaki
\usepackage{amsmath}		%matematyczne znaczki :3
\usepackage{enumerate}		%Dodatkowe opcje do funkcji enumerate
\usepackage{geometry} 		%Ustawianie marginesow
\usepackage{graphicx}		%Grafika
\usepackage{wrapfig}		%Grafika obok textu
\usepackage{float}			%Allows H in fugire
\pagestyle{empty} 			%usuwa nr strony

\newgeometry{tmargin=2cm, bmargin=2cm, lmargin=2cm, rmargin=2cm} 

\begin{document}
	\begin{center}
		\LARGE Ciągi - sprawdzian
	\end{center}
	\vspace{1.5cm}
	\begin{flushright}
		\textbf{2 TERMIN}
	\end{flushright}
	\begin{tabular}{p{13cm} r}
		Imię i nazwisko: ............................................................................
		&[....../30pkt]\\ 
		\vspace{0.5cm}
	\end{tabular}
	\begin{enumerate}[1.]
		\item  \begin{tabular}{p{13cm} r}
			Zbadaj monotoniczność ciągu  &[3pkt]\\ 
		\end{tabular}
		$$a_n=8n-n^2$$
		
		\item  \begin{tabular}{p{13cm} r}
			Dany jest ciąg arytmetyczny $a_n$, którego suma trzeciego i piątego wyrazu wynosi 14, a różnica czwartego i siódmego wynosi -12. Wyznacz wzór ogólny tego ciągu. 
			&[5pkt]\\ 
		\end{tabular}
		
		\item \begin{tabular}{p{13cm} r}
			Wyznacz paramert "k", dla którego trójwyrazowy ciąg
			&[5pkt]\\
		\end{tabular}
		$$(k^2-5,\quad k+1, \quad k^2+4k-5)$$
		jest ciągiem arytmetycznym. 
		
		\item \begin{tabular}{p{13cm} r}
			Oblicz sumę:&[3pkt]\\
		\end{tabular}
			$$S=2+5+8+11+\dots + 158$$ 
		
		\item \begin{tabular}{p{13cm} r}
			Sprawdź czy podany ciąg jest ciagiem geometrycznym.&[4pkt]\\ 
		\end{tabular}
			$$(2\sqrt{5}-\sqrt{7},\quad \sqrt{13}, \quad 2\sqrt{5}+\sqrt{7})$$
		
		\item \begin{tabular}{p{13cm} r}
			Wiedząc, że ciąg $a_n$ jest ciągiem geometrycznym. oraz że $a_4=25$ i $a_2=5$ oblicz: $q,a_1,a_6$. &[3pkt]\\ 
		\end{tabular}
		
		\item \begin{tabular}{p{13cm} r}
			Dany jest trójwyrazowy ciąg arytmetyczny, którego średnia arytmetyczna wynosi 12. Jeżeli pierwszy wyraz tego ciągu zwiększymy o 2, to otrzymamy ciąg geometryczny. Wyznacz ten ciąg. &[7pkt]\\ 
		\end{tabular}
		
	\end{enumerate}
	

	
\end{document}