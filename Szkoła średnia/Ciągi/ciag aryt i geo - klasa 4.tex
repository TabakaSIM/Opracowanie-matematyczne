\documentclass[12pt,a4paper]{article}
\usepackage[utf8]{inputenc} %polskie znaki
\usepackage[T1]{fontenc}	%polskie znaki
\usepackage{amsmath}		%matematyczne znaczki :3
\usepackage{enumerate}		%Dodatkowe opcje do funkcji enumerate
\usepackage{geometry} 		%Ustawianie marginesow
\usepackage{graphicx}		%Grafika
\usepackage{wrapfig}		%Grafika obok textu
\usepackage{float}			%Allows H in fugire
%\pagestyle{empty} 			%usuwa nr strony

\newgeometry{tmargin=2cm, bmargin=2cm, lmargin=2cm, rmargin=2cm} 

\begin{document}
	\begin{center}
		\LARGE Ciąg arytmetyczny
	\end{center}
	\vspace{1cm}
	
	
	\begin{enumerate}[1.]	
		\item Które z podanych ciągów są ciągami arytmetycznymi?
		
		
		\begin{enumerate}[a)] \begin{tabular}{p{7cm} p{7cm}} 
				\item $a_n=3n+1$& \vspace{0.4cm} 	\item $a_n=7$ \\
				\item $a_n=5n^2-3$&  	\item $a_n=\sqrt{2}n$ \\
				\item $a_n=\frac{2n+1}{3}$&  	\item $a_n=2n$ \\
		\end{tabular} \end{enumerate}
	
		\item Wypisz wyrazy $a_3$, $a_7$ i $a_{13}$ ciągu arytmetycznego, gdzie: 
		\begin{enumerate}[a)] \begin{tabular}{p{7cm} p{7cm}} 
				\item $a_1=-1 \quad r=3$& \vspace{0.4cm} 	\item $a_1=10 \quad r=-3$ \\
				\item $a_1=2 \quad a_2=5$&  	\item $a_1=0 \quad a_2=4$ \\
				\item $a_1=-1 \quad a_5=7$&  	\item $a_1=5 \quad a_5=3$ \\
		\end{tabular} \end{enumerate}
	
		\item Wyznacz $a_1$ ciągu arytmetycznego wiedząć, że:
		\begin{enumerate}[a)] \begin{tabular}{p{7cm} p{7cm}} 
				\item $a_{22}=-92 \quad r=-3$& \vspace{0.4cm} 	\item $a_7=37 \quad r=6$ \\
				\item $a_{39}15 \quad a_{35}=11$&  	\item $a_{30}=4 \quad a_{20}=3$ \\
		\end{tabular} \end{enumerate}
	
		\item między liczby 65 i 35 wstaw dziewięć liczb tak, aby liczby te utworzyły ciąg arytmetyczny.
		
		\item Suma czwartego i siódmego wyrazu ciągu arytmetycznego jest równa 86, natomiast suma drugiego i trzynastego jest równa 22. Wyznacz wzór ogólny tego ciągu.
		
		\item Suma dwóch pierwszych wyrazów ciągu arytmetycznego jest równa 27, natomiast suma suma trzeciego i suma piątego i siódmego jest równa 0. Wyznacz wzór ogólny tego ciągu.
		
		\item Widząc, że suma drugiego i dziesiątego wyrazu ciągu arytmetycznego jest równa 10, oblicz szósty wyraz tego ciągu.
		
		\item Dla jakich wartości $x$ podany ciąg jest arytmetyczny?
		\begin{enumerate}[a)]
			\item $(3,\qquad x ,\qquad 17) $ 
			\item $(3x+1, \quad 10,\quad 16) $
			\item $(3x+1, \quad 2x-4, \quad 5x+3) $
			\item $(x^2+1, \quad 5x-2, \quad2x^2+x+1) $
		\end{enumerate}
	\newpage
		\item Oblicz sumę:
		\begin{enumerate}[a)]
			\item trzydziestu kolejnych liczb będących wielokrotnością liczby 9, z których najmniejszą jest 9
			\item pięćdziesięciu kolejnych liczb będących wielokrotnością liczby 12, z których najmniejszą jest 24.
			\item wszystkich liczb całkowitych od 0 do 150 włącznie
			\item liczb dwucyfrowych podzielnych przez 7
			\item $3+7+11+15+\dots+103=$
			\item $29+22+15+8+\dots+(-272)=$
		\end{enumerate}
	\item Wyznacz liczbę wyrazów ciągu arytmetycznego, mając dane:
		\begin{enumerate}[a)] \begin{tabular}{p{7cm} p{7cm}} 
		\item $S_n=407 \quad a_1=62 \quad a_n=12$& \vspace{0.4cm} 	\item $S_n=420 \quad a_1=62 \quad r=3$ \\
		\item $S_n=1016,5 \quad a_1=22 \quad a_n=85$&  	\item $S_n=578 \quad a_1=58 \quad a_n=-3$ \\
		\end{tabular} \end{enumerate}
	\end{enumerate}

	\begin{center}
	\LARGE Ciąg geometryczny
	\end{center}
	\vspace{1cm}
	
	\begin{enumerate}[1.]
		
		\item Wyznacz wzór ogólny ciągu geometrycznego:

	\begin{enumerate}[a)] \begin{tabular}{p{7cm} p{7cm}} 
			\item $6,12,24,\dots$& \vspace{0.4cm} 	\item $6,12,24,\dots$ \\
			\item $8,-4,2,\dots$&  	\item $\frac{1}{2},\frac{1}{4},\frac{1}{8},\dots$ \\
			\item $\frac{2}{5},\frac{1}{2},\frac{5}{8},\dots$&  	\item $2,3,4\frac{1}{2},\dots$ \\
	\end{tabular} \end{enumerate}

		\item Wypisz pierwsze 5 wyrazów ciągu geometrycznego, gdzie: 
		\begin{enumerate}[a)] \begin{tabular}{p{7cm} p{7cm}} 
		\item $a_1=1 \quad q=2$& \vspace{0.4cm} 	\item $a_1=\frac{1}{3} \quad q=3$ \\
		\item $a_1=2 \quad q=-5$&  	\item $a_1=16 \quad q=\frac{1}{2}$ \\
		\end{tabular} \end{enumerate}
	
	\item Wyznacz iloraz ciągu geometrycznego q, jeśli:
	\begin{enumerate}[a)] \begin{tabular}{p{7cm} p{7cm}} 
			\item $a_1=27 \quad a_2=9$& \vspace{0.4cm} 	\item $a_1=-1 \quad a_{10}=-512$ \\
			\item $a_2=1 \quad a_4=625$&  	\item $a_1=16 \quad a_5=\frac{1}{2}$ \\
	\end{tabular} \end{enumerate}

	\item Suma trzech wyrazów ciągu geometrycznego jest równa 21, a ich iloczyn jest równy 216. Wyznacz wyrazy tego ciągu.

\end{enumerate}
\end{document}