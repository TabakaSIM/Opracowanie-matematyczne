\documentclass[12pt,a4paper]{article}
\usepackage[utf8]{inputenc} %polskie znaki
\usepackage[T1]{fontenc}	%polskie znaki
\usepackage{amsmath}		%matematyczne znaczki :3
\usepackage{enumerate}		%Dodatkowe opcje do funkcji enumerate
\usepackage{geometry} 		%Ustawianie marginesow
\usepackage{graphicx}		%Grafika
\usepackage{wrapfig}		%Grafika obok textu
\usepackage{float}			%Allows H in fugire
%\pagestyle{empty} 			%usuwa nr strony

\newgeometry{tmargin=2cm, bmargin=2cm, lmargin=2cm, rmargin=2cm} 

\begin{document}
	\begin{center}
		\LARGE Ciągi mieszane - arytmetyczne i geometryczne
	\end{center}
	\vspace{1cm}
	
	
	\begin{enumerate}[1.]	
		\item Wyznacz wszystkie ciągi które są jednocześnie arytmetyczne i geometryczne.
		
		\item Ciąg $(a,b,1)$ jest arytmetyczny, zaś ciąg $(1,a,b)$ jest geometryczny. Oblicz $a$ i $b$.
		
		\item Trzy liczby, których suma wynosi 9, tworzą ciąg arytmetyczny. Jeśli do pierwszej z nich dodamy $3\frac{1}{8}$ to otrzymamy ciąg geometryczny. Wyznacz liczby tego ciągu.
		
		\item Ciąg $(9,x,19)$ jest arytmetyczny, a ciąg $(x,42,y,z)$ jest geometryczny. Oblicz $x$, $y$ i $z$.
		
		\item Dany jest nieskończony ciąg arytmetyczny taki, że $a_5=18$. Wyrazy $a_1$, $a_3$ i $a_13$ tego ciągu są odpowiednio pierwszym, drugim i trzecim wyrazem pewnego ciągu geometrycznego. Wyznacz wzór ogólny tego ciągu arytmetycznego.
		
		\item Trzy liczby, które tworzą rosnący ciąg geometryczny, dają w sumie 35. Jeśli do pierwszej liczby dodamy 4, do drugiej 5, a do trzeciej 1, to otrzymane sumy w tej kolejności dadzą ciąg arytmetyczny. Wyznacz ten ciąg geometryczny.
		
		
	\end{enumerate}

	\begin{center}
	\LARGE Ciągi mieszane - arytmetyczne i geometryczne
\end{center}
\vspace{1cm}


\begin{enumerate}[1.]	
	\item Wyznacz wszystkie ciągi które są jednocześnie arytmetyczne i geometryczne.
	
	\item Ciąg $(a,b,1)$ jest arytmetyczny, zaś ciąg $(1,a,b)$ jest geometryczny. Oblicz $a$ i $b$.
	
	\item Trzy liczby, których suma wynosi 9, tworzą ciąg arytmetyczny. Jeśli do pierwszej z nich dodamy $3\frac{1}{8}$ to otrzymamy ciąg geometryczny. Wyznacz liczby tego ciągu.
	
	\item Ciąg $(9,x,19)$ jest arytmetyczny, a ciąg $(x,42,y,z)$ jest geometryczny. Oblicz $x$, $y$ i $z$.
	
	\item Dany jest nieskończony ciąg arytmetyczny taki, że $a_5=18$. Wyrazy $a_1$, $a_3$ i $a_13$ tego ciągu są odpowiednio pierwszym, drugim i trzecim wyrazem pewnego ciągu geometrycznego. Wyznacz wzór ogólny tego ciągu arytmetycznego.
	
	\item Trzy liczby, które tworzą rosnący ciąg geometryczny, dają w sumie 35. Jeśli do pierwszej liczby dodamy 4, do drugiej 5, a do trzeciej 1, to otrzymane sumy w tej kolejności dadzą ciąg arytmetyczny. Wyznacz ten ciąg geometryczny.
	
	
\end{enumerate}
\end{document}