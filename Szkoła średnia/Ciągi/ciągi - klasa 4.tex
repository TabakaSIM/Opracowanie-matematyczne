\documentclass[12pt,a4paper]{article}
\usepackage[utf8]{inputenc} %polskie znaki
\usepackage[T1]{fontenc}	%polskie znaki
\usepackage{amsmath}		%matematyczne znaczki :3
\usepackage{enumerate}		%Dodatkowe opcje do funkcji enumerate
\usepackage{geometry} 		%Ustawianie marginesow
\usepackage{graphicx}		%Grafika
\usepackage{wrapfig}		%Grafika obok textu
\usepackage{float}			%Allows H in fugire
%\pagestyle{empty} 			%usuwa nr strony

\newgeometry{tmargin=2cm, bmargin=2cm, lmargin=2cm, rmargin=2cm} 

\begin{document}
	\begin{center}
		\LARGE Ciągi
	\end{center}
	\vspace{1cm}
	
	
	\begin{enumerate}[1.]
		\item Wyznacz 3 kolejne wyrazy podanego ciągu:
		
		\begin{enumerate}[a)] \begin{tabular}{p{7cm} p{7cm}} 
				\item $1,-2,3,-4,5,-6,\dots $& \vspace{0.4cm} \item $5,7,9,11,13,15,\dots $ \\
				\item $0,0,0,0,0,0,\dots $& \item $0,1,1,2,3,5,8,\dots $ \\
				\item $0,\frac{1}{2},\frac{2}{3},\frac{3}{4},\frac{4}{5},\dots $& \item $1,2,4,8,16,32,\dots $ \\
		\end{tabular} \end{enumerate}
	
		\item Zapisz wzór na ogólny wyraz ciągu:
	
		\begin{enumerate}[a)] \begin{tabular}{p{7cm} p{7cm}} 
			\item $1,2,3,4,5,6,\dots $& \vspace{0.4cm} \item $\frac{1}{2},\frac{1}{4},\frac{1}{6},\frac{1}{8},\frac{1}{10},\dots $ \\
			\item $3,6,9,12,15,18,\dots $& \item $5,3,1,-1,-3,-5,\dots $ \\
			\item $1,-1,1,-1,1,-1,\dots $& \item $1,4,9,16,25,36,\dots $ \\
		\end{tabular} \end{enumerate}
		
		\item Wyznacz pierwsze 5 wyrazów ciągu o wzorze:
		
		\begin{enumerate}[a)] \begin{tabular}{p{7cm} p{7cm}}
				\item $a_n=n+5 $& \vspace{0.4cm} \item $a_n=-n+7 $ \\
				\item $a_n=\frac{1}{2}n+3 $& \item $a_n=\sqrt{n-1} $ \\
				\item $a_n=3^n $& \item $a_n=\frac{n+3}{2n+1} $ \\
		\end{tabular} \end{enumerate}
	
		\item Zbadaj monotoniczność ciągu:
	
		\begin{enumerate}[a)] \begin{tabular}{p{7cm} p{7cm}}
			\item $a_n=n-3 $& \vspace{0.4cm} \item $a_n=2n+1 $ \\
			\item $a_n=-3n+3 $& \item $a_n=n^2+3 $ \\
			\item $a_n=n^2-5n $& \item $a_n=5 $ \\
		\end{tabular} \end{enumerate}

		
	\end{enumerate}
\end{document}