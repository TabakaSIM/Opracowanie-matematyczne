\documentclass[12pt,a4paper]{article}
\usepackage[utf8]{inputenc} %polskie znaki
\usepackage[T1]{fontenc}	%polskie znaki
\usepackage{amsmath}		%matematyczne znaczki :3
\usepackage{enumerate}		%Dodatkowe opcje do funkcji enumerate
\usepackage{geometry} 		%Ustawianie marginesow
\usepackage{graphicx}		%Grafika
\usepackage{wrapfig}		%Grafika obok textu
\usepackage{float}			%Allows H in fugire
\pagestyle{empty} 			%usuwa nr strony

\newgeometry{tmargin=2cm, bmargin=2cm, lmargin=2cm, rmargin=2cm} 

\begin{document}
	\begin{enumerate}[1.]
		\item Wyznacz równanie prostej $AB$, gdzie $A=(-5,4)$, $B=(1,0)$. Naszkicuj równanie tej prostej oraz wyznacz jej miejsce zerowe (graficznie oraz algebraicznie).
		\item Wyznacz równanie prostej prostopadłej do prostej $k: \: y=-\frac{2}{3}x+5$ przechodzącej przez środek układu współrzędnych.
		\item Wyznacz równanie prostej równoległej do prostej $k: \: y=2x-7$ przechodzącej przez punkt $P=(-5,3)$
		\item Wyznacz równanie prostej symetralnej do odcinka $AB$, gdzie $A=(8,2)$, $B=(-4,-6)$.
		\item Wyznacz dla jakiego parametru "m" prosta $y=(2m+3)x-4m-3$ jest malejąca.
	\end{enumerate}


	W trójącie $ABC$ dane są $|AC|=3\sqrt{3}$, $|BC|=3$ i $\angle BAC=30^\circ$. Wyznacz brakujące kąty tego trójkąta.
\end{document}