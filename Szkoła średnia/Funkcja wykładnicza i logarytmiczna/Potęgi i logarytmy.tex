\documentclass[12pt,a4paper]{article}
\usepackage[utf8]{inputenc} %polskie znaki
\usepackage[T1]{fontenc}	%polskie znaki
\usepackage{amsmath}		%matematyczne znaczki :3
\usepackage{enumerate}		%Dodatkowe opcje do funkcji enumerate
\usepackage{geometry} 		%Ustawianie marginesow
\usepackage{graphicx}		%Grafika
\usepackage{wrapfig}		%Grafika obok textu
\usepackage{float}			%Allows H in fugire
%\pagestyle{empty} 			%usuwa nr strony

\newgeometry{tmargin=2cm, bmargin=2cm, lmargin=2cm, rmargin=2cm} 

\begin{document}
	\begin{center}
		\LARGE Funkcja wykładnicza i logarytmiczna
	\end{center}
	\vspace{1.5cm}
	
	\begin{enumerate}[1.]
		
		\item Oblicz (zapisz w postaci wykładniczej):
		
		\begin{enumerate}[a)] \begin{tabular}{p{7cm} p{7cm}}
				\item $6^5\cdot6^{-3}:6^7=$& \vspace{0.4cm}\item $2^{10}\cdot2^{-7}:2^{-3}=$ \\
				\large\item $\frac{5^{21}:5^6}{5^{10}}=$& \item$\frac{8^5}{4^7}=$ \\\normalfont
				\item $32\cdot16^2\cdot8^3=$& \item$25\cdot5^2\cdot125^3=$ \\
				\item $8\cdot16^3\cdot32^2=$& \item$6^{10}\cdot36^{-3}:6^5=$ \\
				\item $(-4)^6\cdot4^3\cdot(-4)^{-2}=$& \item$\frac{13^{12}}{(-13)^{-12}}=$ \\
				\large\item $\frac{(2^7)^4}{2^{30}:2^{14}}=$& \item$32^6:2^{18}=$ \\
				\item $\frac{44^4}{22^3}=$& \item$\frac{81\cdot25}{15^2}=$ \\
				\item $\frac{8^3\cdot6^3}{27\cdot4^5}=$& \item$\frac{50\cdot8^2}{250\cdot2^2}=$ \\\normalfont

		\end{tabular} \end{enumerate}
	
	\item Zapisz w postaci $a^x$:
	
			\begin{enumerate}[a)] \begin{tabular}{p{7cm} p{7cm}} \large
			\item $\frac{(a^5)^3\cdot a^7}{a^3:(a^{-4})^{-2}}=$& \vspace{0.4cm}\item $\frac{(a^3)^4:a}{a^{-9}:a^3}=$ \\ \normalfont
			\item $(a^7:(a^3)^{-2})^{-1}\cdot(a^{-7}\cdot a^5)^2=$& \item$(\frac{1}{a}^5\cdot a^7):(\frac{1}{a^2}^{-3}:a^4)=$ \\
	\end{tabular} \end{enumerate}

	\item Oblicz korzystając z własności potęg:
	
	\begin{enumerate}[a)] \begin{tabular}{p{7cm} p{7cm}}
			\item $\frac{3^3\cdot3^6+(3^3)^3}{3^8}=$& \vspace{0.4cm}\item $\frac{8^4:(2^7)^{-2}-4^12}{2^24}=$ \\
			\item $\frac{(2^2)^3-2^8:4^2}{8}=$& \item$\frac{2^{102}+2^{103}+2^{104}}{3\cdot4^{25}}=$ \\
			\item $(a^7:(a^3)^{-2})^{-1}\cdot(a^{-7}\cdot a^5)^2=$& \item$(3^7\cdot\frac{1}{9}^3)^{-3}:(\frac{1}{3}^7:\frac{1}{27})^{4}=$ \\
	\end{tabular} \end{enumerate}
	
	
	\item Wykaż, że liczba 
	$$k_n = 3^n+3^{n+1}+3^{n+2}$$
		określona dla wszystkich $n\geq1$ jest podzielna przez 13.
		
	\item Wykaż, że liczba $a=8^{10}+4^{16}+3\cdot16^8$ jest podzielna przez 17.
	\newpage
	\item Oblicz:	
			\begin{enumerate}[a)] \begin{tabular}{p{7cm} p{7cm}}
			\item $\sqrt{36}=$& \vspace{0.4cm}\item $\sqrt{18}=$ \\
			\item $\sqrt{200}=$& \item$\sqrt[3]{27}=$ \\
			\item $\sqrt[3]{250}=$& \item$\sqrt[5]{64}=$ \\
			\item $\sqrt{9}+\sqrt{16}=$& \item$\sqrt{8}+\sqrt{32}=$ \\
			\item $4\sqrt{2}+\sqrt{8}=$& \item$\sqrt{200}-\sqrt{50}=$ \\
			\item $\sqrt{32}-3\sqrt{2}=$& \item$\sqrt{800}+\sqrt{242}-\sqrt{162}=$ \\
			\item $\sqrt{48}-\sqrt{3}=$& \item$\sqrt{12}-\sqrt{27}=$ \\
			\item $3\sqrt{20}-\frac{1}{3}\sqrt{45}-5\sqrt{180}=$& \item$\sqrt{10}\cdot\sqrt{40}=$ \\
			\item $\sqrt{32}:\sqrt{2}=$& \item$\frac{\sqrt{72}}{\sqrt{18}}=$ \\
			
	\end{tabular} \end{enumerate}

	\item Oblicz:
	
		\begin{enumerate}[a)] \begin{tabular}{p{7cm} p{7cm}}
				\item $32^\frac{1}{5}=$& \vspace{0.4cm}\item $4^\frac{1}{2}=$ \\
				\item $\sqrt[3]{2\sqrt{2}}=$& \item$\frac{1}{2\sqrt[3]{2}}\cdot\sqrt[3]{2}^2:2^{-2\frac{1}{2}}=$ \\
				
		\end{tabular} \end{enumerate}
	
	\item Naszkicuj wykres funkcji wykładniczej:
	
		\begin{enumerate}[a)] \begin{tabular}{p{7cm} p{7cm}}
			\item $f(x)=2^x$& \vspace{0.4cm}\item $g(x)=3^{-x}$ \\
			\item $h(x)=2^{x+3}-2$& \item$i(x)=\frac{1}{2}^{x-2}+1$ \\
	\end{tabular} \end{enumerate}
	
	
	\newpage
	
	\item Oblicz:
	
			\begin{enumerate}[a)] \begin{tabular}{p{7cm} p{7cm}}
			\item $\log_28=$& \vspace{0.4cm}\item $\log_232=$ \\
			\item $\log_2512=$& \item$\log_4256=$ \\
			\item $\log_381=$& \item$\log_981=$ \\
			\item $\log_3\frac{1}{27}=$& \item$\log_6\frac{1}{216}=$ \\
			\item $\log_\frac{1}{2}\frac{1}{16}=$& \item$\log_\frac{2}{3}\frac{3}{2}=$ \\
			\item $\log_749=$& \item$\log_51=$ \\
			\item $\log_\frac{1}{4}16=$& \item$\log_5\frac{1}{625}=$ \\
			\item $\log100=$& \item$\log0,001=$ \\
			
	\end{tabular} \end{enumerate}

		\item Oblicz:
	
	\begin{enumerate}[a)] \begin{tabular}{p{7cm} p{7cm}}
			\item $\log_48=$& \vspace{0.4cm}\item $\log_9\frac{1}{27}=$ \\
			\item $\log_\frac{1}{2}16\sqrt{2}=$& \item$\log_{\sqrt{3}}9=$ \\
			\item $\log_6\sqrt{216}=$& \item$\log_{3\sqrt[3]{3}}\sqrt{3}=$ \\
			\item $\log_5125\sqrt{5}=$& \item$\log_{3\sqrt{3}}81\sqrt[5]{3}=$ \\
			
	\end{tabular} \end{enumerate}
		
	\end{enumerate}
\end{document}