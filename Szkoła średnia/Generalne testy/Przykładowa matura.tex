\documentclass[12pt,a4paper]{article}
\usepackage[utf8]{inputenc} %polskie znaki
\usepackage[T1]{fontenc}	%polskie znaki
\usepackage{amsmath}		%matematyczne znaczki :3
\usepackage{enumerate}		%Dodatkowe opcje do funkcji enumerate
\usepackage{geometry} 		%Ustawianie marginesow
\usepackage{graphicx}		%Grafika
\usepackage{wrapfig}		%Grafika obok textu
\usepackage{float}			%Allows H in fugire
\usepackage{xcolor}     	% for colour
\usepackage{lipsum}     	% for sample text
\usepackage{ntheorem}   	% for theorem-like environments
\usepackage{mdframed}   	% for framing
%\usepackage{amsthm}		%Dodaje przerwa u góry w twierdzeniach
%\pagestyle{empty} 			%usuwa nr strony

\theoremstyle{break}
\theoreminframepreskip{0.5cm}
\theoremheaderfont{\bfseries}
\newmdtheoremenv[%
linecolor=white,%
innertopmargin=\topskip,
shadowsize=0,%
innertopmargin=5,%
innerbottommargin=5,%
leftmargin=10,%
rightmargin=10,%
backgroundcolor=gray!20,%
innertopmargin=0pt,%
ntheorem]{zad}{Zadanie}


\newgeometry{tmargin=2cm, bmargin=2cm, lmargin=2cm, rmargin=2cm} 

\begin{document}
	
	\begin{zad}[0-1]
		\textbf{Dokończ zdanie. Wybierz właściwą odpowiedź spośród podanych.}
	\end{zad} 
	
	Połowa liczby \Large$\frac{4^{150};\cdot;4^{50}}{4^{100}}:$\normalsize wynosi:
	
	\vspace{0.5cm}
	\begin{tabular}{p{3.5cm} p{3.5cm} p{3.5cm} p{3.5cm}}
		\textbf{A. }$2^{100}$&
		\textbf{B. }$2^{50}$&
		\textbf{C. }$4^{99}$&
		\textbf{D. }$4^{50}$\\
	\end{tabular}
	
	%%%%%%%%%%%%%%%%%%%%%%%%%%%%%
	
	\begin{zad}[0-1]
		\textbf{Dokończ zdanie. Wybierz właściwą odpowiedź spośród podanych.}
	\end{zad} 
	
	Liczba $2\log_69-\log_6\frac{3}{8}$ wynosi:
	
	\vspace{0.5cm}
	\begin{tabular}{p{3.5cm} p{3.5cm} p{3.5cm} p{3.5cm}}
		\textbf{A. }$\log_617\frac{5}{8}$&
		\textbf{B. }$3$&
		\textbf{C. }$\frac{1}{3}$&
		\textbf{D. }$16$\\
	\end{tabular}
	
	%%%%%%%%%%%%%%%%%%%%%%%%%%%%%
	
	\begin{zad}[0-1]
		\textbf{Dokończ zdanie. Wybierz właściwą odpowiedź spośród podanych.}
	\end{zad} 
	
	Wartość wyrażenia \large$\frac{42^6}{6^6\cdot7^5}\:$\normalsize wynosi:
	
	\vspace{0.5cm}
	\begin{tabular}{p{3.5cm} p{3.5cm} p{3.5cm} p{3.5cm}}
		\textbf{A. }$\frac{6}{7^5}$&
		\textbf{B. }$\frac{6^6}{7^7}$&
		\textbf{C. }$6$&
		\textbf{D. }$\frac{1}{7^5}$\\
	\end{tabular}
	
	%%%%%%%%%%%%%%%%%%%%%%%%%%%%%
	
	\begin{zad}[0-1]
		\textbf{Dokończ zdanie. Wybierz właściwą odpowiedź spośród podanych.}
	\end{zad} 
	
	Dla pewnego ostrego kąta $\alpha$ dane jest, że $\sin\alpha = \frac{2\sqrt{2}}{3}$. Wówczas $\text{tg }\alpha$ wynosi
	
	\vspace{0.5cm}
	\begin{tabular}{p{3.5cm} p{3.5cm} p{3.5cm} p{3.5cm}}
		\textbf{A. }$2\sqrt{2}$&
		\textbf{B. }$\frac{\sqrt{2}}{4}$&
		\textbf{C. }$3$&
		\textbf{D. }$\frac{1}{3}$\\
	\end{tabular}

	%%%%%%%%%%%%%%%%%%%%%%%%%%%%%
	
	\begin{zad}[0-1]
		\textbf{Dokończ zdanie. Wybierz właściwą odpowiedź spośród podanych.}
	\end{zad} 
	
	Wartośc wyrażenia $(2a-3b)^2-(-2a-3b)^2$ wynosi
	
	\vspace{0.5cm}
	\begin{tabular}{p{3.5cm} p{3.5cm} p{3.5cm} p{3.5cm}}
		\textbf{A. }$24ab$&
		\textbf{B. }$18b^2$&
		\textbf{C. }$-24ab$&
		\textbf{D. }$0$\\
	\end{tabular}

	%%%%%%%%%%%%%%%%%%%%%%%%%%%%%

	\begin{zad}[0-2]
		Wykaż, że dla dowolnych dla dowolnej liczby całkowitej $k$ wyrażenie $k^3+3k^2-40k$ jest podzielne przez 6.
	\end{zad} 

	%%%%%%%%%%%%%%%%%%%%%%%%%%%%%
	
	\begin{zad}[0-1]
		\textbf{Dokończ zdanie. Wybierz właściwą odpowiedź spośród podanych.}
	\end{zad} 
	
	Dla pewnego ostrego kąta $\alpha$ dane jest, że $\sin\alpha = \frac{2\sqrt{2}}{3}$. Wówczas $\text{tg }\alpha$ wynosi
	
	\vspace{0.5cm}
	\begin{tabular}{p{3.5cm} p{3.5cm} p{3.5cm} p{3.5cm}}
		\textbf{A. }$2\sqrt{2}$&
		\textbf{B. }$\frac{\sqrt{2}}{4}$&
		\textbf{C. }$3$&
		\textbf{D. }$\frac{1}{3}$\\
	\end{tabular}

	%%%%%%%%%%%%%%%%%%%%%%%%%%%%%
	\newpage
	\begin{zad}[0-1]
		\textbf{Dokończ zdanie. Wybierz właściwą odpowiedź spośród podanych.}
	\end{zad} 
	
	Rówanie \large$\frac{(x^2-5)(x-3)}{(-2x+6)(x-5)}$$=0\quad$\normalsize ma 
	
	\vspace{0.5cm}
	\begin{tabular}{p{14cm}}
		\textbf{A. }zero rozwiązań\\
		\textbf{B. }jedno rozwiązanie\\
		\textbf{C. }dwa rozwiązania\\
		\textbf{D. }trzy rozwiązania\\
	\end{tabular}

	%%%%%%%%%%%%%%%%%%%%%%%%%%%%%
	
	\begin{zad}[0-3]
		Rozwiąż równanie
		$$x^6+8=7x^3$$
	\end{zad} 

	\begin{zad}[0-2]
		\textbf{Dokończ zdanie. Wybierz \underline{dwie} właściwe odpowiedzi spośród podanych.}
	\end{zad} 

	Poniżej podano pary pewnych prostych.
	\\
	
	Pary prostych prostopadłych to pary: ...... i ...... .
	
	\vspace{0.5cm}
	\begin{tabular}{p{14cm}}
		\textbf{A. }$y=4x-5$ i $y=\frac{1}{4}x-5$\\
		\\
		\textbf{B. }$y=\frac{1}{4}x+5$ i $y=-4x+5$\\
		\\
		\textbf{C. }$y=4x-5$ i $y=4x-\sqrt{5}$\\
		\\
		\textbf{D. }$2x-3y-7=0$ i $2x+3y+7=0$\\
		\\
		\textbf{E. }$4x-5y+6=0$ i $5x+4y-6=0$\\
		\\
		\textbf{F. }$x+5y+4=0$ i $5x+y-4=0$\\
	\end{tabular}
		

	

\end{document}