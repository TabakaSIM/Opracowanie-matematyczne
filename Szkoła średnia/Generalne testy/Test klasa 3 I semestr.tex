\documentclass[12pt,a4paper]{article}
\usepackage[utf8]{inputenc} %polskie znaki
\usepackage[T1]{fontenc}	%polskie znaki
\usepackage{amsmath}		%matematyczne znaczki :3
\usepackage{enumerate}		%Dodatkowe opcje do funkcji enumerate
\usepackage{geometry} 		%Ustawianie marginesow
\usepackage{graphicx}		%Grafika
\usepackage{wrapfig}		%Grafika obok textu
\usepackage{float}			%Allows H in fugire
\pagestyle{empty} 			%usuwa nr strony

\newgeometry{tmargin=2cm, bmargin=2cm, lmargin=2cm, rmargin=2cm} 

\begin{document}
	\begin{center}
		\LARGE Test matematyka - klasa 3 - I semestr
	\end{center}
	\vspace{1.5cm}
	\begin{tabular}{p{13cm} r}
		Imię i nazwisko: ............................................................................
		&[....../30pkt]\\ 
		\vspace{0.5cm}
	\end{tabular}
	\begin{enumerate}[1.]
		
		\item  \begin{tabular}{p{13cm} r}	%Zad3
	Oblicz promień okręgu opisanego na trójkącie prostokątnym o przyprostokątnych długości 10cm i 24cm.&[4pkt]\\ 
\end{tabular}
		
		\item  \begin{tabular}{p{13cm} r}
	Blaszany dach pewnego budynku został skonstruowany pod kątem $37^\circ$. Wiedząc, że wysokość samej części dachu wynosi 4m, oblicz jakiej długości powinny być płaty blachy pokrywające dach. 
	&[3pkt]\\ 
\end{tabular}

		\item \begin{tabular}{p{13cm} r}
	Wiemy o pewnym kącie $\alpha$, że $\cos \alpha = \frac{12}{13}$ oraz $\alpha\in (270^\circ,360^\circ)$. Oblicz pozostałe funkcje trygonometryczne tego kąta.
	&[4pkt]\\
\end{tabular}
		
		\item \begin{tabular}{p{13cm} r}
	Oblicz: &[6pkt]\\ 
\end{tabular}
\begin{enumerate}[a)]
	\item $\sin300^\circ=$
	\item $\cos 660^\circ= $
	\item $(\sin 60^\circ + \cos 30^\circ) : \cos210^\circ= $
	\item $(\text{tg }225^\circ \cdot \text{tg } 120^\circ) \cdot \sin (-210^\circ)=$
\end{enumerate}

		\begin{tabular}{p{13cm} r}
	\item Rozwiąż trójkąt o bokach 6 i 8 oraz kącie między nimi $30^\circ$. &[4pkt]\\ 
\end{tabular}

		
		\begin{tabular}{p{13cm} r}
	\item Dany jest trójkąt $ABC$, w którym bok $AB$ jest o 6 krótszy od boku $AC$ oraz $|BC|=5\sqrt{2}$. Wiedząc, że $\angle ABC = 135^\circ$: &[9pkt]\\ 
\end{tabular}

\begin{enumerate}[a)]
	\item Oblicz boki $AB$ i $AC$
	\item Oblicz pole tego trójkąta
	\item Wyznacz pozostałe kąty tego trójkąta
	\item Oblicz promień okręgu opisanego na tym trójkącie
\end{enumerate}
		
	\end{enumerate}
	
\end{document}