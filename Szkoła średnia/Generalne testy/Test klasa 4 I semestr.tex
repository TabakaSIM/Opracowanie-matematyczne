\documentclass[12pt,a4paper]{article}
\usepackage[utf8]{inputenc} %polskie znaki
\usepackage[T1]{fontenc}	%polskie znaki
\usepackage{amsmath}		%matematyczne znaczki :3
\usepackage{enumerate}		%Dodatkowe opcje do funkcji enumerate
\usepackage{geometry} 		%Ustawianie marginesow
\usepackage{graphicx}		%Grafika
\usepackage{wrapfig}		%Grafika obok textu
\usepackage{float}			%Allows H in fugire
\pagestyle{empty} 			%usuwa nr strony

\newgeometry{tmargin=2cm, bmargin=2cm, lmargin=2cm, rmargin=2cm} 

\begin{document}
	\begin{center}
		\LARGE Test matematyka - klasa 4 - I semestr
	\end{center}
	\vspace{1.5cm}
	\begin{tabular}{p{13cm} r}
		Imię i nazwisko: ............................................................................
		&[....../35pkt]\\ 
		\vspace{0.5cm}
	\end{tabular}
	\begin{enumerate}[1.]
		\item  \begin{tabular}{p{13cm} r}
			Oblicz obwód trójkąta o wierzchołkach $A=(-3,-5)$,$B=(9,0)$, $C=(6,4)$.
			&[4pkt]\\ 
		\end{tabular}
	
	\item \begin{tabular}{p{13cm} r}
		Wyznacz równanie prostej $AB$, gdzie $A=(-3,-1)$, $B=(1,1)$.
		&[5pkt]\\
	\end{tabular}

	\item \begin{tabular}{p{13cm} r}
		Wyznacz porstą prostopadłą do prostej $k:\:y=\frac{2}{3}x+5\sqrt{2}$ przechodzącą przez punkt $P=(6,-3)$.&[3pkt]\\
	\end{tabular}

	\item \begin{tabular}{p{13cm} r}
		Rozwiąż układ równań: &[6pkt]\\ 
	\end{tabular}
	$$\left\{\begin{array}{l}
		x^2+y^2+4x-6y+4=0\\
		x-y=-2
	\end{array}\right.$$

	\item  \begin{tabular}{p{13cm} r}
		Dany jest ciąg arytmetyczny $a_n$, którego suma trzeciego i piątego wyrazu wynosi 14, a różnica czwartego i siódmego wynosi -12. Wyznacz wzór ogólny tego ciągu. 
		&[5pkt]\\ 
	\end{tabular}

	\item \begin{tabular}{p{13cm} r}
		Wyznacz paramert "k", dla którego trójwyrazowy ciąg
		&[5pkt]\\
	\end{tabular}
	$$(k^2-5,\quad k+1, \quad k^2+4k-5)$$
	jest ciągiem arytmetycznym. 
	
	\item \begin{tabular}{p{13cm} r}
		Oblicz sumę:&[3pkt]\\
	\end{tabular}
	$$S=2+5+8+11+\dots + 158$$ 
	
	\item \begin{tabular}{p{13cm} r}
		Sprawdź czy podany ciąg jest ciagiem geometrycznym.&[4pkt]\\ 
	\end{tabular}
	$$(2\sqrt{5}-\sqrt{7},\quad \sqrt{13}, \quad 2\sqrt{5}+\sqrt{7})$$

	\end{enumerate}
	
\end{document}