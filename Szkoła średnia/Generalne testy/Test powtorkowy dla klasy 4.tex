\documentclass[12pt,a4paper]{article}
\usepackage[utf8]{inputenc} %polskie znaki
\usepackage[T1]{fontenc}	%polskie znaki
\usepackage{amsmath}		%matematyczne znaczki :3
\usepackage{enumerate}		%Dodatkowe opcje do funkcji enumerate
\usepackage{geometry} 		%Ustawianie marginesow
\usepackage{graphicx}		%Grafika
\usepackage{wrapfig}		%Grafika obok textu
\usepackage{float}			%Allows H in fugire
\pagestyle{empty} 			%usuwa nr strony

\newgeometry{tmargin=2cm, bmargin=2cm, lmargin=2cm, rmargin=2cm} 


\begin{document}
	
	\begin{center}
		\LARGE Test powtórkowy z klasy 3
	\end{center}
	\vspace{1.5cm}
	
	\begin{enumerate}[1.]
		
	\item \begin{tabular}{p{13cm} r}
	Oblicz długość boku kwadratu, jesli jego przekątna jest o 2 cm dłuższa od boku. &[4pkt]\\ 
	\end{tabular}

	\item \begin{tabular}{p{13cm} r}
	Pole trapezu jest równe $72 cm^2$, zaś pole koła wpisanego w ten trapez wynosi $16\pi cm^2$. Oblicz długość ramion trapezu wiedąc, że ten trapez jest równoramienny.&[4pkt]\\ 
	\end{tabular}

	\item \begin{tabular}{p{13cm} r}
	Oblicz pole rombu o boku równym 5 cm i kącie ostrym $30^\circ$. &[3pkt]\\ 
	\end{tabular}

	\item \begin{tabular}{p{13cm} r}
	Trapez, w którym można opisać okrąg i można opisać na okęgu, ma podstawy długości 12 cm i 3 cm. Oblicz pole tego trapezu. &[3pkt]\\ 
	\end{tabular}

	\item \begin{tabular}{p{13cm} r}
	Prostokąt $ABCD$ ma boki równe 4cm i 8cm, natomiast prostokąt niego podobny ma przekątną o długości $\sqrt{180}$. Oblicz pole drugiego prostokąta. &[4pkt]\\ 
	\end{tabular}

	\item \begin{tabular}{p{13cm} r}
	Oblicz: &[5pkt]\\ 
	\end{tabular}
	\begin{enumerate}[a)]
		\item $\sin(30^\circ)=$
		\item $\cos(150^\circ)=$
		\item $\sin(-30^\circ)+\cos(420^\circ)=$ 
	\end{enumerate}

	\item \begin{tabular}{p{13cm} r}
	Dla pewnego kąta ostrego $\alpha$ mamy $tg(\alpha)=\frac{5}{12}$. Oblicz pozostałe funkcje trygonometryczne tego kąta (tj. sin, cos, ctg). &[4pkt]\\ 
	\end{tabular}

	\end{enumerate}
\end{document}