\documentclass[12pt,a4paper]{article}
\usepackage[utf8]{inputenc} %polskie znaki
\usepackage[T1]{fontenc}	%polskie znaki
\usepackage{amsmath}		%matematyczne znaczki :3
\usepackage{enumerate}		%Dodatkowe opcje do funkcji enumerate
\usepackage{geometry} 		%Ustawianie marginesow
\usepackage{graphicx}		%Grafika
\usepackage{wrapfig}		%Grafika obok textu
\usepackage{float}			%Allows H in fugire
\pagestyle{empty} 			%usuwa nr strony

\newgeometry{tmargin=2cm, bmargin=2cm, lmargin=2cm, rmargin=2cm} 

\begin{document}
	\begin{center}
		\LARGE Pytania ustne - klasa 3 - semestr I
	\end{center}
	\vspace{1.5cm}
	\begin{enumerate}[1.]

	\item Wyjaśnij konstrukcję okręgu opisanego na trójkącie.
	
	\item Wyjaśnij konstrukcję okręgu wpisanego w okrąg.
	
	\item Dany jest trójkąt prostokątny o przyprostokątnej 5 i przeciwprostokątnej 7. Oblicz wysokość padającą na przeciwprostokątną.

	\item W trójkącie prostokątnym o przyprostokątnych 6 i 8, poprowadzono środkową przeciwprostokątnej, taką która przecina ten trójkąt w punktach $A$ i $B$. Oblicz długość odcinka $AB$.
	
	\item Promień okręgu wpisanego w trójkąt równoboczny ma długość 3. Oblicz długość boku tego trójkąta.
	
	\item Oblicz pole i obwód trójkąta równoramiennego o ramieniu długości 6, wiedząc że dolna podstawa jest trzy razy dłuższa od górnej, a kąt między rameniem i tą podstawą wynosi $60^\circ$.
	
	\item Oblicz promień okręgu wpisanego w kwadrat o przekątnej równej 2.
	
	\item Oblicz pole trójkąta o bokach 5 i 8 oraz kącie między nimi $120^\circ$.
	
	\item Oblicz promień okręgu opisanego na prostokącie o bokach 12 i 16.
	
	\item Dany jest $\cos \alpha = \frac{3}{5}$, dla pewnego kąta $\alpha \in (270^\circ,360^\circ)$. Wyznacz jego pozostałe funkcje trygonometryczne.
	
	\item Dany jest trójkąt prostokątny o przeciwprostokątnej 10 i najmniejszym kącie $20^\circ$. Wyznacz jego dłuższą przyprostokątną.
	
	\item Oblicz:
	$\sin120^\circ - \cos240^\circ \cdot \sin(-330^\circ)=$
	
	\item Oblicz dokładną wartość: 
	$\sin 15^\circ=$
	
	\item $6\cdot (\sin(30^\circ)\cdot\cos(45^\circ)\cdot\text{ctg }(60^\circ)):(\text{ctg }(30^\circ)\cdot\sin(45^\circ))=$	
	
	\item $12\cdot(\text{tg }(60^\circ)-\cos(60^\circ))\cdot(\text{tg }(30^\circ)+\text{tg }(30^\circ)+\cos(30^\circ))=$	

	\item Dany jest trójkąt $ABC$ o bokach 4 i 6 oraz kącie między nimi równy $60^\circ$. Oblicz pole tego trójkąta.
	
	\item Dany jest trójkąt $ABC$ o bokach $|AB|=8$, $|BC|=12$ i $|AC|=7$. Oblicz największy kąt tego trójkąta.
	 
	\item W trójkącie $ABC$ mamy dane $|BC|=4$ i $\angle BAC=150^\circ$. Oblicz promień koła opisanego na tym trójkącie.
	
	\item W trójkącie $ABC$ są dane $|BC|=5$, $\angle BAD=48^\circ$ oraz $\angle ACB=70^\circ$. Oblicz długość boku $AC$ tego trójkąta.
	
	\item Dany jest trójkąt $ABC$, którego boki są równe $|AB|=4,5$, $|BC|=6,2$ i $|AC|=3,7$. Oblicz najmniejszy kąt tego trójkąta.
	
	\item Oblicz pole trójkąta równoramiennego o ramieniu równym 8 i kącie przy podstawie $75^\circ$.
	
	\item Promienie słoneczne padają pod kątem $15^\circ$. Oblicz długość cienia, który rzuca masz mający 12,5m wysokości.
	
	\item Sprawdź czy podane równanie jest toższamością trygonometryczną:
	$$1-2\sin^2\alpha = 2\cos^2\alpha - 1$$
	
	\item Sprawdź czy podane równanie jest toższamością trygonometryczną:
	$$\sin\alpha \cdot (\frac{1}{\sin\alpha}-\sin\alpha)=\cos^2\alpha$$
	
	\item Sprawdź czy podane równanie jest toższamością trygonometryczną:
	$$(\sin\alpha + \cos^2\alpha)=1$$
	
	\item Sprawdź czy podane równanie jest toższamością trygonometryczną:
	$$\sin\alpha + \sin\alpha \cdot \text{tg }^2\alpha=\frac{\text{tg }\alpha}{\cos\alpha}$$
	
	\item Omów boki trójkątów 45,45,90 i 30,60,90.
	
	\item Czy da się stworzyć trójkąt o kątach $70^\circ,100^\circ,30^\circ$?\newline
	Czy da się utworzyć trójkąt o bokach $15,10,8$?\newline
	\textit{Odpowiedź uzasadnij.}
	
	\item Czworokąt $ABCD$ wpisano w okrąg tak, że bok $AB$ jest średnicą tego okręgu. Udowodnij, że $|AD|^2+|BD|^2=|BC|^2+|AC|^2$.
		
	\end{enumerate}
	
\end{document}