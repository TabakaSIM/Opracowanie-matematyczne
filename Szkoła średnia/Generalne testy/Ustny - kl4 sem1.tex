\documentclass[12pt,a4paper]{article}
\usepackage[utf8]{inputenc} %polskie znaki
\usepackage[T1]{fontenc}	%polskie znaki
\usepackage{amsmath}		%matematyczne znaczki :3
\usepackage{enumerate}		%Dodatkowe opcje do funkcji enumerate
\usepackage{geometry} 		%Ustawianie marginesow
\usepackage{graphicx}		%Grafika
\usepackage{wrapfig}		%Grafika obok textu
\usepackage{float}			%Allows H in fugire
\pagestyle{empty} 			%usuwa nr strony

\newgeometry{tmargin=2cm, bmargin=2cm, lmargin=2cm, rmargin=2cm} 

\begin{document}
	\begin{center}
		\LARGE Pytania ustne - klasa 4 - semestr I
	\end{center}
	\vspace{1.5cm}
	\begin{enumerate}[1.]
		
		\item Sprawdź czy podane punkty $ABCD$ tworzą równoległobok. Odpowiedź uzasadnij.\newline
		$A=(1,-4)$, $B=(2,-1)$, $C=(1,2)$, $D=(-1,-1)$.
		
		\item Oblicz obwód trójkąta $ABC$, gdzie $A=(-2,-1)$, $B=(4,-1)$, $C=(1,5)$.
		
		\item Oblicz równanie prostej $AB$, gdzie $A=(2,-3)$, $B=(-1,3)$.
		
		\item Oblicz równanie prostej $AB$, gdzie $A=(-1,4)$, $B=(3,2)$.
		
		\item Wyznacz równanie prostej prostopadłej do prostej $y=3x-1$ przechodzącej przez punkt $P=(-6,1)$.
		
		\item Wyznacz równanie symetralnej odcinka $AB$, gdzie $A=(2,6)$, $B=(0,2)$.
		
		\item Wyznacz równanie okręgu
		$$x^2+y^2-4x+6y-12=0$$
		w postaci kanonicznej.
		
		\item Dany jest równoległobok $ABCD$, gdzie $A=(-1,3)$, $B=(-4,-2)$ oraz punkt $S=(2,2)$ który jest środkiem symetrii tego równoległoboku. Wyznacz punkty $C$ i $D$.
		
		\item		Rozwiąż układ równań:
$$\left\{\begin{array}{l}
x^2+y^2+2x+2y-5=0\\
x+y=-2
\end{array}\right.$$

		\item Wyznacz środek i promień okręgu $x^2+y^2-6y+8x=0$
		
		\item Omów rónanie prostej w postaci kierunkowej i postaci ogólnej. Jaka jest zasadnicza różnica między tymi postaciami?
		
		\item Dwie proste $y=(a+1)x-3$ i $y=(4-2a)x-5$ są równoległe. Wyznacz "a".
		
		\item Dwie proste $y=(2a+3)x-3$ i $y=\frac{1}{3}x-5$ są prostopadłe. Wyznacz "a".
		
		\item Określ monotoniczność prostej $2x-3y+5=0$. Przez które ćwiartki układu współrzędnych przechodzi ta prosta?
		
		\item Oblicz równanie prostej $AB$, gdzie $A=(-3,-4)$, $B=(-3,2)$.
		
		\item Zbadaj monotoniczność ciągu: $a_n=n^2-6n$.
		
		\item Zbadaj monotoniczość ciągu: $a_n=\frac{n-1}{n}$.
		
		\item Suma czwartego i siódmego wyrazu ciągu arytmetycznego jest równa 31,a suma piątego i ósmego wyrazu jest równa 37. Wyznacz wzór ogólny tego ciągu.
		
		\item Suma czwartego i piątego wyrazu ciągu arytmetycznego jest równa 17,a różnica piątego i ósmego wyrazu jest równa -9. Wyznacz wzór ogólny tego ciągu.
		
		\item Dla jakiego "x" podany ciąg jest ciągiem arytmetycznym?
		$$4x^2 - 1,\quad 6x + 1,\quad x^2 + 7$$
		
		\item Dla jakiego "x" podany ciąg jest ciągiem arytmetycznym?
		$$2x+ 1,\quad x^2 - 2x + 8,\quad 4x + 3$$
		
		\item Czy podany ciąg jest ciągiem arytmetycznym?
		$$\frac{1}{\sqrt{5}-2},\quad \sqrt{5},\quad \sqrt{5}-2$$
		
		\item Oblicz sumę: $S=-5-7-9-11-13-\dots - 101$.
		
		\item Oblicz sumę pięćdziesięciu kolejnych liczb będących wielokrotnościami 12, z których
		najmniejszą jest 24.
		
		\item Czy podany ciąg jest ciągiem geometrycznym?
		$$\sqrt{5}-2,\quad \frac{1}{2},\quad \frac{\sqrt{5}-2}{4}$$
		
		\item Trzeci wyraz ciągu geometrycznego wynosi 8, a piąty 18. Wyznacz wzór ogólny tego ciągu.
		
		\item Oblicz długości boków trójkąta prostokątnego wiedząc, że tworzą one ciąg arytmetyczny
		o różnicy 2.
		
		\item Miary trzech kolejnych kątów czworokąta wpisanego w koło tworzą ciąg arytmetyczny
		o różnicy $47^\circ$. Oblicz miary kątów tego czworokąta.
		
		\item Trzy liczby a, b, 1 tworzą ciąg arytmetyczny. Liczby 1, a, b tworzą ciąg geometryczny. Znajdź liczby a i b.
		
		\item Wyznacz liczby x i y tak, aby ciąg $(-45,\: x,\: 75)$ był arytmetyczny, a ciąg $(x,\: 75,\: y)$ był geometryczny.
		
	\end{enumerate}
	
\end{document}