\documentclass[12pt,a4paper]{article}
\usepackage[utf8]{inputenc} %polskie znaki
\usepackage[T1]{fontenc}	%polskie znaki
\usepackage{amsmath}		%matematyczne znaczki :3
\usepackage{enumerate}		%Dodatkowe opcje do funkcji enumerate
\usepackage{geometry} 		%Ustawianie marginesow
\usepackage{graphicx}		%Grafika
\usepackage{wrapfig}		%Grafika obok textu
\usepackage{float}			%Allows H in fugire
\pagestyle{empty} 			%usuwa nr strony

\newgeometry{tmargin=2cm, bmargin=2cm, lmargin=2cm, rmargin=2cm} 

\begin{document}
	\begin{center}
		\LARGE Wielomiany i ułamki algebraiczne - karta pracy
	\end{center}
	\vspace{1.5cm}

	\begin{tabular}{p{13cm} r}
		Imię i nazwisko: ............................................................................
		&[....../40pkt]\\ 
		\vspace{0.5cm} 
		
	\end{tabular}
	\begin{enumerate}[1.]\large	
		\item  Rozwiąż równania
		\begin{enumerate}[a)]
			\item \begin{tabular}{p{13cm} r}
				$2x^5-40x^3+128x=0$&[5pkt]\\ 
			\end{tabular}
			\item  \begin{tabular}{p{13cm} r}
				$x^3-7x^2=4x-28$&[5pkt]\\ 
			\end{tabular}
			\item  \begin{tabular}{p{13cm} r}
				$x^3-7x^2+17x-15=0$&[5pkt]\\ 
			\end{tabular}
			\item  \begin{tabular}{p{13cm} r}
				$(x^2-2)(x+2)-(x-4)^2=5x^2+10x-36$&[5pkt]\\ 
			\end{tabular}
			\item  \begin{tabular}{p{13cm} r}
				$\frac{x^2+x-6}{x^2-2x-15}=4$&[5pkt]\\ 
			\end{tabular}
			\item  \begin{tabular}{p{13cm} r}
				$\frac{2x-1}{x-1}=\frac{4x+1}{2x+1}$&[5pkt]\\ 
			\end{tabular}
		\end{enumerate}
		
		\item Uprość wyrażenie:
		
		\begin{enumerate}[a)]
			\item  \begin{tabular}{p{13cm} r}
				$\frac{x+1}{x-3}-\frac{2x+3}{x+2}=$&[5pkt]\\ 
			\end{tabular}
			\item  \begin{tabular}{p{13cm} r}
				$\frac{x^2+7x+10}{3x+15}:\frac{x^2-4}{x+5}=$&[5pkt]\\ 
			\end{tabular}
		\end{enumerate}
		
		
		
		
		
	\end{enumerate}
	
\end{document}
