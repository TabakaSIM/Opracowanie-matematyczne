\documentclass[12pt,a4paper]{article}
\usepackage[utf8]{inputenc} %polskie znaki
\usepackage[T1]{fontenc}	%polskie znaki
\usepackage{amsmath}		%matematyczne znaczki :3
\usepackage{enumerate}		%Dodatkowe opcje do funkcji enumerate
\usepackage{geometry} 		%Ustawianie marginesow
\usepackage{graphicx}		%Grafika
\usepackage{wrapfig}		%Grafika obok textu
\usepackage{float}			%Allows H in fugire
\pagestyle{empty} 			%usuwa nr strony

\newgeometry{tmargin=2cm, bmargin=2cm, lmargin=2cm, rmargin=2cm} 

\begin{document}
	\begin{center}
		\LARGE Wielomiany i ułamki algebraiczne - test
	\end{center}
	\vspace{1.5cm}
	\begin{flushright}
		\textbf{1 TERMIN}
	\end{flushright}
	\begin{tabular}{p{13cm} r}
		Imię i nazwisko: ............................................................................
		&[....../40pkt]\\ 
		\vspace{0.5cm}
	\end{tabular}
	\begin{enumerate}[1.]	\large
		\item  Rozwiąż równania
		\begin{enumerate}[a)]
			\item \begin{tabular}{p{13cm} r}
				$-3x^5-9x^3+12x=0$ &[5pkt]\\ 
			\end{tabular}
			\item \begin{tabular}{p{13cm} r}
				$4x^3+16x=x+4$ &[5pkt]\\ 
			\end{tabular}
			\item  \begin{tabular}{p{13cm} r}
				$x^3-6x-4=0$&[5pkt]\\ 
			\end{tabular}
			\item  \begin{tabular}{p{13cm} r}
				$(x^2-3)(x-5)-3(x-5)=-2x^2+3x+3$&[5pkt]\\ 
			\end{tabular}
			\item  \begin{tabular}{p{13cm} r}
				$\frac{x^3-3x-10}{x^2-8x+15}=1$&[5pkt]\\ 
			\end{tabular}
			\item  \begin{tabular}{p{13cm} r}
				$\frac{x-4}{x-2}=\frac{x+2}{2x+1}$&[5pkt]\\ 
			\end{tabular}
		\end{enumerate}
		
		\item Oblicz i uprość wyrażenie:
		
		\begin{enumerate}[a)]
			\item  \begin{tabular}{p{13cm} r}
				$\frac{2x+1}{x-3}-\frac{x+3}{2x-1}$=	&[5pkt]\\ 
			\end{tabular}
			\item  \begin{tabular}{p{13cm} r}
				$\frac{5x-15}{x^2-2x}\cdot\frac{x^2+x-6}{x^2-9}=$&[5pkt]\\ 
			\end{tabular}
		\end{enumerate}
		
		
		
		
		
	\end{enumerate}
	
\end{document}
