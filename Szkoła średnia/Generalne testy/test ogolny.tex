\documentclass[12pt,a4paper]{article}
\usepackage[T1]{fontenc}
\usepackage[utf8x]{inputenc}
\usepackage{hyperref}
\usepackage{url}
\usepackage{amsfonts}
\usepackage{graphicx}
\usepackage[normalem]{ulem}
\usepackage{enumerate}
\usepackage{amsthm}
\usepackage[export]{adjustbox}


\addtolength{\hoffset}{-1.5cm}
\addtolength{\marginparwidth}{-1.5cm}
\addtolength{\textwidth}{3cm}
\addtolength{\voffset}{-1cm}
\addtolength{\textheight}{2.5cm}
\setlength{\topmargin}{0cm}
\setlength{\headheight}{0cm}
\newtheorem{zad}{Zadanie}
\newtheorem{prz}{Przykład}
\newtheorem{tw}{Twierdzenie}
\newtheorem{wl}{Własność}
\newtheorem{wn}{Wniosek}
\newtheorem{de}{Definicja}
\newtheorem{lemma}{Lemat}

\begin{document}
	\begin{zad}
		Oblicz pole całkowite i objętość graniastosłupa czworokątnego, którego podstawą jest trapez o podstawach długości 6cm i 18cm oraz ramię tego trapezu ma długość 12cm, wiedząc, że  kąt między przekątną ściay bocznej a ramieniem trapezu w podstawie tworzy kąt 30$^\circ$.
	\end{zad}
\begin{zad}
	Oblicz:
\end{zad}
\begin{enumerate}[a)]
	\item $\frac{(6^3)^2}{2^4\cdot3^4}=$
	\item $\frac{125\cdot15}{3^3\cdot5^4}$=
	\item $\sqrt{50}+\sqrt{2}=$
	\item $\sqrt{300}-2\sqrt{3}=$
\end{enumerate}
\begin{zad}
	Siedziały wróble na strachu na wróble. Początkowo na lewym ramieniu siedziało dwa razy więcej wróbli niż na prawym. Potem sześć wróbli przeniosło się z lewego ramienia na prawe i wówczas po oby stronach było tyle samo wróbli. Ile wróbli siedziało na strachu na wróble?
\end{zad}
\begin{zad}
	W konkursie matematycznym było 20 zadań, za dobrą odpowiedź dostawało się 5 punktów, a za złą lub brak odpowiedzi traciło się 2 punkty. Romek zdobył w tym konkursie 72 punkty, na ile zadań odpowiedział dobrze?
\end{zad}
\begin{zad}
	Biznesmen zapłacił za obiad 300 zł wraz z 20\% napiwkiem. Jaka była cena obiadu bez napiwku ?
\end{zad}
\begin{zad}
	Prędkość rozchodzenia się impulsu elektrycznego u człowieka wynosi około 2 metrów na
	sekundę. U roślin impuls elektryczny może rozchodzić się z prędkością około 60 centymetrów
	na minutę.
\end{zad}
\textbf{Ile razy prędkość rozchodzenia się impulsu elektrycznego u człowieka jest większa od
	prędkości rozchodzenia się impulsu elektrycznego u roślin?}
\begin{zad}
	Z miasta A wyjechał pociąg jadący do miasta B o godzinie 9:30,
	a z miasta B wyjechał pociąg do miasta A o 10:00. Pierwszy pociąg
	jechał z prędkością 80km/h, a drugi z prędkością 120km/h. Miasto A i B są
	oddalone od siebie o 160km. Oblicz:
	\begin{enumerate}[a)]
		\item W jakiej odległości od miasta A mineły się pociągi?
		\item Który pociąg dojechał szybciej na stację końcową?
	\end{enumerate}
\end{zad}
\end{document}