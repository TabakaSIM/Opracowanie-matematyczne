\documentclass[12pt,a4paper]{article}
\usepackage[utf8]{inputenc} %polskie znaki
\usepackage[T1]{fontenc}	%polskie znaki
\usepackage{amsmath}		%matematyczne znaczki :3
\usepackage{enumerate}		%Dodatkowe opcje do funkcji enumerate
\usepackage{geometry} 		%Ustawianie marginesow
\usepackage{graphicx}		%Grafika
\usepackage{wrapfig}		%Grafika obok textu
\usepackage{float}			%Allows H in fugire
\pagestyle{empty} 			%usuwa nr strony

\newgeometry{tmargin=2cm, bmargin=2cm, lmargin=2cm, rmargin=2cm} 

\begin{document}
	\begin{center}
		\LARGE Geometria analityczna - powtórka przed sprawdzianem
	\end{center}
	\vspace{1.5cm}
	\begin{tabular}{p{13cm} r}
		Imię i nazwisko: ............................................................................
		&[....../30pkt]\\ 
		\vspace{0.5cm}
	\end{tabular}
	\begin{enumerate}[1.]
		\item  \begin{tabular}{p{13cm} r}
		Oblicz brakujące elementy: &[3pkt]\\ 
		\end{tabular}
	
	\begin{tabular}{|p{3cm} |p{3cm} |p{3cm} |p{3cm}|}
		\hline
		$A$ & $B$ & $\overrightarrow{AB}$ & $\overrightarrow{BA}$\\ 
		\hline
		$A=(5,-1)$ & $B=(2,5)$ & $\overrightarrow{AB}=[\dots,\dots]$ & $\overrightarrow{BA}=[\dots,\dots]$\\ 
		\hline
		$A=(6,-1)$ & $B=(\dots,\dots)$ & $\overrightarrow{AB}=[3,4]$ & $\overrightarrow{BA}=[\dots,\dots]$\\ 
		\hline
		$A=(1,-3)$ & $B=(\dots,\dots)$ & $\overrightarrow{AB}=[\dots,\dots]$ & $\overrightarrow{BA}=[8,-5]$\\ 
		\hline
	\end{tabular}

	\item  \begin{tabular}{p{13cm} r}
		Oblicz obwód trójkąta o wierzchołkach: \newline $A=(-2,-1)$, $B=(4,-1)$, $C=(1,5)$. &[3pkt]\\ 
	\end{tabular}

	\item \begin{tabular}{p{13cm} r}
		Wyznacz równanie prostej $AB$, dla $A=(-3,-3)$, $B=(6,0)$. &[4pkt]\\ 
	\end{tabular}

	\item \begin{tabular}{p{13cm} r}
		Wyznacz prostą równoległą do prostej $k: \: y=\frac{2}{3}x+\sqrt{3}$ przechodzącą przez punkt $P=(4,1)$.&[4pkt]\\ 
	\end{tabular}

	\item \begin{tabular}{p{13cm} r}
		Zapisz równanie okręgu o środku $S=(2,-5)$ i promieniu 6.&[2pkt]\\ 
	\end{tabular}
\vspace{0.5cm}

	.................................................................................................
	
	\item \begin{tabular}{p{13cm} r}
		Dany jest równoległobok $ABCD$, gdzie $A=(-1,3)$, $B=(-4,-2)$ oraz punkt $S=(2,2)$ który jest środkiem symetrii tego równoległoboku. Wyznacz punkty $C$ i $D$. &[4pkt]\\ 
	\end{tabular}
	\item \begin{tabular}{p{13cm} r}
		Wyznacz pole trójkąta o wierzchołkach \newline $A=(\frac{1}{2} ,4)$, $B=(3\frac{1}{2},\frac{1}{2})$, $C=(1\frac{1}{2},-3\frac{1}{2})$. &[5pkt]\\ 
	\end{tabular}
	\item \begin{tabular}{p{13cm} r}
		Rozwiąż układ równań: &[5pkt]\\ 
	\end{tabular}
		$$\left\{\begin{array}{l}
			x^2+y^2+2x+2y-5=0\\
			x+y=-2
		\end{array}\right.$$
	\end{enumerate}

	\newpage
	\begin{enumerate}[1.]

	\item Oblicz odległość prostej $k$ od punktu $P$:
	
	\begin{enumerate}[a)] 
		\item $k: \; x+y+3=0 \qquad \quad P=(-2,-5)$
		\item $k: \; 3x-4y+6=0 \qquad P=(4,2)$
		\item $k: \; y=2x-7 \:\qquad \qquad P=(-1,1)$
	 \end{enumerate}

		\item Oblicz odległość pomiędzy dwoma prostymi:
		
	\begin{enumerate}[a)] 
		\item $k: \; 2x-y-1=0 \qquad l:\; 2x-y+3=0\; $
		\item $k: \; x+6=0 \qquad \qquad\ l:\; 5x-10=0$
		\item $k: \; x+y+2=0 \qquad\ \;  l:\; x+y-6=0$
	\end{enumerate}

	\item Dany jest trójkąt o wierchołkach $A=(-2,3)$, $B=(-2,2)$, $C=(2,0)$. Oblicz jego pole.
	
	\item Dany jest trójkąt o wierchołkach $A=(-3,-2)$, $B=(6,1)$, $C=(-2,6)$. Oblicz jego pole.
	
	\item Boki trójkąta zwierają się w prostych:
	$$k:\; 3x-y-9=0, \qquad l:\; 2x+y-1=0, \qquad m:\; x+y-3=0$$	wyznacz pole tego trójkąta.
	\end{enumerate}

\end{document}