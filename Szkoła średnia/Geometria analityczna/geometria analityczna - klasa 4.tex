\documentclass[12pt,a4paper]{article}
\usepackage[utf8]{inputenc} %polskie znaki
\usepackage[T1]{fontenc}	%polskie znaki
\usepackage{amsmath}		%matematyczne znaczki :3
\usepackage{enumerate}		%Dodatkowe opcje do funkcji enumerate
\usepackage{geometry} 		%Ustawianie marginesow
\usepackage{graphicx}		%Grafika
\usepackage{wrapfig}		%Grafika obok textu
\usepackage{float}			%Allows H in fugire
\pagestyle{empty} 			%usuwa nr strony

\newgeometry{tmargin=2cm, bmargin=2cm, lmargin=2cm, rmargin=2cm} 

\begin{document}
	\begin{center}
		\LARGE Geometria analityczna
	\end{center}
	\vspace{1.5cm}
	
	\begin{enumerate}[1.]
		\item Dla podanej pary punktów $A$ i $B$ wyznacz wektor $\overrightarrow{AB}$
		
		\begin{enumerate}[a)] \begin{tabular}{p{7cm} p{7cm}}
				\item $A=(-5,3)$, $B=(7,2)$& \item $A=(3,7)$, $B=(8,12)$ \\
				\item $A=(-4,0)$, $B=(1,-6)$& \item $A=(-12,6)$, $B=(8,-6)$ \\
				\item $A=(-11,5)$, $B=(22,1)$& \item $A=(-9,6)$, $B=(-4,-11)$ \\
	 	\end{tabular} \end{enumerate}
 	
 	\item Wyznacz punkt $B$ wiedząc, że
 	
 		\begin{enumerate}[a)] \begin{tabular}{p{7cm} p{7cm}}
 			\item $A=(-1,2)$, $\overrightarrow{AB}=[ 3,6 ]$ & \item $A=(0,-2)$, $\overrightarrow{AB}=[3,5]$ \\
 			\item $A=(-5,5)$, $\overrightarrow{AB}=[ 3,4 ]$ & \item $A=(-2,1)$, $\overrightarrow{AB}=[-1,-9]$ \\
 		\end{tabular} \end{enumerate}
 	
 	 	\item Wyznacz punkt $B$ wiedząc, że
 	
 	\begin{enumerate}[a)] \begin{tabular}{p{7cm} p{7cm}}
 			\item $A=(8,9)$, $\overrightarrow{BA}=[ -3,-6 ]$ & \item $A=(-1,3)$, $\overrightarrow{BA}=[-2,-4]$ \\
 			\item $A=(5,-3)$, $\overrightarrow{BA}=[ -6,1 ]$ & \item $A=(8,-7)$, $\overrightarrow{BA}=[2,-8]$ \\
 	\end{tabular} \end{enumerate}
 
 	\item Podziel odcinek $AB$ na trzy równe części, jeśli
 	
 			\begin{enumerate}[a)] \begin{tabular}{p{7cm} p{7cm}}
 			\item $A=(-5,0)$, $B=(4,0)$& \item $A=(1,-5)$, $B=(7,7)$ \\
 			\item $A=(-4,8)$, $B=(2,-1)$& \item $A=(-5,1)$, $B=(3,3)$ \\
 	\end{tabular} \end{enumerate}
 
 	\item Wyznacz obwód trójkąta $ABC$
 	
 	 \begin{enumerate}[a)] \begin{tabular}{p{7cm} p{7.5cm}}
 			\item $A=(1,1)$, $B=(8,2)$, $C=(4,5)$& \item $A=(2,-3)$, $B=(8,-3)$, $C=(11,-8)$ \\
 			\item $A=(0,-1)$, $B=(5,1)$, $C=(2,4)$& \item $A=(-2,-2)$, $B=(3,-1)$, $C=(1,3)$ \\
 	\end{tabular} \end{enumerate}
 
  	\item Wyznacz równanie odcinka $AB$
 
 	\begin{enumerate}[a)] \begin{tabular}{p{7cm} p{7cm}}
 		\item $A=(-4,0)$, $B=(4,0)$& \item $A=(1,-5)$, $B=(7,7)$ \\
 		\item $A=(-4,8)$, $B=(2,-1)$& \item $A=(-5,1)$, $B=(3,3)$ \\
 	\end{tabular} \end{enumerate}
 
 	\newpage
 	\item Dla podanych prostych wyznacz ich współczynnik kierunkowy oraz określ jej monotoniczność
 	
 	 \begin{enumerate}[a)] \begin{tabular}{p{7cm} p{7cm}}
 			\item $y=2x+5$& \item $y=-3x+8$ \\
 			\item $y=\sqrt{2}x-3\sqrt{5}$& \item $y=\frac{1}{2}x+1\frac{3}{4}$ \\
 			\item $y=(3\sqrt{2}-2\sqrt{3})x+5$& \item $y=3x+6-\sqrt{10}x$ \\
 	\end{tabular} \end{enumerate}
 	
 	\item Dla podanych prostych, podaj przykład prostej równoległej do tej prostej
 	
 	\begin{enumerate}[a)] \begin{tabular}{p{7cm} p{7cm}}
 			\item $y=-2x+1$& \item $y=x+12$ \\
 			\item $y=\frac{2}{3}x+2$& \item $y=-\frac{1}{3}x+8$ \\
 	\end{tabular} \end{enumerate}
 	
 	\item Dla podanych, prostych podaj przykład prostej prostopadłej do tej prostej
 	
 	\begin{enumerate}[a)] \begin{tabular}{p{7cm} p{7cm}}
		\item $y=-2x+1$& \item $y=x+12$ \\
		\item $y=\frac{2}{3}x+2$& \item $y=-\frac{1}{3}x+8$ \\
	\end{tabular} \end{enumerate}

 	
 	\item Podaj równanie prostej równoległej do prostej $k$ przechodzącą przez punkt $P$
 	
 	\begin{enumerate}[a)] \begin{tabular}{p{7cm} p{7cm}}
 			\item $k:y=3x+1$, $P=(2,9)$& \item $k:y=\frac{1}{2}x+4$, $P=(-4,5)$ \\
 			\item $k:y=-x-5$, $P=(-3,4)$& \item $k:y=\frac{3}{5}x-3\sqrt{2}$, $P=(10,10)$ \\
 	\end{tabular} \end{enumerate}
 	
 	\item Podaj równanie prostej prostopadłej do prostej $k$ przechodzącą przez punkt $P$
 	
 	\begin{enumerate}[a)] \begin{tabular}{p{7cm} p{7cm}}
 			\item $k:y=-\frac{1}{2}x+4$, $P=(1,-3)$& \item $k:y=\frac{1}{2}x+4$, $P=(-2,-1)$ \\
 			\item $k:y=x$, $P=(-7,4)$& \item $k:y=4x+\pi$, $P=(-4,3)$ \\
 	\end{tabular} \end{enumerate}
 	
 	\item Wyznacz środek odcinka $AB$
 	
 	\begin{enumerate}[a)] \begin{tabular}{p{7cm} p{7cm}}
 			\item $A=(1,3)$, $B=(7,3)$& \item $A=(7,3)$, $B=(8,5)$ \\
 			\item $A=(-3,0)$, $B=(1,-6)$& \item $A=(-5,6)$, $B=(-9,-6)$ \\
 			\item $A=(12,-5)$, $B=(22,11)$& \item $A=(-4,5)$, $B=(-5,4)$ \\
 	\end{tabular} \end{enumerate}
 	
 	\item Wyznacz równanie symetralnej odcinka $AB$
 	
 	 \begin{enumerate}[a)] \begin{tabular}{p{7cm} p{7cm}}
 			\item $A=(0,0)$, $B=(2,4)$& \item $A=(-4,-3)$, $B=(2,3)$ \\
 			\item $A=(3,1)$, $B=(-1,1)$& \item $A=(-5,4)$, $B=(5,5)$ \\
 	\end{tabular} \end{enumerate}
 	\newpage
 	
 	\item Wyznacz środek i promień okręgu
 	
 	\begin{enumerate}[a)] \begin{tabular}{p{7cm} p{7cm}}
	\item $(x-3)^2 + (y+2)^2=25$& \item $(x+3)^2 + (y+5)^2=9$ \\
	\item $(x-1)^2 + (y+4)^2=8$& \item $(x-7)^2 + (y-4)^2=96$ \\
	\item $x^2 + y^2=1$& \item $(x-\sqrt{2})^2 + (y+4\sqrt{2})^2=\sqrt{2}$ \\
	\end{tabular} \end{enumerate}
 	
 	\item Zapisz równanie okręgu w postaci ogólnej
 	
 	\begin{enumerate}[a)] \begin{tabular}{p{7cm} p{7cm}}
 	\item $(x+1)^2 + (y-1)^2=1$& \item $(x+2)^2 + (y-3)^2=4$ \\
 	\item $(x-4)^2 + (y+3)^2=10$& \item $(x-3)^2 + (y-3)^2=18$ \\
 	\end{tabular} \end{enumerate}
 	
 	\item Zapisz równanie okręgu w postaci kanonicznej oraz wyznacz środek i promień okręgu
 	
 	 	\begin{enumerate}[a)] \begin{tabular}{p{7cm} p{7cm}}
 			\item $x^2+y^2+6x-4y-12=0$& \item $x^2+y^2-6y+5=0$ \\
 			\item $x^2+y^2-2x+4y-5=0$& \item $x^2+y^2+8x-12y+25=0$ \\
 	\end{tabular} \end{enumerate}
 	
 	\item Napisz równanie okręgu o środku w punkcie $S=(-4,3)$:
 	
 	\begin{enumerate}[a)] \begin{tabular}{p{7cm} p{7cm}}
 			\item stycznego do osi OX& \item przechodzącego przez punkt $(0,0)$ \\
 			\item stycznego do osi OY& \item przechodzącego przez punkt $(2,-5)$ \\
 	\end{tabular} \end{enumerate}
 
 	\item Rozwiąż układ równań drugiego stopnia:
 	
 	\begin{enumerate}[a)]
 				\item $\left\{\begin{array}{l}
 			x^2-4x+y^2-4y=0\\
 			x+y-8=0
 		\end{array}\right.$
 			\item $\left\{\begin{array}{l}
 		x^2+y^2+10x-8y+25=0\\
 		x-y+3=0
 	\end{array}\right.$
 		\item $\left\{\begin{array}{l}
 	x^2+y^2+6x-4y-13=0\\
 	x+y-5=0
 \end{array}\right.$
 	\end{enumerate} 
 
	\end{enumerate}
\end{document}