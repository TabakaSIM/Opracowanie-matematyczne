\documentclass[12pt,a4paper]{article}
\usepackage[T1]{fontenc}
\usepackage[utf8x]{inputenc}
\usepackage{hyperref}
\usepackage{url}
\usepackage{amsfonts}
\usepackage{graphicx}
\usepackage[normalem]{ulem}
\usepackage{enumerate}
\usepackage{amsthm}
\usepackage[export]{adjustbox}


\addtolength{\hoffset}{-1.5cm}
\addtolength{\marginparwidth}{-1.5cm}
\addtolength{\textwidth}{3cm}
\addtolength{\voffset}{-1cm}
\addtolength{\textheight}{2.5cm}
\setlength{\topmargin}{0cm}
\setlength{\headheight}{0cm}
\newtheorem{zad}{Zadanie}
\newtheorem{prz}{Przykład}
\newtheorem{tw}{Twierdzenie}
\newtheorem{wl}{Własność}
\newtheorem{wn}{Wniosek}
\newtheorem{de}{Definicja}
\newtheorem{lemma}{Lemat}



\begin{document}
	\LARGE \begin{center}
		Zestaw 1
	\end{center}
	\normalsize 
	\begin{enumerate}[1.]
		\item Grządzielka ma trzy razy więcej pętelek niż kopałka.Gdyby kompałka miała sześć pętelek więcej, to miała by ich tyle samo co grządzielka. Ile pętelek mają grządzielka z kopałką?
		\item Koszykarze podczas pewnego meczy wykonali łącznie 37 skutecznych rzutów do kosza za 2 i za 3 punkty, zdobywając łącznie 82 punkty. Ile było rzutów do kosza za 2 punty, a ile za 3?
		\item Filiżanka jest o 8 zł droższa od podstawki. Za podstawkę z filiżanką trzeba zapłacić 38zł. Ile kosztuje filiżanka bez podstawki?
		\item W hotelu są 74 miejsca noclegowe. Pokoi dwuosobowych jest o 10 więcej niż jednoosobowych, pokoi trzyosobowych jest 3 razy mniej niż dwuosobowych i na ostatnim piętrze są 3 pokoje czteroosobowe. Ile pokoi jest w tym hotelu?
		\item Uczniowie klasy VIIb przygotowali na loterię 120 losów, w tym tylko 15 wygrywających. Ile powinni dołożyć 
losów wygrywających, aby loteria spełniała warunek, że co czwarty los wygrywa?
		\item W trapezie o polu $24cm^2$ wysokość jest równa 6 cm, a jedna z podstaw jest o 2 cm dłuższa od drugiej. Oblicz długości podstaw tego trapezu.
		\item Ania i Bogdan ważą razem 70 kg. Gdyby Ania przytyła 4 kg, a Bogdan 3 kg schudł, to oboje ważyliby tyle samo. Ile waży Ania, a ile Bogdan?
		\item Gdy zmieszamy błękit paryski z czerwienią cynobrową w ten sposób, że farby niebieskiej będzie 1,5 raza 
więcej niż czerwonej, to otrzymamy farbę koloru ultramaryna. Ile należy wziąć kilogramów błękitu, a ile czerwieni, 
aby otrzymać 1,5 kg ultramaryny?
		\item Dwaj sąsiedzi, pan Jan i pan Kazimierz, hodują gołębie. Mają ich razem 47. Pan Jan ma o $12\%$ gołębi mniej 
niż pan Kazimierz. Ile gołębi ma każdy z nich?
		\item Aby uzyskać eliksir spokoju, należy zmieszać ze sobą jedną miarkę naparu z melisy, dwie marki lubczyku, 
ćwierć miarki kropli walerianowych i 3 miarki sekretnego składnika, którego nazwy nie można zdradzić. Ile litrów 
powinna mieć taka miarka, aby można było poczęstować 500 osób, tak aby każdy dostał 50ml eliksiru.
	\end{enumerate}
\newpage
	\LARGE \begin{center}
	Zestaw 2
\end{center}
\normalsize 
\begin{enumerate}[1.]
	\item W przepełnionym wagonie klasy II było cztery razy więcej pasażerów niż w wagonie klasy I. Gdyby 54 osoby przeniosły się z klasy II do klasy I, to w obu wagonach byłoby tyle samo pasażerów. Ile pasażerów jechało w tych wagonach?
	\item Pan Kowalski sprzedawał czereśnie po tej samej cenie jak pan Nowak. Pod wieczór obniżyli ceny - Kowalski o $20\%$, a Nowak o 4,50zł. Okazało się, że cena czereśni u Kowalskiego było o $10\%$ wyższa niż u Nowaka. Ile kosztowały czereśnie przed obniżką?
	\item W trójkącie prostokątnym jeden z kątów ostrych ma miarę 2 razy mniejszą niż suma dwóch pozostałych. Oblicz miary kątów tego trójkąta.
	\item W stawie pana Ludwika pływające grube ryby stanowią $5\%$ wszystkich ryb, oprócz tego są płotki, które stanowią $80\%$ pozostałych ryb, oraz leszcze, których jest ich 38. Ile grubych rybpływa w tym stawie?
	\item Turysta przeszedł 72 km w ciągu trzech dni. Pierwszego dnia przeszedł najwięcej. Drugiego dnia dwa razy mniej niż pierwszego, a trzeciego dnia połowę tego, co pierwszego i drugiego w sumie. Ile kilometrów turysta przeszedł każdego dnia?
	\item Dwie cegły ważą o 1 kg więcej niż 1,5 cegły. Ile waży cegła?
	\item Przewiduje się, że wycieczka szkolna będzie kosztować 270 zł dziennie. Gdyby
	udało się ten koszt obniżyć o 54 zł, to za tę samą kwotę można by zorganizować wycieczkę o 3 dni dłuższą. Ile dni miała trwać wycieczka?
	\item Dwie grupy szkolne wybrały się do muzeum. Pierwsza grupa kupując 2 bilety normalne i 15 ulgowych zapłaciła 66 złotych. Druga grupa zapłaciła za 3 bilety normalne i 20 ulgowych 90 złotych. W jakiej cenie są bilety do tego muzeum? Ile procent ceny biletu normalnego wynosi zniżka?
	\item Za trzy pary nożyczek i dwa zszywacze zapłacono razem 68 zł. Ile kosztuje jedna para nożyczek, jeśli zszywacz kosztuje 16 zł?
	\item Na podłodze leżała listewka długości 106 cm. Niegrzeczne dziecko połamało tę listewkę na	dwie części, z których jedna jest krótsza od drugiej o 12 cm. Jaka jest długość krótszej części listewki?
\end{enumerate}
\newpage
\LARGE \begin{center}
	Zestaw 3
\end{center}
\normalsize 
\begin{enumerate}[1.]
	\item W pewnej szkole pracuje 95 nauczycieli, przy czym kobiet jest 4 razy więcej niż mężczyzn. Ile nauczycielek pracuje w tej szkole?
	\item Trapez o wysokości 14 cm ma pole równe $260cm^2$. Oblicz długość obu podstaw trapezu, jeżeli górna do dolnej podstawy jest w skali jeden do trzech.
	\item Bodgan wziął na obóz w góry 80zł. Na napoje wydał 2 razy więcej, niż na lody, a o 5 zł mniej niż na napoje wydał na słodycze. Za 15 zł kupił upominki dla rodziców. Wrócił bez pieniędzy, gdyż ostatniego dnia zgubił 10zł. Ile pieniędzy wydał na lody?
	\item W trzydziestoosobowej klasie $30\%$ stanowiły dziewczęta. W II półroczu odeszło kolka z nich i teraz stanowią $16\%$ klasy. Ile dziewcząt odeszło z klasy?
	\item W pewnym trójkącie równoramiennym kąt przy podstawie jest o $30^\circ$ większy od kąta między ramionami. Jakie miary mają kąty tego trójkąta?
	\item Na początku bardzo ważnego zebrania było o 5 kobiet więcej niż mężczyzn. Po piętnastu minutach przyszli jeszcze spóźnieni pan Józio i trzy panie z księgowości. Wtedy znudzony pan Tadeusz zauważył, że kobiet jest dwa razy więcej niż mężczyzn. Ile osób było wówczas na sali?
	\item Na bardzo nudnym wykładzie połowa studentów drzemie, a jedna trzecia studentów rozwiązuje krzyżówki. Wśród pozostałych sześciu słuchaczy pięć czyta książkę i tylko jedna studentka pilnie notuje. Ilu studentów jest obecnych na tym wykładzie?
	\item Krzyś rozbił skarbonkę i policzył, że było w niej 32 zł w monetach o nominałach 2zł, 1zł 50gr i 20gr. Dwuzłotówek było 2 razy mniej niż złotówek, a pięćdziesięciogroszówek było 2 razy mniej niż złotówek. Dwudziestogroszówek było 10. Ile monet było w skarbonce?
	\item Czwartą część wszystkich pasażerów autobusu stanowiły kobiety. Na przystanku wysiadła jedna kobieta i wsiadło dwóch mężczyzn. Kiedy autobus ruszał z tego przystanku, kobiety stanowiły jedną piątą pasażerów. Ilu pasażerów było w autobusie?
	\item Asia trzyma w skarbonce 245 zł oszczędności. Ma łącznie 100 monet, z czego stosunek pięciozłotówek do pozostałych (dwu i jednozłotowych) jest jeden do trzech. Ile monet o nominale 2 zł ma Asia?

\end{enumerate}
\newpage
\LARGE \begin{center}
	Zestaw 4
\end{center}
\normalsize 
\begin{enumerate}[1.]
	\item W trójkącie jeden z kątów ma miarę 3 razy większą od drugiego i o $12^\circ$ mniejszą od trzeciego. Oblicz miary kątów tego trójkąta.
	\item Na zgaduj zgaduli postawiono 30 pytań. Za każda poprawną odpowiedź zaliczano
	7 punktów, zaś za każda nieprawidłową uczestnik tracił 12 punktów. Ile dobrych odpowiedzi dał jeden uczestników, jeżeli przy podsumowaniu okazało się, że zdobył 77 punktów?
	\item Pręt zbrojeniowy o długości 28 m zgięto pod kątem prostym w taki sposób, że stosunek długości jednej części do drugiej jest równy 3 : 4. Oblicz odległość między końcami tego pręta po zgięciu.
	\item Drut długości 44cm przecięto na dwie części. Z każdej z nich wykonano ramkę - jedną kwadratową, drugą prostokątną. Jeden bok prostokąta jest równy połowie boku kwadratu, a drugi bok prostokąta jest o 1 cm dłuższy od boku kwadratu. Oblicz sumę pól powierzchni ograniczonymi tymi rankami.
	\item W firmie budowlanej pracuje o 6 mężczyzn więcej niż kobiet. Na każde 4 kobiety przypada 5 mężczyzn. Ilu pracowników liczy ta firma?
	\item W trapezie o polu $18cm^2$ wysokość jest równa 3cm, a jedna z podstaw jest o 5 cm krótsza od drugiej podstawy. Oblicz długości podstaw tego trapezu.
	\item Za 3 lata Grześ będzie 3 razy starszy niż 3 lata temu. Ile lat ma Grześ?
	\item Książki Ali stoją na trzech półkach. Na drugiej półce jest o 15 książek mniej niż na pierwszej i 2 razy więcej niż na drugiej. Ile książek ma Ala?
	\item Szymon zdobył 54 punkty ze sprawdzianu składającego się z piętnastu zadań. Za każdą poprawną odpowiedź otrzymywał 5 punktów, a za każdą złą lub brak odpowiedzi tracił 2 punkty. Ile zadań rozwiązał prawidłowo?
	\item W wielkiej loterii promocyjnej przygotowano 900 losów. Okazało się, że tylko $5\%$ tych losów wygrywało. Ile jeszcze losów wygrywających trzeba przygotować, aby spełniona była zasada, że wygrywa $10\%$ losów?
\end{enumerate}
\newpage
\LARGE \begin{center}
	Zestaw 5
\end{center}
\normalsize 
\begin{enumerate}[1.]
	\item Chart ujrzał zająca w odległości 150 stóp i ruszył w pogoń. Skok zająca ma 7 stóp, a skok charta wykonany w tym samym czasie 9 stóp. Po ilu skokach chart dogonił zająca?
	\item Jeden z kątów przyległych jest o $70^\circ$ większy od drugiego. Jakie miary mają te kąty?
	\item Kostka sześcienna o krawędzi 10cm wykonana z mosiądzu waży 8 kg. Ile będzie ważyć mosiężna kostka o krawędzi 20 cm?
	\item W dzbanku mieści się 1,8l wody, a w kubku 0,25l. Napełnienie kubka wodą z kranu trwa 5 sekund. Ile zajmie napełnienie dzbanka wodą?
	\item O godzinie 9:00 z miejscowości Mini do odległej o 175 km miejscowości Maxi wyjechał samochodem pan Janusz a w tym samym czasie z miejscowości Maxi do Mini pani Grażyna. Oboje jechali ze stałą prędkością, pani Grażyna jechała o $10\frac{km}{h}$ szybciej niż pan Janusz. Minęli się o 10:15. O której godzinie dojechała pani Grażyna, a o której pan Janusz?
	\item W klasie liczącej 24 osoby, ze sprawdzianu z matematyki uczniowie otrzymali dwa razy więcej ocen bardzo dobrych niż dobrych i ocen dobrych o 6 więcej niż dostatecznych. Dwie osoby dostały oceny celujące. Oblicz, ilu uczniów otrzymało poszczególną ocenę, jeżeli wiadomo, że nikt nie otrzymał oceny dopuszczającej ani niedostatecznej.
	\item Suma cyfr pewnej dwucyfrowej liczby wynosi 10. Jeśli zamienimy cyfry miejscami, to otrzymamy liczbę o 36 mniejszą. Znajdź tę liczbę.
	\item Basen można napełnić woda dwiema rurami w ciągu 6 godzin. Napełnianie
	basenu pierwszą rurą trwa o 5 godzin krócej, niż drugą. Ile godzin trwa napełnianie basenu każdą rurą oddzielnie?
	\item W konkursie matematycznym było do rozwiązania 20 zadań. Franek zdobył w tym konkursie 56 punktów. Za każdą dobrą odpowiedź otrzymywał 3 punkty, a za złą lub brak tracił 1 punkt. Ile zadań Franek rozwiązał poprawnie?
	\item Za cztery ekierki i linijkę uczeń zapłacił 6,40 zł. Linijka jest o 60 groszy tańsza od ekierki. Ile kosztuje linijka?
\end{enumerate}
\newpage
\LARGE \begin{center}
	Zestaw 6
\end{center}
\normalsize 
\begin{enumerate}[1.]
	\item Jarek jest o 30 lat młodszy od swojego taty. Za 5 lat będzie od niego 3 razy młodszy. Ile lat ma Jarek, a ile jego tata?
	\item Siedziały wróble na strachu na wróble. Początkowo na lewym ramieniu siedziało dwa razy więcej wróbli niż na prawym. Potem cztery wróble przeniosły się z lewego ramienia na prawe i wówczas po oby stronach było tyle samo wróbli. Ile wróbli siedziało na strachu na wróble.
	\item W trójkącie prostokątnym jeden z kątów jest czterokrotnie mniejszy od sumy dwóch pozostałych. Podaj miary kątów ostrych tego trójkąta.
	\item Jacek obliczył, że od poniedziałku do piątki, chodząc do szkoły i z powrotem pokonał odległość 16,4 km. Jaką długość ma jego droga do szkoły?
	\item Dwie trzecie wszystkich pracowników firmy handlowej JANUSZ stanowią sprzedawcy, a jedną piątą urzędnicy. Oprócz tego firma ta zatrudnia sprzątaczkę, dyrektora i dwóch zastępców dyrektora. Ile osób pracuje w JANUSZU?
	\item Magda jest o 6 kg lżejsza od Zuzanny. Razem ważą 96 kg. Ile waży Zuzanna?
	\item Pani Anna ważyła o 3kg mniej niż pani Zofia. Po kuracji odchudzającej pani Anna waży o $15\%$ mniej, a pani Zofia o 15kg mniej i obie ważą tyle samo. Ile kilogramów ważyła każda z tych pań przed kuracją odchudzającą?
	\item Ania kupiła mleko za 2,50zł, twaróg za 3,40zł oraz 10 bułek. Za zakupy zapłaciła 8,40. Ile kosztowała bułka?
	\item Do budowy drogi używane są dwie ciężarówki o różnych ładownościach. Jedną z nich można
	wywieźć 3 razy więcej piachu niż drugą. Aby przygotować teren pod budowę drogi
	oszacowano, że trzeba wywieźć 1200 ton piachu. O jakich ładownościach są te ciężarówki,
	jeśli mniejsza z nich wykonała 30 kursów, a większa z nich o 10 kursów więcej?
	\item Radek jest o 2 lata starszy od Magdy. Sześć lat temu razem mieli 46 lat. Ile lat ma obecnie Magda, a ile Radek?
\end{enumerate}
\newpage
\LARGE \begin{center}
	Zestaw 7
\end{center}
\normalsize 
\begin{enumerate}[1.]
	\item Grupa turystów przebywała 6 godzin na wycieczce pieszej. Przewodnik stwierdził, że 8 razy dłużej wędrowali, niż odpoczywali. Jak długo trwał odpoczynek?
	\item Agata jest o 2 lata młodsza od Darii, a Justyna o 5 lat starsza od Darii. Razem dziewczynki mają 42 lata. Ile lat ma najmłodsza, a ile najstarsza z nich?
	\item Kasie i Basia są bliźniaczkami. Kiedy się urodziły, ich mama miała 28 lat, a ich tata miał 30 lat. Obecnie mają wszyscy razem 126 lat. Ile lat mają teraz bliźniaczki?
	\item Agnieszka i Jacek zbierali kasztany. Jacek uzbierał 3 razy więcej niż Agnieszka. Gdyby oddał Agnieszce 15 z nich, to oboje mieliby tyle samo kasztanów. Ile kasztanów ma Agnieszka, a ile Jacek?
	\item Dorota jest cztery razy starsza od Maćka. Za dwa lata będzie od niego trzy razy starsza. Ile lat ma Maciek, a ile Dorota?
	\item W konkursie matematycznym było 18 zadań, w tym 10 zadań za 4 punkty i 8 za 5 punktów. Za każde wskazanie błędnej odpowiedzi odejmowano $\frac{1}{4}$ punktów możliwych do zdobycia za zadanie. Marta wskazała odpowiedzi przy wszystkich pytaniach. W zadaniach za 4 punkty wszystkie jej odpowiedzi były poprawne. W sumie zdobyła 67,5 punktu. Na ile pytań odpowiedziała prawidłowo?
	\item W czasie kwesty zebrano do puszki 150 zł. Wśród 41 zebranym monet była tylko jedna złotówka, poza tym były tylko dwuzłotówki i pięciozłotówki. Ile było pięciozłotówek, a ile dwuzłotówek?
	\item Pani Kowalska zarabia miesięcznie o 200zł więcej od pana Kowalskiego. W ciągu roku oboje razem zarobili 72tyś. zł. Jakie jest miesięczne wynagrodzenie panie Kowalskiej, a jakie pana Kowalskiego?
	\item W pewnym trójkącie jeden z kątów jest dwa razy większy od drugiego i o $20^\circ$ mniejszy od trzeciego. Oblicz miary kątów tego trójkąta.
	\item Widownia w pewnym teatrze może pomieścić 178 widzów. Na parterze znajduje się 178 miejsc, natomiast na balkonie jest 5 rzędów. W pierwszym znich jest o jedno miejsce mniej niż w każdym z pozostałych. Ile jest miejsc w pierwszym rzędzie na balkonie?
\end{enumerate}
\newpage
\LARGE \begin{center}
	Zestaw 8
\end{center}
\normalsize 
\begin{enumerate}[1.]
	\item Obwód trójkąta $ABC$ wynosi 38 cm. Oblicz długości boków tego trójkąta, jeśli wiadomo, że bok $AB$ jest trzy razy dłuższy od boku $AC$, a bok $BC$ jest o 4 cm krótszy od boku $AB$.
	\item Obwód prostokąta jest równy 48 cm. Jeżeli jeden bok zwiększymy o 25\%, a drugi zmniejszymy o 6 cm, to otrzymamy kwadrat. Oblicz pole tego kwadratu.
	\item Ewa jest trzy razy starsza od Adama. Za sześć lat będzie dwa razy starsza od Adama. Ile lat ma teraz Adam?
	\item W pewnej restauracji stosunek liczby stolików dwuosobowych do liczby stolików czteroosobowych wynosi 3 : 5. Oblicz, ile jest stolików dwuosobowych, a ile czteroosobowych, jeśli wiadomo, że w restauracji przy wszystkich stolikach może usiąść 156 osób?
	\item Krzyś postanowił, że będzie czytał po 40 stron książki dziennie. Niestety, czytał tylko 30 stron dziennie i przeczytanie całek książki zajęło mu 3 dni więcej, niż planował. W ciągu ilu dni Krzyś przeczytał książkę? Ile stron liczyła owa lektura?
	\item Pierwsza cyfra pewnej liczby jest o 3 większa od drugiej. Gdyby z tych cyfr ułożyć wszystkie możliwe liczby dwucyfrowe to suma ich byłaby równa 286. Jaka to liczba?
	\item Przez prawie wszystkie miesiące tego roku pani Beata zarabiała tyle samo, a dopiero w grudniu zarobiła 600zł więcej. Dzięki temu jej średnia miesięcznych zarobków wynosiła w tym roku równo 2000zł. Ile wyniosła grudniowa pensja panie Beaty?
	\item Tomek ma 14 lat, a jego mama 38. Oblicz, kiedy mama była od Tomka pięć razy starsza. Za ile lat będzie od niego dwa razy starsza?
		\item Ela z Marcinem wybrali się na pieszą wycieczkę. Postanowili przejść trasę o długości 21 km. Po trzech godzinach marszu Marcin stwierdził, że pozostało im do przejścia o 5 km mniej, niż już przeszli. Jaką odległość pokonali w ciągu pierwszych trzech godzin wycieczki?
		\item W pierwszym kwartale (3 miesiącach) roku sprzedano 1300 samochodów pewnej marki. W lutym sprzedano o 20\% mniej samochodów tej marki niż w styczniu, w marcu o 130 więcej niż w lutym. Ile samochodów sprzedano w marcu?
\end{enumerate}
\newpage
\LARGE \begin{center}
	Zestaw 9
\end{center}
\normalsize 
\begin{enumerate}[1.]
	\item Zegar z kukułką waży 5,5 kg. Kukułka jest o 5 kg lżejsza od zegara. Ile waży zegar, a ile kukułka?
	\item Na pacu zabaw bawią się dzieci. Połowa z nich gra w piłkę, $10\%$ biega bez celu, $20\%$ gra w gumę, a reszta, czyli sześcioro bawi się w piaskownicy. Ile dzieci jest na placu zabaw?
	\item Na forum internetowym zarejestrowało się 120 tyś, osób. Twórca tego forum prześledził wpisy i podzielił uczestników tego forum na nudziarzy i fantastów, przy czym tych pierwszych jest pięć razy więcej. Ily nudziarze i fanatystów zarejestrowało się na tym forum?
	\item Tadek ma 50zł w monetach dwuzłotowych i pięciozłotowych. W sumie ma 16 monet. Ile ma monet dwuzłotowych, a ile pięciozłotowych?
	\item Trzej robotnicy, pracując po 8 godzin dziennie, wykonali w ciągu 6 dni 40\% planowanej pracy. Ilu robotników wykona resztę tej pracy w ciągu 4 dni, pracując po 9 godzin dziennie?
	\item Kasia i Małgosia oszczędzają pieniądze. Kasia ma 400 zł, a Małgosia – o 70zł mniej. Dziewczęta postanowiły co tydzień odkładać pewną kwotę: Kasia 30 zł, a Małgosia 40 zł. Po ilu tygodniach oszczędności Małgosi przewyższą oszczędności Kasi?
	\item Kapelusz z piórkiem kosztuje 110zł. Kapelusz jest droższy od piórka o 100zł. Ile kosztuje sam kapelusz, a ile samo pióro?
	\item Jeśli dwa przeciwległe boki pewnego kwadratu przedłużymy o 8cm, a dwa pozostałe skrócimy o 2cm, to obwód czworokąta zwiększy się dwukrotnie. Oblicz długość boku tego kwadratu.
	\item W wyborach na przewodniczącego klasy startowało dwóch kandydatów: Andrzej i Rafał. Głosowało na nich łącznie 168 głosów. Głosowanie wygrał Andrzej ze znaczną przewagą, bo nawet gdyby 20 osób głosujących na niego oddało głosy na Rafała, to Andrzej miałby dwa razy więcej głosów niż Rafał. Ile głosów otrzymał Andrzej, a ile Rafał?
	\item W pewnym trójkącie równoramiennym kąt między ramionami ma miarę o $33^\circ$ mniejszą niż kąt przy podstawie. Oblicz miary kątów tego trójkąta.
\end{enumerate}
\newpage
\LARGE \begin{center}
	Zestaw 10
\end{center}
\normalsize 
\begin{enumerate}[1.]
	\item Hurtownik kupił 2 tony bananów. $\frac{4}{5}$ sprzedał z zyskiem $12\%$, a resztę sprzedał z zyskiem $5\%$. Na całej transakcji zarobił 424zł. Ile zapłacił za wszystkie banany?
	\item W trapezie równoramiennym o obwodzie 32 cm jedna z podstaw jest o 2 cm krótsza od drugiej i dwa razy krótsza od ramienia. Oblicz długości boków tego trapezu.
	\item Gospodyni zebrała jajka w kurniku. Może zapakować te jajka do pojemników po 6 jajek albo po 10 jajek. W każdym wypadku zostanie jej 5 jajek. Wybierając tylko większe pojemniki, zużyje o 4 pojemniki mniej, niż gdyby użyła tylko mniejszych pojemników. Ile jaj zebrała gospodyni w kurniku?
	\item W trójkącie o obwodzie 41cm jeden z boków jest dwa razy dłuższy od najkrótszego, a inny bok jest o 13cm dłuższy od najkrótszego. Podaj długości boków tego trójkąta.
	\item Babcia chce podzielić 50 cukierków między trzech wnuków tak, by Romek dostał ich 2 razy tyle, co Marcin, a Marcin 3 razy tyle, co Wojtek. Ile cukierków otrzyma każdy z chłopców?
	\item Jeden z boków kwadratu wydłużono o 3 cm, a drugi skrócono o 1 cm. Otrzymano prostokąt o obwodzie 32cm. Oblicz pole i obwód tego początkowego kwadratu.
	\item Siostra Marty jest od niej o 7 lat starsza. Razem mają 33 lata. Ile lat ma każda z dziewcząt?
	\item W trójkącie prostokątnym różnica miar kątów ostrych wynosi 50° . Oblicz miary kątów tego trójkąta.
	\item Za 6 bułek i chleb Franek zapłacił 8 zł. Bułka jest 4 razy tańsza od chleba. Ile kosztuje bułka?
	\item Roman pisze powieść. Ma już $\frac{1}{3}$ zaplanowanej liczby stron. Jeszcze 37 i połowa powieści będzie gotowa. Ile stron ma liczyć powieść Romana?
	\item Jaś kupował haczyki w sklepie wędkarskim. Duży haczyk był o 7 groszy droższy od małego. Za 8 dużych haczyków i 6 małych zapłacił 3,50zł. Ile kosztował duży haczyk, a ile mały?
\end{enumerate}
\end{document}