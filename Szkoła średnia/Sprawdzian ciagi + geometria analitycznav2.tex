\documentclass[12pt,a4paper]{article}
\usepackage[utf8]{inputenc} %polskie znaki
\usepackage[T1]{fontenc}	%polskie znaki
\usepackage{amsmath}		%matematyczne znaczki :3
\usepackage{enumerate}		%Dodatkowe opcje do funkcji enumerate
\usepackage{geometry} 		%Ustawianie marginesow
\usepackage{graphicx}		%Grafika
\usepackage{wrapfig}		%Grafika obok textu
\usepackage{float}			%Allows H in fugire
\usepackage{hyperref}		%Allows hyperlinks
%\pagestyle{empty} 			%usuwa nr strony
\usepackage{todonotes}		%Todo notatki
\usepackage{lipsum}         %Lorem text
\usepackage{ntheorem}   	% for theorem-like environments
\usepackage{mdframed}   	% for framing
\usepackage{subcaption}		% subfigure (image placing)
\usepackage{pdfcomment}		% Komentarze (z bazowego pdf'a)
\usepackage{xparse}			% New commands with optional arguments
\usepackage{ifthen}			% If then - funkcje!
\usepackage{expl3}			% Deklarowanie zmiennych
\usepackage{pgf}			% Aktualne rachunki \pgfmathparse{}

\newgeometry{tmargin=2cm, bmargin=2cm, lmargin=2cm, rmargin=2cm} 

%Counter commands{
	\newcounter{counter} % Creates a new counter
	\setcounter{counter}{1} % Sets the counter to 5
	\newcommand{\counter}[1]{
		\arabic{#1} \stepcounter{#1} 
	}
	\newcommand{\counterreset}[1]{\setcounter{#1}{1}}
	%}

%Define styles{
	\theoremstyle{break}
	\theoreminframepreskip{0.5cm}
	\theoremheaderfont{\bfseries}
	\newmdtheoremenv[%
	linecolor=white,%
	innertopmargin=\topskip,
	shadowsize=0,%
	innertopmargin=5,%
	innerbottommargin=5,%
	leftmargin=10,%
	rightmargin=10,%
	backgroundcolor=gray!20,%
	innertopmargin=0pt,%
	ntheorem]{zad}{Zadanie}
	
	\mdfdefinestyle{zadanie}{
		linecolor=white,%
		innertopmargin=5,%
		innerbottommargin=5,%
		leftmargin=10,%
		rightmargin=10,%
		backgroundcolor=gray!20,%
		innertopmargin=8,
		innerbottommargin=8,
		skipabove = 5,
	}
	\mdfdefinestyle{wzor}{
		linecolor=cyan,%
		linewidth=2pt,%
		innertopmargin=8,
		innerbottommargin=8,
		leftmargin=10,%
		rightmargin=10,%
		backgroundcolor = white, 
		fontcolor = black,
		skipabove = 5,
		skipbelow = 5,
	}
	%}

%Zadania templatex%{
	\newcommand{\Wzor}[1]{
		\begin{mdframed}[style=wzor]
			\centering #1
		\end{mdframed}
	}
	\newcommand{\ZadanieTextowe}[1]{
		\begin{mdframed}[style=zadanie]
			\textbf{Zadanie \counter{counter} } \\
			#1
		\end{mdframed}
	}
	\newcommand{\Zadanie}[2]{
		\ZadanieTextowe{#1}
		#2
	}
	\newcommand{\ZadanieABCD}[6]{
		\ZadanieTextowe{#1}
		#2 \\\\
		\begin{tabular}{p{7cm} p{7cm}}
			\textbf{A. }#3&
			\textbf{B. }#4\\\\
			\textbf{C. }#5&
			\textbf{D. }#6\\
		\end{tabular}
	}
	\newcommand{\ZadanieABCDEF}[8]{
		\ZadanieTextowe{#1}
		#2 \\\\
		\begin{tabular}{p{7cm} p{7cm}}
			\textbf{A. }#3&
			\textbf{B. }#4\\\\
			\textbf{C. }#5&
			\textbf{D. }#6\\\\
			\textbf{E. }#7&
			\textbf{F. }#8\\\\
		\end{tabular}
	}
	
	\newcommand{\ZadaniePF}[3]{
		\ZadanieTextowe{#1}
		\textbf{Wybierz odpowiedź prawda (P) lub fałsz (F).} \\\\
		\PF{#2}{#3}
	}
	\newcommand{\PF}[2]{\setlength\arrayrulewidth{2pt}{
			\def\arraystretch{2}
			\begin{tabular}{|p{14cm}| c|c|}\hline
				#1 & {\Large P} & {\Large F}\\\hline
				#2 & {\Large P} & {\Large F}\\\hline
		\end{tabular}}
	}
	\newcommand{\PFEmpty}[2]{\setlength\arrayrulewidth{2pt}{
			\def\arraystretch{2}
			\begin{tabular}{|p{14cm}| c|}\hline
				#1 & {\Large $\dots$}\\\hline
				#2 & {\Large $\dots$}\\\hline
		\end{tabular}}
	}
	\newcommand{\Zadanietwoxtwo}[5]{
		\ZadanieTextowe{#1}
		\begin{tabular}{p{7cm} p{7cm}}
			\textbf{a)} #2&
			\textbf{b)} #3\\\\
			\textbf{c)} #4&
			\textbf{d)} #5\\\\
		\end{tabular}
	}
	\newcommand{\Zadanietwoxthree}[7]{
		\ZadanieTextowe{#1}
		\begin{tabular}{p{7cm} p{7cm}}
			\textbf{a)} #2&
			\textbf{b)} #3\\\\
			\textbf{c)} #4&
			\textbf{d)} #5\\\\
			\textbf{e)} #6&
			\textbf{f)} #7\\\\
		\end{tabular}
	}
	\newcommand{\Zadanietwoxfour}[9]{
		\ZadanieTextowe{#1}
		\begin{tabular}{p{7cm} p{7cm}}
			\textbf{a)} #2&
			\textbf{b)} #3\\\\
			\textbf{c)} #4&
			\textbf{d)} #5\\\\
			\textbf{e)} #6&
			\textbf{f)} #7\\\\
			\textbf{g)} #8&
			\textbf{h)} #9\\\\
		\end{tabular}
	}
	\newcommand{\Informacja}[2]{
		\begin{mdframed}[style=zadanie]
			\textbf{Informacja do zadań \arabic{counter} - \pgfmathparse{\arabic{counter}+#1-1}\pgfmathprintnumber[assume math mode=true, int detect]{\pgfmathresult}}
		\end{mdframed}
		#2
	}
	
	%}

\newcommand{\tg}{\text{tg}}
\newcommand{\ctg}{\text{ctg}}
\newcommand{\UkladRownan}[2]{
	$\left\{
	\begin{array}{l}
		#1 \\
		#2
	\end{array}
	\right.$
}

\begin{document}
	\begin{center}
		\LARGE Ciągi i geometria analityczna - egzamin
	\end{center}
	\vspace{0.5cm}
	\begin{tabular}{p{13cm} r}
		Imię i nazwisko:\dotfill
		&[....../25 pkt]\\ 
	\end{tabular}
	
	\ZadanieABCD{Dany jest ciąg geometryczny, w którym $a_3=4$ i $a_5=8$}{Wówczas iloraz tego ciągu wynosi:}{2}{$\frac{1}{2}$}{$\sqrt{2}$}{$\frac{1}{\sqrt{2}}$}
	
	\ZadanieTextowe{Dany jest ciąg arytmetyczny, którego suma trzeciego i piątego wyrazu wynosi 4, natomiast różnica siódmego i drugiego wynosi -12. Wyznaczyć wzór ogólny tego ciągu.}
	
	\ZadanieTextowe{Dany jest 3-wyrazowy ciąg
		$$(x+2,\qquad -2x-4,\qquad 6x+2)$$
		Wyznaczyć dla jakich wartości "x" ciąg ten jest ciągiem geometrycznym?}
	
	\ZadanieTextowe{Wyznaczyć sumę wszystkich liczb czterocyfrowych, których reszta z dzielenia przez 7 wynosi 3.}
	
	\ZadaniePF{Dana jest prosta $k$ o równaniu $y=\frac{1}{3}x-4$}{Do prostej $k$ należy punkt $P=(6,-2)$.}{Prosta $k$ jest malejąca.}
	
	\ZadanieABCD{Dany jest trójkąt $ABC$ o wierzchołkach $A=(-1,2)$, $B=(3,5)$, $C=(0,9)$.}{Obwód tego trójkąta wynosi:}{$22$}{$16$}{$5+\sqrt{7}+5\sqrt{2}$}{$10+5\sqrt{2}$}
	
	\ZadanieTextowe{Dany jest równoległobok $ABCD$, gdzie $A=(-3,6)$, $B=(5,-2)$ oraz punkt $S=(4,4)$, który jest środkiem symetrii tego równoległoboku. Wyznaczyć punkty $C$ i $D$.}
	
	\ZadanieTextowe{Zapisać wartości parametru $m$, dla których funkcja
		$$y= \left(5-3m \right)x-\frac{3m-2}{6}$$
		jest niemalejąca.}
	
	\ZadanieTextowe{Wyznaczyć symetralną odcinka $AB$, gdzie $A=(4,2)$ i $B=(-2,-1)$.}
	
	\ZadanieABCD{Dana jest prosta $y=3x-6$.}{Prosta do niej prostopadła na równanie:}{$y=3x+6$}{$y=\frac{1}{3}x-6$}{$y=-3x+\frac{\sqrt{6}}{6}$}{$y=-\frac{1}{3}x+\sqrt{6}$}
	
	
	
	\Zadanie{Zapisać równanie okręgu o środku $S=(3,-8)$ i promieniu $4\sqrt{2}$.}{\vspace{0.5cm}	\begin{center}
			.................................................................................................
	\end{center}}
	
	\begin{center}
		
		\begin{tabular}{|c|c|c|c|c|c|c|c|c|c|c|c|}\hline
			Zadanie&1&2&3&4&5&6&7&8&9&10&1\\\hline
			Max&1&3&3&3&2&1&3&2&4&1&2\\\hline
			Punkty&&&&&&&&&&&\\\hline
		\end{tabular}
	\end{center}
\end{document}

