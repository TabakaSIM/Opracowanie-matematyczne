\documentclass[12pt,a4paper]{article}
\usepackage[utf8]{inputenc} %polskie znaki
\usepackage[T1]{fontenc}	%polskie znaki
\usepackage{amsmath}		%matematyczne znaczki :3
\usepackage{enumerate}		%Dodatkowe opcje do funkcji enumerate
\usepackage{geometry} 		%Ustawianie marginesow
\usepackage{graphicx}		%Grafika
\usepackage{wrapfig}		%Grafika obok textu
\usepackage{float}			%Allows H in fugire
%\pagestyle{empty} 			%usuwa nr strony

\newgeometry{tmargin=2cm, bmargin=2cm, lmargin=2cm, rmargin=2cm} 

\begin{document}
	\begin{center}
		\LARGE Stereometria
	\end{center}
	
	\begin{center}
		\large Graniastosłupy
	\end{center}
	
	\begin{enumerate}[1.]
	\item Wyznacz liczbę ścian, wierzchołków i krawędzi graniastosłupa:
		\begin{enumerate}[a)]
			\item trójkątnego
			\item czworokątnego
			\item sześciokątnego
			\item dziesięcokątnego*
			\item dwudziestoczterokątnego*
		\end{enumerate}
		
	*Spróbuj wyznaczyć wzory ogólne na liczbę ścian, wierzchołków i krawędzi korzystając z podpunktów a) b) c) w zależności od kątów w figurze w podstawie, a następnie podstaw pod wzory.
		
	\item Dany jest graniastosłup o 48 krawędziach. Wyznacz liczbe jego ścian.
	\item Dany jest graniastosłup o 10 ścianach. Wyznacz liczbę jego wierzchołków.
		
	\item Dany jest granastosłup czworokątny prosty, którego podstawą jest prostokąt o bokach 3 i 4, oraz o wysokości równej 12. Oblicz jego:
	\begin{enumerate}[a)]
		\item objętość
		\item pole powierzchni całkowitej
		\item długość przekątnej granastosłupa
	\end{enumerate}
	
	\item Przekątna pewnego graniastołupa prawidłowego czworokątnego ma długość 8 i jest nachylona do płaszczyzny podstawy pod kątem $60^\circ$. Oblicz objętośc i pole powierzchni całkowitej tego graniastosłupa.
	
	\item Krawędź podstawy graniastosłupa prawidłowego czworokątnego ma długość 4, a przekątna tego graniastosłupa ma długość 9. Oblicz objętość i pole powierzchni całkowitej tego graniastosłupa.
	
	\item Podstawą pewnego graniastosłupa jest kwadrat. Przekątna graniastosłupa ma długość 2 i tworzy z krawędzią podstawy kąt $45^\circ$. Oblicz objętość i pole powierzchni tego graniastosłupa.
	
	\item Oblicz objętość i pole powierzchni całkowitej graniastosłupa prawidłowego trójkątnego, w którym długość krawiędzi podstawy jest równa 20, a kąt nachylenia ściany bocznej do sąsiedzniej ściany bocznej wynosi $45^\circ$.
	
	\item Dany jest granistosłup prosty, którego podstawą jest trójkąt prostokątny o przyprostokątnych 3 i 4. O jego wysokości wiadomo, że jest równy średniej arytmetycznej sumy boków jego podstawy. Oblicz objętość i pole powierzchni całkowitej tego graniastosłupa.
	
	\item Oblicz objętość i pole powierzchni całkowitej sześcianu, którego suma krawędzi wynosi 72.
\end{enumerate}
	
	
	
	\newpage
	\begin{center}
		\large Ostrosłupy
	\end{center}
	\begin{enumerate}[1.]
	\item Wyznacz liczbę ścian, wierzchołków i krawędzi ostrosłupa:
	\begin{enumerate}[a)]
		\item trójkątnego
		\item czworokątnego
		\item sześciokątnego
		\item dziesięcokątnego*
		\item dwudziestoczterokątnego*
	\end{enumerate}
	
	\item Dany jest ostrosłup o 20 krawędziach. Wyznacz liczbę jego ścian.

	\item Dany jest ostrosłup prawidłowy czworokątny, którego krawędź podstawy wynosi 4, a wysokość 8. Oblicz jego objętość oraz pole powierzchni bocznej.
	
	\item Dany jest ostrosłup prawidłowy trójkątny o krawędzi podstawy równej 10 oraz krawędzi bocznej równej 13. Oblicz jego:
	\begin{enumerate}[a)]
		\item objętość
		\item pole powierzchni całkowitej
		\item sinus kąta, jaki tworzy krawędź boczna z wysokością
		\item cosinus kąta, jaki tworzy podstawa z wysokością ściany bocznej
	\end{enumerate}

	\item Dany jest ostrosłup prawidłowy czworokątny o którym wiadomo, że krawędź boczna jest dwa razy dłuższa od krawędzi podstawy oraz jego objętośc wynosi 7056. Oblicz jego krawędź podstawy.

	\item Dany jest ostrosłup prawidłowy trójkątny o wysokości równej 6. Wiedząc, że ściany boczne tego ostrosłupa tworzą z płaszczyzną podstawy kąt o mierze $30^\circ$ oblicz pole powierzchni całkowitej i objętość ostrosłupa.

	\item Wysokość ostrosłupa prawidłowego czworokątnego ma długośc 6 i tworzy z krawędzią boczną kąt o mierze $30^\circ$. Oblicz objętośc i pole powierzchni całkowitej tego ostosłupa.

	\item Dany jest ostrosłup prawidłowy sześciokątny o krawędzi bocznej równej 4. Wiedząc, że jego wysokość jest tej samej długości co najdłuższa przekątna podstawy, wyznacz:
	\begin{enumerate}[option]
		\item jego objętość
		\item jego pole powierzchni całkowitej
		\item tanges kąta między wysokością, a krawędzią boczną ostrosłupa.
	\end{enumerate}

	\end{enumerate}

\end{document}
