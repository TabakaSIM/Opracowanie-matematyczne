\documentclass[12pt,a4paper]{article}
\usepackage[utf8]{inputenc} %polskie znaki
\usepackage[T1]{fontenc}	%polskie znaki
\usepackage{amsmath}		%matematyczne znaczki :3
\usepackage{enumerate}		%Dodatkowe opcje do funkcji enumerate
\usepackage{geometry} 		%Ustawianie marginesow
\usepackage{graphicx}		%Grafika
\usepackage{wrapfig}		%Grafika obok textu
\usepackage{float}			%Allows H in fugire
\pagestyle{empty} 			%usuwa nr strony

\newgeometry{tmargin=2cm, bmargin=2cm, lmargin=2cm, rmargin=2cm} 

\begin{document}
	\begin{center}
		\LARGE Trygonometria - sprawdzian
	\end{center}
	\vspace{1.5cm}
	\begin{flushright}
		\textbf{GRUPA A}
	\end{flushright}
	\begin{tabular}{p{13cm} r}
		Imię i nazwisko: ............................................................................
		&[....../30pkt]\\ 
		\vspace{0.5cm}
	\end{tabular}
	\begin{enumerate}[1.]
		\item  \begin{tabular}{p{13cm} r}
			Wyznacz funkcje trygonometryczne obu kątów ostrych w trójkącie prostokątnym o przyprostokątnych $5$ i $5\sqrt{2}$. &[3pkt]\\ 
		\end{tabular}
		
		\item  \begin{tabular}{p{13cm} r}
			Promienie słoneczne padające pod kątem $40^\circ$ padają na słup, który rzuca cień o długości 20m. Oblicz wysokość tego słupa.
			&[3pkt]\\ 
		\end{tabular}
		
		\item \begin{tabular}{p{13cm} r}
			Wiemy o pewnym kącie $\alpha$, że $\sin \alpha = \frac{3}{5}$ oraz $\alpha\in (90^\circ,180^\circ)$. Oblicz pozostałe funkcje trygonometryczne tego kąta.
			&[4pkt]\\
		\end{tabular}
		
		\item \begin{tabular}{p{13cm} r}
			Oblicz: &[6pkt]\\ 
		\end{tabular}
		\begin{enumerate}[a)]
			\item $\sin210^\circ=$
			\item $\cos 690^\circ= $
			\item $(\sin 45^\circ + \cos 45^\circ) : \sin150^\circ= $
			\item $(\text{tg }135^\circ \cdot \text{tg } 60^\circ) \cdot \sin (-420^\circ)=$
		\end{enumerate}
	
		\item \begin{tabular}{p{13cm} r}
	Udowodnij, że podane równanie jest tożsamością trygonometryczną:
	&[5pkt]\\
\end{tabular}

$$(\sin\alpha - \cos\alpha)^2 + 2 \text{tg }\alpha \cdot \cos^2\alpha= 1$$
		
	\begin{tabular}{p{13cm} r}
	\item Dany jest trójkąt $ABC$, w którym bok $AB$ jest o 6 krótszy od boku $AC$ oraz $|BC|=5\sqrt{2}$. Wiedząc, że $\angle ABC = 135^\circ$: &[9pkt]\\ 
	\end{tabular}

	\begin{enumerate}[a)]
	\item Oblicz boki $AB$ i $AC$
	\item Oblicz pole tego trójkąta
	\item Wyznacz pozostałe kąty tego trójkąta
	\item Oblicz promień okręgu opisanego na tym trójkącie
	\end{enumerate}
		
	\end{enumerate}
	
	\newpage
	
	\begin{center}
		\LARGE Trygonometria - sprawdzian
	\end{center}
	\vspace{1.5cm}
	\begin{flushright}
		\textbf{GRUPA B}
	\end{flushright}
	\begin{tabular}{p{13cm} r}
		Imię i nazwisko: ............................................................................
		&[....../30pkt]\\ 
		\vspace{0.5cm}
	\end{tabular}
	\begin{enumerate}[1.]

		\item  \begin{tabular}{p{13cm} r}
			Wyznacz funkcje trygonometryczne obu kątów ostrych w trójkącie prostokątnym o przyprostokątnej równej 5 i przeciwprostokątnej równej 7. &[3pkt]\\ 
		\end{tabular}
	
		\item  \begin{tabular}{p{13cm} r}
			Postanowiono postawić znak drogowy o wysokości 240cm. Oblicz długość cienia który rzuca w momencie kiedy promienie słoneczne padaja na niego pod kątem $20^\circ$.
			&[3pkt]\\ 
		\end{tabular}
	
		\item \begin{tabular}{p{13cm} r}
			Wiemy o pewnym kącie $\alpha$, że $\text{tg } \alpha = \frac{1}{2}$ oraz $\alpha\in (180^\circ,270^\circ)$. Oblicz pozostałe funkcje trygonometryczne tego kąta.
			&[4pkt]\\
		\end{tabular}
	
		\item \begin{tabular}{p{13cm} r}
			Oblicz: &[6pkt]\\ 
		\end{tabular}
		\begin{enumerate}[a)]
			\item $\cos210^\circ=$
			\item $\sin 690^\circ= $
			\item $(\sin 150^\circ - \cos 120^\circ) : \sin300^\circ= $
			\item $(\text{tg }45^\circ \cdot \text{tg } 120^\circ) \cdot \cos (-450^\circ)=$
		\end{enumerate}
	
		\item \begin{tabular}{p{13cm} r}
		Udowodnij, że podane równanie jest tożsamością trygonometryczną:
		&[5pkt]\\
	\end{tabular}
	
	$$(\sin\alpha \cos\alpha)^2 - \text{tg }\alpha \cdot \cos^2\alpha= 1$$
	
		\begin{tabular}{p{13cm} r}
		\item Dany jest trójkąt $ABC$, w którym bok $AB$ jest o 6 krótszy od boku $AC$ oraz $|BC|=5\sqrt{2}$. Wiedząc, że $\angle ABC = 135^\circ$: &[9pkt]\\ 
	\end{tabular}
	
	\begin{enumerate}[a)]
		\item Oblicz boki $AB$ i $AC$
		\item Oblicz pole tego trójkąta
		\item Wyznacz pozostałe kąty tego trójkąta
		\item Oblicz promień okręgu opisanego na tym trójkącie
	\end{enumerate}
		
	\end{enumerate}
	
\end{document}