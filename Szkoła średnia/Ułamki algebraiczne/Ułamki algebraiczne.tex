\documentclass[12pt,a4paper]{article}
\usepackage[utf8]{inputenc} %polskie znaki
\usepackage[T1]{fontenc}	%polskie znaki
\usepackage{amsmath}		%matematyczne znaczki :3
\usepackage{enumerate}		%Dodatkowe opcje do funkcji enumerate
\usepackage{geometry} 		%Ustawianie marginesow
\usepackage{graphicx}		%Grafika
\usepackage{wrapfig}		%Grafika obok textu
\usepackage{float}			%Allows H in fugire
%\pagestyle{empty} 			%usuwa nr strony

\newgeometry{tmargin=2cm, bmargin=2cm, lmargin=2cm, rmargin=2cm} 

\begin{document}
	\begin{center}
		\LARGE Ułamki algebraiczne
	\end{center}
	\vspace{1cm}
	
	
	\begin{enumerate}[1.]
		
		\item Podaj dziedzinę i wykonaj mnożenie/dzielenie ułamków:
		\Large
		\begin{enumerate}[a)] \begin{tabular}{p{7cm} p{7cm}} 
				\item $\frac{3x^2-3}{x^2-4x-5} \cdot \frac{7x-35}{2x-2}=$& \vspace{0.4cm} \item $\frac{5x^2+7x+2}{x^2+2x+1}\cdot\frac{3x^2+2x-1}{25x^2-4}= $ \\
				\item $\frac{2x^2+6x}{x^2+4x+4}:\frac{x^2+6x+9}{x^2-4}= $& \item $\frac{x^3-5x^2+2x-10}{x^2+x-30}:\frac{x^3+2x}{x+6}= $ \\
				\item $\frac{x^2-1}{x^2+x-6}\cdot\frac{x^2+7x+12}{x^2+x-2}= $& \item $\frac{5x^2+7x+2}{x^2+2x+1}\cdot\frac{3x^2+2x-1}{25x^2-4}= $ \\
				\item $\frac{x^3+3x^2-x-3}{x^2-1}\cdot\frac{1}{x^2+8x+15}= $& \item $\frac{2x^3+6x^2+x+3}{4x^2+11x-3}\cdot\frac{3x^2+2x-1}{2x^3+2x^2+x+1}= $ \\
				\item $\frac{25x^2-10x+1}{x^2-9}:\frac{10x-2}{5x+15}= $& \item $\frac{x^3-5x^2+2x-10}{x^2+x-30}:\frac{x^3+2x}{x+6}= $ \\
		\end{tabular} \end{enumerate}
		\normalsize
		\item Podaj dziedzinę i wykonaj dodawanie/odejmowanie ułamków:
		\Large
		\begin{enumerate}[a)] \begin{tabular}{p{7cm} p{7cm}} 
				\item $\frac{x+1}{x-5}+\frac{2}{x+1}= $& \vspace{0.4cm} \item $\frac{x-1}{x^2+x-2}+\frac{x+5}{x^2-x-6}= $ \\
				\item $\frac{2x+4}{x+1}-\frac{2x}{x-8}= $& \item $\frac{x-1}{x^2-2x-15}-\frac{-2x-3}{x^2-x-20}= $ \\
				\item $\frac{x-1}{x-3}-\frac{x+1}{x-3}+\frac{x-4}{x-9}= $& \item $\frac{x-2}{x^2-4x}+\frac{x^2-1}{x^2-8x+16}-\frac{1}{2x}= $ \\
				\item $\frac{2x-1}{2x+1}+\frac{2-3x}{2x-1}= $& \item $\frac{1}{x-2}-\frac{1}{x}-\frac{1}{x^2-2x}= $ \\
				\item $\frac{5x}{x^2-6x+9}+\frac{4}{x-3}-\frac{1}{x^+3x}= $& \item $\frac{x-1}{x}-\frac{3}{x^2}+\frac{1}{x+1}= $ \\
		\end{tabular} \end{enumerate}	
		
		\normalsize
		\item Rozwiąż równania, pamiętaj o dziedzinie:
		\Large
		\begin{enumerate}[a)] \begin{tabular}{p{7cm} p{7cm}} 
				\item $\frac{x+1}{x-5}=\frac{x-2}{x+1} $& \vspace{0.4cm} \item $\frac{2x+3}{4x-5}=\frac{4x+5}{8x-7} $ \\
				\item $\frac{x^3-2x^2-5x+10}{2x-10}=0 $& \item $\frac{x^2+5}{3x^2-6x}=0 $ \\
				\item $\frac{3}{x^2-4}=1 $& \item $\frac{2x+3}{4x-5}=\frac{4x+5}{8x-7} $ \\
				\item $\frac{x}{x-3}+\frac{4}{x-3}=2x-2 $& \item $\frac{x+1}{x-3}+\frac{x-2}{x+1}=\frac{x^2+x+12}{x^2-2x-3} $  \\
		\end{tabular} \end{enumerate}	
		
		\normalsize
		\item Pierwsza koparka wykonała połowę wykopu w ciągu 6 godzin, resztę wykopu wykonała druga koparka w ciągu 9 godzin. Ile czasu zajęłoby wykonanie wykopu, gdyby obie koparki pracowały jednocześnie?
		
		\item Pompa o większej wydajności opróżnia pełny zbiornik w ciągu 8 godzin. Pompa o mniejszej wydajności opróżnia go w czasie trzy razy dłuższym, niż trwa opróżnianie zbiornika przez obie pompy pracujące jednocześnie. Ile czasu opróżnia zbiornik pompa o mniejszej wydajności?
		
		
		
	\end{enumerate}
\end{document}