\documentclass[12pt,a4paper]{article}
\usepackage[utf8]{inputenc} %polskie znaki
\usepackage[T1]{fontenc}	%polskie znaki
\usepackage{amsmath}		%matematyczne znaczki :3
\usepackage{enumerate}		%Dodatkowe opcje do funkcji enumerate
\usepackage{geometry} 		%Ustawianie marginesow
\usepackage{graphicx}		%Grafika
\usepackage{wrapfig}		%Grafika obok textu
\usepackage{float}			%Allows H in fugire
\usepackage{hyperref}		%Allows hyperlinks
%\pagestyle{empty} 			%usuwa nr strony
\usepackage{todonotes}		%Todo notatki
\usepackage{lipsum}         %Lorem text
\usepackage{ntheorem}   	% for theorem-like environments
\usepackage{mdframed}   	% for framing
\usepackage{subcaption}		% subfigure (image placing)
\usepackage{pdfcomment}		% Komentarze (z bazowego pdf'a)
\usepackage{xparse}			% New commands with optional arguments
\usepackage{ifthen}			% If then - funkcje!
\usepackage{expl3}			% Deklarowanie zmiennych

\newgeometry{tmargin=2cm, bmargin=2cm, lmargin=2cm, rmargin=2cm} 

%Counter commands{
	\newcounter{counter} % Creates a new counter
	\setcounter{counter}{1} % Sets the counter to 5
	\newcommand{\counter}[1]{
		\arabic{#1} \stepcounter{#1} 
	}
	\newcommand{\counterreset}[1]{\setcounter{#1}{1}}
	%}

%Define styles{
	\theoremstyle{break}
	\theoreminframepreskip{0.5cm}
	\theoremheaderfont{\bfseries}
	\newmdtheoremenv[%
	linecolor=white,%
	innertopmargin=\topskip,
	shadowsize=0,%
	innertopmargin=5,%
	innerbottommargin=5,%
	leftmargin=10,%
	rightmargin=10,%
	backgroundcolor=gray!20,%
	innertopmargin=0pt,%
	ntheorem]{zad}{Zadanie}
	
	\mdfdefinestyle{zadanie}{
		linecolor=white,%
		innertopmargin=5,%
		innerbottommargin=5,%
		leftmargin=10,%
		rightmargin=10,%
		backgroundcolor=gray!20,%
		innertopmargin=8,
		innerbottommargin=8,
		skipabove = 5,
	}
	\mdfdefinestyle{wzor}{
		linecolor=cyan,%
		linewidth=2pt,%
		innertopmargin=8,
		innerbottommargin=8,
		leftmargin=10,%
		rightmargin=10,%
		backgroundcolor = white, 
		fontcolor = black,
		skipabove = 5,
		skipbelow = 5,
	}
	%}

%Zadania templatex%{
	\newcommand{\Wzor}[1]{
		\begin{mdframed}[style=wzor]
			\centering #1
		\end{mdframed}
	}
	\newcommand{\ZadanieTextowe}[1]{
		\begin{mdframed}[style=zadanie]
			\textbf{Zadanie \counter{counter} } \\
			#1
		\end{mdframed}
	}
	\newcommand{\Zadanie}[2]{
		\ZadanieTextowe{#1}
		#2
	}
	\newcommand{\ZadanieABCD}[6]{
		\ZadanieTextowe{#1}
		#2 \\\\
		\begin{tabular}{p{7cm} p{7cm}}
			\textbf{A. }#3&
			\textbf{B. }#4\\\\
			\textbf{C. }#5&
			\textbf{D. }#6\\
		\end{tabular}
	}
	\newcommand{\ZadanieABCDEF}[8]{
		\ZadanieTextowe{#1}
		#2 \\\\
		\begin{tabular}{p{7cm} p{7cm}}
			\textbf{A. }#3&
			\textbf{B. }#4\\\\
			\textbf{C. }#5&
			\textbf{D. }#6\\\\
			\textbf{E. }#7&
			\textbf{F. }#8\\\\
		\end{tabular}
	}
	\newcommand{\Zadanietwoxtwo}[5]{
		\ZadanieTextowe{#1}
		\begin{tabular}{p{7cm} p{7cm}}
			\textbf{a)} #2&
			\textbf{b)} #3\\\\
			\textbf{c)} #4&
			\textbf{d)} #5\\\\
		\end{tabular}
	}
	\newcommand{\Zadanietwoxthree}[7]{
		\ZadanieTextowe{#1}
		\begin{tabular}{p{7cm} p{7cm}}
			\textbf{a)} #2&
			\textbf{b)} #3\\\\
			\textbf{c)} #4&
			\textbf{d)} #5\\\\
			\textbf{e)} #6&
			\textbf{f)} #7\\\\
		\end{tabular}
	}
	\newcommand{\Zadanietwoxfour}[9]{
		\ZadanieTextowe{#1}
		\begin{tabular}{p{7cm} p{7cm}}
			\textbf{a)} #2&
			\textbf{b)} #3\\\\
			\textbf{c)} #4&
			\textbf{d)} #5\\\\
			\textbf{e)} #6&
			\textbf{f)} #7\\\\
			\textbf{g)} #8&
			\textbf{h)} #9\\\\
		\end{tabular}
	}
	%}

\newcommand{\tg}{\text{tg}}
\newcommand{\ctg}{\text{ctg}}
\newcommand{\UkladRownan}[2]{
	$\left\{
	\begin{array}{l}
		#1 \\
		#2
	\end{array}
	\right.$
}

\begin{document}
	\begin{center}
		\LARGE Funkcja kwadratowa
	\end{center}
	
	\Zadanie{Uzupełnij tabelę oraz podaj własności fukcji kwadratowej (zbiór wartości, przedziały monotoniczności, maksimum/minimum, wierzchołek i miejsca zerowe)}{
		\begin{tabular}{| p{5cm}| p{5cm}| p{5cm}|}\hline
			\begin{center}
				Postać kanoniczna
			\end{center}&\begin{center}
			Postać ogólna
		\end{center}&\begin{center}
		Postać iloczynowa
	\end{center}\\
			\hline
			&&\\&$x^2-4x+3$&\\&&\\\hline
			&&\\&&$2(x-1)(x-3)$\\&&\\\hline
			&&\\&$x^2-6x+9$&\\&&\\\hline
			&&\\$-\frac{1}{2}(x-3)^2+9$&&\\&&\\\hline
			&&\\&$-3x^2+6x$&\\&&\\\hline
			&&\\&&$-(x+2)(x-6)$\\&&\\\hline
			&&\\&$x^2+5x+13$&\\&&\\\hline
			&&\\$-2(x+2)^2-2$&&\\&&\\\hline
		\end{tabular}
	}
	\ZadanieTextowe{Dana jest funkcja kwadratowa, która ma tylko jedno miejsce zerowe oraz jest rosnąca w przedziale $\langle -2,\infty )$. Wiedząc, że przechodzi ona przez punkt $P=(0,2)$ wyznacz jej w wzór w postaci ogólnej.}
	\ZadanieTextowe{Pewna funkcja kwadratowa przyjmuje największą wartość równą 3, a jej dwa miejsca zerowe to -2 i 4. Wyznacz jej wzór w postaci ogólnej.}
	\newpage
	\ZadanieTextowe{Rozwiąż równania i nierówności:}
	\begin{enumerate}[a)]
		\item $x^2+2x=3$
		\item $2x^2+6x-8=0$
		\item $-x^2+4x-4=0$
		\item $2x^2-7x=15$
		\item $x^2+4x-5=0$
		\item $x^2+7x+12\leq 0$
		\item $x^2-16>0$
		\item $-x^2+6x-13<0$
		\item $2x^2+8x+8>0$
		\item $x^2-6x+6<0$
		\item $x^2-10x+25\leq0$
		\item $x(2-x)<3x-6$
		\item $\frac{x+2}{3}-x^2<4$
		\item $10(x-5)-2x(x-5)=2$
		\item $4(x+1)^2-(2x-2)^2<16$
	\end{enumerate}
	\newpage
	\ZadanieTextowe{ Dane są dwie liczby całkowite, których suma wynosi 36. Wyznacz te liczby tak, aby suma ich kwadratów była jak
	najmniejsza.}
	\ZadanieTextowe{Dany jest trójkąta równoramienny $ABC$, którego suma długości podstawy $AB$ i wysokości poprowadzonej z wierzchołka $C$ wynosi 14cm. Oblicz długość podstawy $AB$, dla którego pole tego trójkąta jest największe.}
	\ZadanieTextowe{Pani Ania sprzedaje w swoim sklepie książki. Każdą książkę kupuje za 20 zł, a sprzedaje za 35 zł. W miesiącu Pani Ania sprzedaje 50 takich książek. Zakładając, że obniżenie
	ceny książki o 5 zł powoduje wzrost liczby sprzedanych sztuk o 2. Oblicz przy jakiej cenie książki zysk byłby największy.}
	\ZadanieTextowe{Liczbę 7 dzielimy na trzy części tak, aby pierwsza była dwa razy większa od drugiej. Jak należy dokonać
	tego podziału, aby suma kwadratów wszystkich trzech części była najmniejsza?}
	\ZadanieTextowe{Firma wynajmująca samochody ma do dyspozycji 180 pojazdów. Wszystkie samochody są wynajęte wówczas, gdy koszt wynajmu jednego samochodu za jeden tydzień wynosi 1200 zł. Właściciel firmy oszacował, że każda kolejna podwyżka ceny wynajmu samochodu o 40 zł tygodniowo powoduje zmniejszenie liczby wynajmowanych samochodów o 3. Jaki tygodniowy koszt wynajmu powinna ustalić firma, aby jej przychód był maksymalny? Ile wynosi ten największy przychód?}
	\newpage
	\counterreset{counter}
	
	\Zadanie{Informacja do zadań \textbf{2-4}.}{
Poniżej przedstawiono fragment funkcji kwadratowej $f(x)$:
\begin{center}
	\includegraphics[scale=0.4]{4_1_1.png}
\end{center}	
}
\ZadanieABCD{Dokończ zdanie. Wybierz odpowiedź spośród ABCD}{Oś symetrii tej funkcji kwadratowej da się zapisać za pomocą równania}{$x=3$}{$x=-3$}{$y=5$}{$y=-3$}
	\Zadanie{Zapisz zbiór wartości powyższej funkcji kwadratowej}{
	\vspace{0.5cm}	\dots\dots\dots\dots\dots\dots\dots\dots\dots\dots\dots\dots\dots\dots\dots\dots\dots\dots\dots\dots\dots\dots\dots\dots\dots\dots\dots\dots\dots\dots\dots}
\ZadanieTextowe{Wyznacz wzór tej funkcji kwadratowej w postaci ogólnej.}
\end{document}