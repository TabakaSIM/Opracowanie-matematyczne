\documentclass[12pt,a4paper]{article}
\usepackage[utf8]{inputenc} %polskie znaki
\usepackage[T1]{fontenc}	%polskie znaki
\usepackage{amsmath}		%matematyczne znaczki :3
\usepackage{enumerate}		%Dodatkowe opcje do funkcji enumerate
\usepackage{geometry} 		%Ustawianie marginesow
\usepackage{graphicx}		%Grafika
\usepackage{wrapfig}		%Grafika obok textu
\usepackage{float}			%Allows H in fugire
\usepackage{hyperref}		%Allows hyperlinks
%\pagestyle{empty} 			%usuwa nr strony
\usepackage{todonotes}		%Todo notatki
\usepackage{lipsum}         %Lorem text
\usepackage{ntheorem}   	% for theorem-like environments
\usepackage{mdframed}   	% for framing
\usepackage{subcaption}		% subfigure (image placing)
\usepackage{pdfcomment}		% Komentarze (z bazowego pdf'a)
\usepackage{xparse}			% New commands with optional arguments
\usepackage{ifthen}			% If then - funkcje!
\usepackage{expl3}			% Deklarowanie zmiennych

\newgeometry{tmargin=2cm, bmargin=2cm, lmargin=2cm, rmargin=2cm} 

%Counter commands{
	\newcounter{counter} % Creates a new counter
	\setcounter{counter}{1} % Sets the counter to 5
	\newcommand{\counter}[1]{
		\arabic{#1} \stepcounter{#1} 
	}
	\newcommand{\counterreset}[1]{\setcounter{#1}{1}}
	%}

%Define styles{
	\theoremstyle{break}
	\theoreminframepreskip{0.5cm}
	\theoremheaderfont{\bfseries}
	\newmdtheoremenv[%
	linecolor=white,%
	innertopmargin=\topskip,
	shadowsize=0,%
	innertopmargin=5,%
	innerbottommargin=5,%
	leftmargin=10,%
	rightmargin=10,%
	backgroundcolor=gray!20,%
	innertopmargin=0pt,%
	ntheorem]{zad}{Zadanie}
	
	\mdfdefinestyle{zadanie}{
		linecolor=white,%
		innertopmargin=5,%
		innerbottommargin=5,%
		leftmargin=10,%
		rightmargin=10,%
		backgroundcolor=gray!20,%
		innertopmargin=8,
		innerbottommargin=8,
		skipabove = 5,
	}
	\mdfdefinestyle{wzor}{
		linecolor=cyan,%
		linewidth=2pt,%
		innertopmargin=8,
		innerbottommargin=8,
		leftmargin=10,%
		rightmargin=10,%
		backgroundcolor = white, 
		fontcolor = black,
		skipabove = 5,
		skipbelow = 5,
	}
	%}

%Zadania templatex%{
	\newcommand{\Wzor}[1]{
		\begin{mdframed}[style=wzor]
			\centering #1
		\end{mdframed}
	}
	\newcommand{\ZadanieTextowe}[1]{
		\begin{mdframed}[style=zadanie]
			\textbf{Zadanie \counter{counter} } \\
			#1
		\end{mdframed}
	}
	\newcommand{\Zadanie}[2]{
		\ZadanieTextowe{#1}
		#2
	}
	\newcommand{\ZadanieABCD}[6]{
		\ZadanieTextowe{#1}
		#2 \\\\
		\begin{tabular}{p{7cm} p{7cm}}
			\textbf{A. }#3&
			\textbf{B. }#4\\\\
			\textbf{C. }#5&
			\textbf{D. }#6\\
		\end{tabular}
	}
	\newcommand{\ZadanieABCDEF}[8]{
		\ZadanieTextowe{#1}
		#2 \\\\
		\begin{tabular}{p{7cm} p{7cm}}
			\textbf{A. }#3&
			\textbf{B. }#4\\\\
			\textbf{C. }#5&
			\textbf{D. }#6\\\\
			\textbf{E. }#7&
			\textbf{F. }#8\\\\
		\end{tabular}
	}
	\newcommand{\Zadanietwoxtwo}[5]{
		\ZadanieTextowe{#1}
		\begin{tabular}{p{7cm} p{7cm}}
			\textbf{a)} #2&
			\textbf{b)} #3\\\\
			\textbf{c)} #4&
			\textbf{d)} #5\\\\
		\end{tabular}
	}
	\newcommand{\Zadanietwoxthree}[7]{
		\ZadanieTextowe{#1}
		\begin{tabular}{p{7cm} p{7cm}}
			\textbf{a)} #2&
			\textbf{b)} #3\\\\
			\textbf{c)} #4&
			\textbf{d)} #5\\\\
			\textbf{e)} #6&
			\textbf{f)} #7\\\\
		\end{tabular}
	}
	\newcommand{\Zadanietwoxfour}[9]{
		\ZadanieTextowe{#1}
		\begin{tabular}{p{7cm} p{7cm}}
			\textbf{a)} #2&
			\textbf{b)} #3\\\\
			\textbf{c)} #4&
			\textbf{d)} #5\\\\
			\textbf{e)} #6&
			\textbf{f)} #7\\\\
			\textbf{g)} #8&
			\textbf{h)} #9\\\\
		\end{tabular}
	}
	%}

\newcommand{\tg}{\text{tg}}
\newcommand{\ctg}{\text{ctg}}
\newcommand{\UkladRownan}[2]{
	$\left\{
	\begin{array}{l}
		#1 \\
		#2
	\end{array}
	\right.$
}

\begin{document}
	\begin{center}
		\LARGE Wielomiany i ułamki algebraiczne
	\end{center}
	\Wzor{Metody rozkładania wielomianów:
	\begin{enumerate}[1)]
		\item $\Delta$?
		\item Wyciąganie czynnika przed nawias
		\item Metoda podstawiania $(t=x^2)$, $(t=x^3)$
		\item Metoda grupowania
		\item Wzory skróconego mnożenia*
		\item Twierdzenie Bezouta i schemat Hornera
\end{enumerate}}
	\Zadanietwoxtwo{Rozłóż na czynniki, a następnie rozwiąż równanie stosując metodę wyciągania czynnika przed nawias:}
	{$x^4+6x^3+5x^2=0$}{$x^6-4x^4=0$}
	{$5x^3+20x^2+25x=0$}{$-2x^5-2x^4+24x^3=0$}
	\Zadanietwoxtwo{Rozłóż na czynniki, a następnie rozwiąż równanie stosując metodę podstawiania:}
	{$x^4+3x^2-4=0$}{$x^4-13x^2+36=0$}
	{$x^6-7x^3-8=0$}{$x^5+20x^3+64x=0$}
	\Zadanietwoxthree{Rozłóż na czynniki, a następnie rozwiąż równanie stosując metodę grupowania:}
	{$x^3-4x-2x+8=0$}{$x^3+x^2+9x+9=0$}
	{$2x^3+3x^2-10x-15=0$}{$x^2(x-3)=2(x-3)$}
	{$3x^4-2x^2-12x+8$}{$4x^4+x^2-36x-9=0$}
	\newpage
	\Zadanietwoxthree{Rozłóż na czynniki, a następnie rozwiąż równanie stosując twierdzenie Bezouta i schemat Hornera:}
	{$x^3-6x^2+5x+12=0$}{$x^3-x^2-15x-25=0$}
	{$x^4+2x^3-16x^2-2x+15=0$}{$-2x^3+16x^2-40x+32=0$}
	{$x^4-2x^3-11x^2+12x36=0$}{$-3x^5-6x^4-18x^3-30x^2-15x=0$}
	\Zadanietwoxtwo{Wykonaj działanie dodawania/odejmowania (Pamiętaj o określeniu dziedziny!)}
	{\Large$\frac{2}{x-4}+\frac{x-1}{x+1}=$}
	{\Large$\frac{x+4}{x-1}-\frac{2}{x+1}=$}
	{\Large$\frac{x-3}{x^2-2x}+\frac{x+3}{x^2-4}=$}
	{\Large$\frac{3}{x^2-3x+2}-\frac{x-3}{x^2+4x-5}=$}
	\Zadanietwoxtwo{Wykonaj działanie mnożenia/dzielenia (Pamiętaj o określeniu dziedziny!)}
	{\Large$\frac{3x^2+3x}{x^2-1}\cdot\frac{x-1}{9x}=$}
	{\Large$\frac{2x^2+5x}{x^2+4x+4}:\frac{x^2+6x+9}{x^2-4}=$}
	{\Large$\frac{3x^2-3}{x^2-4x-5}\cdot\frac{5x-25}{15x-15}=$}
	{\Large$\frac{x^3-5x^2+2x-10}{x^2+x-30}:\frac{x^3+2x}{x+6}=$}
	\Zadanietwoxthree{Rozwiąż równanie (Pamiętaj o określeniu dziedziny!)}
	{\Large$\frac{x+3}{x-3}=0$}{\Large$\frac{(x-2)(4-8x)}{x^-4}=0$}
	{\Large$\frac{x^2+4}{x^3-4x}=0$}{\Large$\frac{(2x-1)(2x+1)(x-1)}{2x^2-x-1}=0$}
	{\Large$\frac{x+1}{x+2}=\frac{x+2}{x-3}$}{\Large$\frac{x+3}{x+5}=\frac{x-3}{x-3}$}
	\newpage
	\ZadanieABCD{Dany jest wielomian f(x) z parametrem "m":
	$$x^3-mx^2+11x-28$$
	Wiemy o tym wielomianie, że jest on podzielny przed dwómian $(x-4)$}{Zatem parametr "m" jest wynosi:}
	{$m=1$}{$m=5$}
	{$m=-2$}{$m=-5$}
	\ZadanieTextowe{Rozwiąż równanie
	$$-2x^6+4x^4+16x^2=0$$}
	\ZadanieTextowe{Rozwiąż równanie
		$$2x^5-6x^3=8x$$}
	\ZadanieABCD{Dane jest równanie: $$\frac{x^3-4x^2+3x}{(x-1)(x+2)(x-3)}=0$$}{Równanie to ma:}{0 rozwiązań}{1 rozwiązanie}{2 rozwiązania}{3 rozwiązania}
\end{document}