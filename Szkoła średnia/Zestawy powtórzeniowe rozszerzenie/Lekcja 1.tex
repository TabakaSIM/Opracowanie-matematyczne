\documentclass[12pt,a4paper]{article}
\usepackage[utf8]{inputenc} %polskie znaki
\usepackage[T1]{fontenc}	%polskie znaki
\usepackage{amsmath}		%matematyczne znaczki :3
\usepackage{enumerate}		%Dodatkowe opcje do funkcji enumerate
\usepackage{geometry} 		%Ustawianie marginesow
\usepackage{graphicx}		%Grafika
\usepackage{wrapfig}		%Grafika obok textu
\usepackage{float}			%Allows H in fugire
\usepackage{hyperref}		%Allows hyperlinks
\usepackage{xcolor}
\pagestyle{empty} 			%usuwa nr strony

\newgeometry{tmargin=2cm, bmargin=2cm, lmargin=2cm, rmargin=2cm} 

\begin{document}
	
	\begin{center}
		\LARGE Zadania na lekcje
	\end{center}
	\vspace{1.5cm}
	
	\begin{enumerate}[1.]
		\item Wykaż, że:
		
		\begin{enumerate}[a)]
			\item jeśli $x+y=2$, to $x^3+y^3\geq 2$
			\item jesli $x-y=5$, to $x^3-y^3\geq31,25$
		\end{enumerate}
	
	\item Wyznacz wzór funkcji kwadratowej $f$ w postaci ogólnej, jeśli wiadomo, że przyjmuje ona wartości dodatnie $\Leftrightarrow x\in (-8;2)$, zaś największą wartością tej funkcji jest $2\frac{1}{4}$.
	
	\item Dana jest fukcja kwadratowa, dla której $f(-3)=0$ oraz $f(-1)=f(5)=3$. Czy funkcja ma najmniejszą czy największą wartość? Wyznacz tę wartość.
	
	\item Oblicz najmniejszą i najwięszką wartość funkcji:
	\begin{enumerate}[a)]
		\item $x^2-4x+5$, gdzie $x\in\langle0;3\rangle$
		\item $-\frac{1}{2}x^2+2x$, gdzie $x\in\langle0;6\rangle$
		\item $\frac{1}{4}x^2+2x+3$, gdzie $x\in\langle-2;6\rangle$
	\end{enumerate} 

	\item Dla jakich wartości parametru "k" pierwiastki funkcji
	
	$$x^2-(k-1)x + 1 = 0$$ 
	
	spełniają warunek $\frac{1}{x_1^2}+\frac{1}{x_2^2}\geq 2k^2-k-21$?
	
	\item Wyznacz wszystkie wartości paramatru "k", dla których rozwiązania $x_1,x_2$ równania
	
	$$2x^2+kx+2k=0$$
	
	spełniają warunek: $x_1^2 x_2+x_1 x_2^2 + 3x_1x_2\geq x_1+x_2-4$.
		
	\end{enumerate}

	\newpage
	
	\begin{center}
		\LARGE Zadania domowe
	\end{center}
	\vspace{1.5cm}
	
	\begin{enumerate}[1.]
		\item Przekształć poniższe wyrażenia tak, aby skorzystać ze wzorów Viete'a, a następnie je zastosuj dla funkcji: $f(x)=\sqrt{3}x^2-(2\sqrt{3}+1)x+\sqrt{2}$:
		
		\begin{enumerate}[a)]
			\item $\frac{1}{x_1}+\frac{1}{x_2}$
			\item $x_1^2+x_2^2$
			\item $\frac{1}{x_1^2}+\frac{1}{x_2^2}$
			\item $\frac{x_1}{x_2}+\frac{x_2}{x_1}$
		\end{enumerate}
	
		\item Wykaż, że:
		\begin{enumerate}[a)]
			\item $(x_1-x_2)^2=(x_1+x_2)^2-4x_1x_2$
			\item $|x_1-x_2|=\sqrt{(x_1+x_2)^2-4x_1x_2}$
			\item $x_1^3+x_2^3=(x_1+x_2)^3-3(x_1+x_2)\cdot x_1x_2$
			\item $x_1^4+x_2^4=((x_1+x_2)^2-2x_1x_2)^2-2(x_1x_2)^2$
		\end{enumerate}
	\end{enumerate}
	\textcolor{red}{(spoiler text for the hint goes here)}
	
	
	
\end{document}