\documentclass[12pt,a4paper]{article}
\usepackage[utf8]{inputenc} %polskie znaki
\usepackage[T1]{fontenc}	%polskie znaki
\usepackage{amsmath}		%matematyczne znaczki :3
\usepackage{enumerate}		%Dodatkowe opcje do funkcji enumerate
\usepackage{geometry} 		%Ustawianie marginesow
\usepackage{graphicx}		%Grafika
\usepackage{wrapfig}		%Grafika obok textu
\usepackage{float}			%Allows H in fugire
\usepackage{hyperref}		%Allows hyperlinks
\pagestyle{empty} 			%usuwa nr strony

\newgeometry{tmargin=2cm, bmargin=2cm, lmargin=2cm, rmargin=2cm} 

\begin{document}
	
	\begin{center}
		\LARGE Zadania na lekcje 2
	\end{center}
	\vspace{1.5cm}
	
	\begin{enumerate}[1.]
		\item Określi liczbę rozwiązań równia w zależności od parametru "m""
		\begin{enumerate}[a)] \begin{tabular}{p{7cm} p{7cm}} 
				\item $(m-3)x^2+(m-2)x+1=0$& \vspace{0.4cm} 	\item $(2m-3)x^2+4mx+m-1=0$ \\
		\end{tabular} \end{enumerate}
	
		\item Dla jakich wartości parametru "m" fukncja ma dwa pierwiastki dodatnie:
		\begin{enumerate}[a)] \begin{tabular}{p{7cm} p{7cm}} 
				\item $f(x)=x^2+2(m-1)x+2m+1$& \vspace{0.4cm} 	\item $f(x)=x^2+(m-5)x+m-2$ \\
		\end{tabular} \end{enumerate}
	
		\item Dla jakich wartości parametru "m" fukncja ma dwa różne pierwiastki ujemne:
		\begin{enumerate}[a)] \begin{tabular}{p{7cm} p{7cm}} 
				\item $f(x)=x^2+5mx+4m^2-3m$& \vspace{0.4cm} 	\item $f(x)=(m-2)x^2-2mx+m^2-3m+4$ \\
		\end{tabular} \end{enumerate}
	
		\item Dla jakich wartości parametru "m" fukncji
		$$f(x)=x^2-2mx+2m-1$$
		ma dwa pierwiastki rzeczywiste, których suma jest równa sumie ich kwadratów?
		
		\item Wyznacz wszystkie wartości parametru "m", dla których równanie
		$$x^2+2(1-m)x+m^2-m=0$$
		ma dwa różne rozwiązania rzeczywiste spełniające warunek $x_1x_2\leq 6m \leq x_1^2+x_2^2 $.
		
		\item Wyznacz wszystkie wartości parametru "m", dla których równanie
		$$x^2-4mx-m^3+6m^2+m-2=0$$
		ma dwa różne pierwisatki rzeczywiste takie, że $(x_1-x_2)^2<9(m+1)$
	\end{enumerate}
	
	\newpage
	
	\begin{center}
		\LARGE Zadania domowe - lekcja 2
	\end{center}
	\vspace{1.5cm}
	
	\begin{enumerate}[1.]

	\item Dla jakich wartości parametru "m" funkcja
	$$f(x)=x^2+mx-16=0$$
	ma dwa pierwiastki, których suma odwrotności jest równa -4?
	
	\item Dla jakich wartości parametru "m" funkcja
	$$x^2-x+2m+3=0$$
	spełnia warunek $(x_1+x_2)^3-(x_1^3+x_2^3)=m^2+4m-6$
	
	\item Wyznacz wszystkie wartości parametru "m", dla których równanie $2x^2-(m-2)x-3m=0$ ma dwa różne pierwiastki rzeczywiste spełniające równanie $x_1^2+x_2^2-2x_1x_2<25$.
	
	\vspace{2cm}
	\Large Dokończ również zadania ze stony wcześniejszej!
	\end{enumerate}
	
	
	
\end{document}