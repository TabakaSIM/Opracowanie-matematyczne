\documentclass[12pt,a4paper]{article}
\usepackage[utf8]{inputenc} %polskie znaki
\usepackage[T1]{fontenc}	%polskie znaki
\usepackage{amsmath}		%matematyczne znaczki :3
\usepackage{enumerate}		%Dodatkowe opcje do funkcji enumerate
\usepackage{geometry} 		%Ustawianie marginesow
\usepackage{graphicx}		%Grafika
\usepackage{wrapfig}		%Grafika obok textu
\usepackage{float}			%Allows H in fugire
\usepackage{hyperref}		%Allows hyperlinks
\pagestyle{empty} 			%usuwa nr strony

\newgeometry{tmargin=2cm, bmargin=2cm, lmargin=2cm, rmargin=2cm} 

\begin{document}
	
	\begin{center}
		\LARGE Zadania na lekcje 3
	\end{center}
	\vspace{1.5cm}
	
	\begin{enumerate}[1.]
		\item Rozwiąż równania:
		\begin{enumerate}[a)] \begin{tabular}{p{7cm} p{7cm}} 
				\item $x^5+5x^4-14x^3=0$ & \vspace{0.4cm} 	\item$\sqrt{2}x^3+2x^2-4\sqrt{2}x=0$ \\
				\item $x^3+4x^2+8x+32=0$ & \item $x^6-7x^3-8=0$ \\
				\item $x^5+4x^3-8x^2-32=0$& \item $x^3+9x^2+27x=-31$ \\
				\item $x^3+x-2=0$& \item $x^4-3x^3-14x^2-20x-24=0$ \\
				\item $4x^3+2x^2-8x+3=0$& \item $6x^3-13x^2+9x-2=0$ \\
		\end{tabular} \end{enumerate}
		
		\item Wyznacz wartości "a" i "b" współczynników wielomianu 
		$$W(x)=x^3+ax^2+bx+1$$
		wiedząc, że $W(2)=7$ oraz że reszta z dzielenia $W(x)$ przez $(x-3)$ jest równa 10.
		
		\item Wyznacz wszystkie wartości parametru "m", dla którego wielomian
		$$x^3+(m-1)x-m=0$$
		ma dokładnie dwa pierwiastki rzeczywiste. Wyznacz te pierwiastki.
		
		\item Wielomian
		$$W(x)=(m-4)x^3-(m+6)x^2-(m-1)x+m+3$$
		jest podzielny przez dwumian $(x+1)$. Dla jakich wartości parametru "m" wielomian ten ma dokładnie dwa pierwiastki?
		
		\item Wyznacz te wartości parametru "m", dla których równanie
		$$(x^2-x-2)(x^2+(m-3)x+1)=0$$
		ma cztery różne rozwiązania
		\item Wykaż, że dla każdej wartości parametru "m" równanie
		$$x^3+2x+m^2x=2m^2+2x^2+4$$
		ma tylko jedno rozwiązanie.
		
	\end{enumerate}
	
	\newpage
	
	\begin{center}
		\LARGE Zadania domowe - lekcja 3
	\end{center}
	\vspace{1.5cm}
	
	\begin{enumerate}[1.]
		
		\item Rozwiąż równania:
		\begin{enumerate}[a)] \begin{tabular}{p{7cm} p{7cm}} 
				\item $5x^5+4x^4-5x-4=0$ & \vspace{0.4cm} 	\item$x^6-64=0$ \\
				\item $x^3+12x^2+44x+48=0$ & \item $x^8-15x^2-16=0$ \\
				\item $(x^2+x)^4-1=0$& \item $x^4-2x^2-2x-2=0$ \\
				\item $x^4+x^3-14x^2+26x-20=0$& \item $3x^3-7x^2-7x+3=0$ \\
				\item $(x-1)^3+(2x+3)^3=27x^3+8$& \item $x^4-x^3-x^2-x-2=0$ \\
		\end{tabular} \end{enumerate}
		
		\item Dla jakiego parametru "m" wielomian
		$$W(x)=6x^3+3x^2-5x+p$$
		jest podzielny przez dwumian $(x-1)$?
		
		\item Wielomian 
		$$W(x)=x^3-4x^2+cx+d$$
		jest podzielny przez wielomian $x^2-2x+1$. Wyznacz c i d.
		
		\item Dla jakich wartości parametru "m" równanie
		$$(m+2)x^3-2x^2+(m+3)x=0$$
		ma trzy równe rozwiązania?
		
		\item Dla jakich wartości parametru "m" równanie
		$$x^3-2(m+1)x^2+(2m^2+3m+1)x=0$$
		ma trzy rozwiązania, z których dwa są dodatnie?
		
		\item Dla jakich wartości parametru "m" równanie
		$$x^3+2(m-3)x^2+m^2=0$$
		ma cztery różne rozwiązania?
		
	\end{enumerate}
	
	
	
\end{document}