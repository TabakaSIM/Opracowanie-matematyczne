\documentclass[12pt,a4paper]{article}
\usepackage[utf8]{inputenc} %polskie znaki
\usepackage[T1]{fontenc}	%polskie znaki
\usepackage{amsmath}		%matematyczne znaczki :3
\usepackage{enumerate}		%Dodatkowe opcje do funkcji enumerate
\usepackage{geometry} 		%Ustawianie marginesow
\usepackage{graphicx}		%Grafika
\usepackage{wrapfig}		%Grafika obok textu
\usepackage{float}			%Allows H in fugire
\usepackage{hyperref}		%Allows hyperlinks
\pagestyle{empty} 			%usuwa nr strony
\usepackage{todonotes}		%Todo notatki
\usepackage{lipsum}         %Lorem text

\newgeometry{tmargin=2cm, bmargin=2cm, lmargin=2cm, rmargin=2cm} 

\newcommand\uwaga[1]{\textcolor{orange}{#1}}
\newcommand\TODO[1]{\textcolor{red}{#1}}

\begin{document}
	
	\begin{center}
		\LARGE Zadania na lekcje 4
	\end{center}
	\vspace{1.5cm}
	
	\begin{enumerate}[1.]
		\item Rozwiąż równania:
		
		\begin{enumerate}[a)] \begin{tabular}{p{7cm} p{7cm}} 
				\item $|3x-4|=7$ & \vspace{0.4cm} 	\item$2|x+6|=-4$ \\
				\item $|2x-3|-|3x+3|=x-6$ & \item $2-|x+4|=x^2+3x-2$ \\
				\item $x^2+4x+|x+2|=16$ & \item $x^2+2x+2=2|x+1|$ \\
				\item $(x+1)(|x|-1)=-0,5$ & \item $|x^2+2x+3|=|2x|$ \\
				\item $|x^2-4x+3|=1 $ & \item $x^2-7|x|+10\leq 10 $ \\
		\end{tabular} \end{enumerate}
	
		\item Wyznacz liczbę rozwiązań równania zależnie od parametru "m"
		
		$$|3-\frac{1}{x}|=m$$
		
		\item Wyznacz liczbę rozwiązań równania zależnie od parametru "m"
		
		$$|x+3|+|x-7|=m^2+m $$

	\end{enumerate}
	
	\newpage
	
	\begin{center}
		\LARGE Zadania domowe - lekcja 4
	\end{center}
	\vspace{1.5cm}
	
	\begin{enumerate}[1.]
		
	\item Rozwiąż równania: \todo[color=red]{todo}
	
	\begin{enumerate}[a)] \begin{tabular}{p{7cm} p{7cm}} 
			\item $||x+3|-4|=5$ & \vspace{0.4cm} 	\item$||x-1|-1|=|x-2|$ \\
			\item $ $ & \item $ $ \\
			\item $ $ & \item $ $ \\
	\end{tabular} \end{enumerate}
		
	\end{enumerate}
	
	
	
\end{document}