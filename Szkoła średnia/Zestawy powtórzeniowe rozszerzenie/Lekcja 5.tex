\documentclass[12pt,a4paper]{article}
\usepackage[utf8]{inputenc} %polskie znaki
\usepackage[T1]{fontenc}	%polskie znaki
\usepackage{amsmath}		%matematyczne znaczki :3
\usepackage{enumerate}		%Dodatkowe opcje do funkcji enumerate
\usepackage{geometry} 		%Ustawianie marginesow
\usepackage{graphicx}		%Grafika
\usepackage{wrapfig}		%Grafika obok textu
\usepackage{float}			%Allows H in fugire
\usepackage{hyperref}		%Allows hyperlinks
\pagestyle{empty} 			%usuwa nr strony
\usepackage{todonotes}		%Todo notatki
\usepackage{lipsum}         %Lorem text

\newgeometry{tmargin=2cm, bmargin=2cm, lmargin=2cm, rmargin=2cm} 

\newcommand\uwaga[1]{\textcolor{orange}{#1}}
\newcommand\TODO[1]{\textcolor{red}{#1}}

\begin{document}
	
	\begin{center}
		\LARGE Zadania na lekcje 5
	\end{center}
	\vspace{1.5cm}
	
	\begin{enumerate}[1.]
		\item Rozwiąż równania:

$$|x^2-x-2|=|x^2-2x-3|$$
		
	\end{enumerate}
	
	\newpage
	
	\begin{center}
		\LARGE Zadania domowe - lekcja 5
	\end{center}
	\vspace{1.5cm}
	
	\begin{enumerate}[1.]
		
		\item Rozwiąż równania: %\todo[color=red]{todo}
		

		
	\end{enumerate}
	
	
	
\end{document}